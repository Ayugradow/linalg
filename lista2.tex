\documentclass[a4paper,12pt]{article}
\usepackage{amsfonts,amsmath,amstext,amssymb,amsthm,color}
\usepackage[brazil,portuguese]{babel}
\usepackage[utf8]{inputenc}
\usepackage[T1]{fontenc}
\usepackage{enumerate}
\usepackage{pgf}
\usepackage{tikz,tikz-cd}









\usepackage{graphicx,graphpap}
\usetikzlibrary{calc,intersections,through,backgrounds,positioning,decorations.pathreplacing,decorations.markings}





%------- DEFINIÇÕES DE COMANDOS UTILIZADOS -----------------------------------
\def\R{\mathbb R}
\def\C{\mathbb C}
\def\N{\mathbb N}
\def\eqmod{\!\!\!\mod}
\def\herm#1{\langle #1\rangle}
\def\hmod#1{\parallel #1\parallel}
\newcommand{\dps}{\displaystyle}
\newcommand{\bn}{\bigskip\noindent}
\newcommand{\mb}{\mathbb}
\newcommand{\mc}{\mathcal}
\newcommand{\mf}{\mathfrak}
\newcommand{\mtt}{\mathtt}
\def\ang{{\rm ang}}
\def\id{{\rm id}}
\def\sen{{\rm sen\ }}
\def\diag{{\rm diag}}
\def\dist{{\rm dist}}
\def\sdo{\raisebox{.06cm}{$\bigcirc$\hspace{-0.38cm}\raisebox{0.0cm}{$\bot$}}\,}
\def\spen{{\rm span}}
\def\rectanglepath{-- ++(1cm,0cm) -- ++(0cm,1cm) -- ++(-1cm,0cm) -- cycle}
\newcommand{\del}{\partial}
\newcounter{exn}
\setcounter{exn}{1}
\newcommand{\exn}{\theexn\stepcounter{exn}}
\newcommand{\rin}{\rotatebox[origin=c]{-90}{\Large $\in$}}
\newcommand{\rsubset}{\rotatebox[origin=c]{-90}{\Large $\subset$}}
\newcommand{\Lsubset}{\rotatebox[origin=c]{0}{\Large $\subset$}}
\renewcommand{\hom}{\mathrm{Hom}}

\newenvironment{sol}{\noindent\normalsize {\sc Solução:}}






%------ Desenha ângulos retos no espaço ---------------------------------------------------------------------
%--- Parâmetros (A,B,C,t,s)
%--- sendo A, B, C pontos no epaço, e t e s números reais entre 0 e 1.
%--- t é a fração do segmento AB, e s é a fração do segmento BC utilizadas para contruir o quadradinho.
%------------------------------------------------------------------------------------------------------------
\newcommand\drawanguloreto[5]{
  \draw[-] ($#2 - #4*#2 + #4*#1$)  -- ($#2 - #4*#2 + #4*#1 - #5*#2 + #5*#3$) -- ($#2 - #5*#2 + #5*#3$);
}

\tikzset{
  % style to apply some styles to each segment of a path
  on each segment/.style={
    decorate,
    decoration={
      show path construction,
      moveto code={},
      lineto code={
        \path [#1]
        (\tikzinputsegmentfirst) -- (\tikzinputsegmentlast);
      },
      curveto code={
        \path [#1] (\tikzinputsegmentfirst)
        .. controls
        (\tikzinputsegmentsupporta) and (\tikzinputsegmentsupportb)
        ..
        (\tikzinputsegmentlast);
      },
      closepath code={
        \path [#1]
        (\tikzinputsegmentfirst) -- (\tikzinputsegmentlast);
      },
    },
  },
  % style to add an arrow in the middle of a path
  mid arrow/.style={postaction={decorate,decoration={
        markings,
        mark=at position .5 with {\arrow[#1]{stealth}}
      }}},
}
%------------------------------------------------------------------------------------------------------------




%--------  AJUSTANDO O TAMANHO DAS PÁGINAS -----------------------------------------------------
\addtolength{\textwidth}{4 cm}
\addtolength{\textheight}{3 cm}
\addtolength{\oddsidemargin}{-2 cm}
\addtolength{\evensidemargin}{-2 cm}
\addtolength{\topmargin}{-3 cm}



%-------- NUMERAÇÃO DE DEFINIÇÕES, TEOREMAS, ETC...  ---------------------------------------------
\newtheorem{df}{Definição}[subsection]
\newtheorem{thm}[df]{Teorema}
\newtheorem{cor}[df]{Corolário}
\newtheorem{prop}[df]{Proposição}
\newtheorem{lemma}[df]{Lema}
\newtheorem{conjec}[df]{Conjectura}
\newtheorem{exerc}[df]{Exercício(s)}
\newtheorem{qst}{Questão}[section]
\newtheorem{ex}[df]{Exemplo(s)}
\newtheorem{enunc}{Enunciado}[exn]

%--------  TÍTULO E DATA   ----------------------------------------------------------
\author{Lista de Exercícios - GAAL}
\date{Segunda Prova}
\title{}





%-------------------------------------------------------------------------------------------
%-------------------------------------------------------------------------------------------
%-----    INÍCIO DO TEXTO   ----------------------------------------------------------------
%-------------------------------------------------------------------------------------------
%-------------------------------------------------------------------------------------------
\begin{document}
\maketitle

\section{Transformações lineares}

\begin{qst}
	Verifique se as funções de $\R^2$ em $\R^2$ abaixo são ou não lineares:
	\begin{enumerate}[a)]
		\item $f(x,y)=(y,x)$;
		\item $f(x,y)=(x+1,2y)$;
		\item $f(x,y)=(x,y)$;
		\item $f(x,y)=\left(\dfrac{x+y}{2},x-y\right)$;
		\item $f(x,y)=(x/y,y/x)$.
	\end{enumerate}
\end{qst}

\begin{qst}
	Verifique se as funções de $\R^3$ em $\R^3$ abaixo são ou não lineares:
	\begin{enumerate}[a)]
		\item $f(x,y,z)=(x,y,0)$;
		\item $f(x,y,z)=(z,y,x)$;
		\item $f(x,y,z)=(2\pi x,2x,\pi x)$;
		\item $f(x,y,z)=(x+y,y+z,x+z)$;
		\item $f(x,y,z)=({\rm sen} x, {\rm cos} y, {\rm tg} z)$.
	\end{enumerate}
\end{qst}

\begin{qst}
	Se $f$ e $g$ são duas funções lineares (de $\R^2$ em $\R^2$ ou de $\R^3$ em $\R^3$), podemos concluir que $f\circ g$ e $g\circ f$ também são lineares?
\end{qst}
\begin{qst}
	Considere $f,g$ duas funções lineares em $\R^2$. Defina a função $f+g:\R^2\to \R^2$ dada por $(f+g)(x,y):=f(x,y)+g(x,y)$. Essa função é linear? E se $f,g$ fossem funções lineares em $\R^3$?
\end{qst}
\pagebreak
\section{Produtos interno e vetorial}
\begin{qst}
	Calcule o produto interno dos vetores abaixo:
	\begin{enumerate}[a)]
		\item $(1,1)$ e $(2,3)$;
		\item $(\pi, 2)$ e $(2, \pi)$;
		\item $(0,7)$ e $(8,1)$;
		\item $(1,1,1)$ e $(\pi,\pi, 4)$;
		\item $(0,7,2)$ e $(3,6,5)$;
		\item $(1,2)$ e $(-2, 1)$;
		\item $(2,4)$ e $(16, -8)$;
		\item $(1,-1, 2)$ e $(1,-1,-1)$.
	\end{enumerate}
\end{qst}
\begin{qst}
	Dos itens acima, quais são pares de vetores ortogonais? Por que?
\end{qst}

\begin{qst}
	Exiba um vetor ortogonal a $(2,-7)$.
\end{qst}
\begin{qst}
	Calcule o produto vetorial dos vetores abaixo:
	\begin{enumerate}[a)]
		\item $(1,2,3)$ e $(2,4,6)$;
		\item $(2,5,3)$ e $(1,1,1)$;
		\item $(0,0,1)$ e $(3,4,8)$;
		\item $(8,8,8)$ e $(2,3,4)$.
	\end{enumerate}
\end{qst}
\begin{qst}
	Exiba um vetor que seja simultaneamente ortogonal a $(0,0,1)$ e $(3,4,8)$.
\end{qst}
\pagebreak
\section{Subespaços}
\begin{qst}
	Verifique se os conjuntos abaixo são subespaços:
	\begin{enumerate}[a)]
		\item $\{(x,y)\in\R^2\mid x=0\}$;
		\item $\{v\in\R^2\mid 2v=0\}$;
		\item $\{v\in \R^2\mid v=0\}$;
		\item $\{(x,y)\in \R^2\mid x+y=2 \mbox{ ou }x+y=4$;
		\item $\{(x,y)\in \R^2\mid y=x^2\}$;
		\item $\{(x,y)\in \R^2\mid 3x+2y=7\}$;
		\item $\{(x,y)\in \R^2\mid x+y=x-y\}$;
		\item $\{(x,y,z)\in \R^3\mid x+y+z=1\}$;
		\item $\{(x,y,z)\in \R^3\mid x+y=2,x+z=5 \mbox{ e } y+z=3\}$;
		\item $\{v\in \R^3\mid v\perp(1,1,1)\}$;
		\item $\{v\in \R^3\mid v=\lambda(2,3,1),\mbox{ para algum }\lambda\in \R\}$;
		\item $\{v\in \R^3\mid v\times (1,1,1)=0\}$;
		\item $\{v\in \R^3\mid \langle v,(2,3,2)\rangle = 9\}$.
	\end{enumerate}
\end{qst}
\pagebreak
\section{Geradores}
\begin{qst}
	Determine os subespaços gerados pelos seguintes vetores:
	\begin{enumerate}[a)]		
		\item $\{(1,0),(0,1)\}$;
		\item $\{(1,1),(1,-1)\}$;
		\item $\{(2,3),(-2,-3)\}$;
		\item $\{(0,0)\}$;
		\item $\{(1,1), (2,3),(4,-2)\}$;
		\item $\{(1,5), (4,20), (-3,-15)\}$;
		\item $\{(1,1,1),(1,1,-1),(7,7,0)\}$;
		\item $\{(1,0,-1),(-1,0,-2), (0,0,1)\}$;
		\item $\{(1,0,0)\}$;
		\item $\{(1,0,6),(2,3,4)\}$;
		\item $\{(1,0,3),(1/3,0,1)\}$;
		\item $\{(1,1,1),(2,3,4),(0,0,1),(0,9,7)\}$.
	\end{enumerate}
\end{qst}

\begin{qst}
	Dados dos subespaços abaixo, exiba um possível conjunto de geradores para cada um deles:
	\begin{enumerate}[a)]
		\item O plano que contém os vetores $(1,0,1)$ e $(0,0,1)$;
		\item A reta em $\R^2$ que contém o vetor $(2,5)$;
		\item O plano ortogonal ao vetor $(1,1,1)$;
		\item A reta em $\R^2$ ortogonal ao vetor $(1,1)$;
		\item A reta em $\R^3$ que contém o vetor $(2,1,-5)$;
		\item Todo o $\R^3$.
	\end{enumerate}
\end{qst}
\pagebreak
\section{L.D. e L.I.}
\begin{qst}
	Verifique se os conjuntos abaixo são l.d. ou l.i.:
	\begin{enumerate}[a)]
		\item $\{(1,0)\}$;
		\item $\{(1,0),(0,1),(2,5)\}$;
		\item $\{(1,1),(2,3)\}$;
		\item $\{(1,-1),(3,-2),(2,2)\}$;
		\item $\{(1,1,1),(1,1,-1)\}$;
		\item $\{(1,7,6),(3,21,18)\}$;
		\item $\{(1,1,1), (1,2,3), (2,5,3), (7,-2,3)\}$;
		\item $\{(1,0,1),(1,1,0),(0,1,1)\}$.
	\end{enumerate}
\end{qst}
\begin{qst}
	Verifique que o conjunto $\{(1,1), (2,3),(3,5), (4,-2)\}$ é l.d. Em seguida, descreva quais vetores você poderia tirar desse conjunto de forma a torná-lo l.i.
\end{qst}
\begin{qst}
	Verifique que o conjunto $\{(1,0,1),(2,3,2)\}$ é l.i. Em seguida, apresente um novo vetor que seja l.i. com o conjunto anterior, mas que também forme, junto com esse conjunto, um conjunto de geradores para $\R^3$ (ou seja, encontre $v$ tal que $\{(1,0,1),(2,3,2),v\}$ é l.i. e gera $\R^3$).
\end{qst}
\pagebreak
\section{Núcleo e Imagem}
\begin{qst}
	Para cada função linear da seção 1 desta lista, exiba seu núcleo e imagem. Em seguida, calcule um conjunto de geradores l.i. para o núcleo e a imagem, e conclua calculando as dimensões de cada um.
\end{qst}
\end{document}