\documentclass[a4paper,10pt]{article}
\usepackage{amsfonts,amsmath,amstext,amssymb,amsthm,color}
\usepackage{thmtools}
\usepackage[brazil,portuguese]{babel}
\usepackage[utf8]{inputenc}
\usepackage[T1]{fontenc}
\usepackage{enumerate}
\usepackage{pgf}
\usepackage{tikz,tikz-cd}
\usepackage{arydshln}
\usepackage{tcolorbox}
\usepackage[bottom=3cm]{geometry}
\usepackage{mathrsfs}
\usetikzlibrary{arrows}








\tcbuselibrary{breakable}
\usepackage{graphicx,graphpap}
\usetikzlibrary{calc,intersections,through,backgrounds,positioning,decorations.pathreplacing,decorations.markings}





%------- DEFINIÇÕES DE COMANDOS UTILIZADOS -----------------------------------
\def\R{\mathbb R}
\def\C{\mathbb C}
\def\N{\mathbb N}
\def\eqmod{\!\!\!\mod}
\def\herm#1{\langle #1\rangle}
\def\hmod#1{\parallel #1\parallel}
\newcommand{\dps}{\displaystyle}
\newcommand{\bn}{\bigskip\noindent}
\newcommand{\mb}{\mathbb}
\newcommand{\mc}{\mathcal}
\newcommand{\mf}{\mathfrak}
\newcommand{\mtt}{\mathtt}
\def\ang{{\rm ang}}
\def\id{{\rm id}}
\def\sen{{\rm sen\ }}
\def\diag{{\rm diag}}
\def\dist{{\rm dist}}
\def\sdo{\raisebox{.06cm}{$\bigcirc$\hspace{-0.38cm}\raisebox{0.0cm}{$\bot$}}\,}
\def\spen{{\rm span}}
\def\rectanglepath{-- ++(1cm,0cm) -- ++(0cm,1cm) -- ++(-1cm,0cm) -- cycle}
\newcommand{\del}{\partial}
\newcounter{exn}
\setcounter{exn}{1}
\newcommand{\exn}{\theexn\stepcounter{exn}}
\newcommand{\rin}{\rotatebox[origin=c]{-90}{\Large $\in$}}
\newcommand{\rsubset}{\rotatebox[origin=c]{-90}{\Large $\subset$}}
\newcommand{\Lsubset}{\rotatebox[origin=c]{0}{\Large $\subset$}}
\renewcommand{\hom}{\mathrm{Hom}}

\newenvironment{sol}{\begin{tcolorbox}[breakable,colback=blue!5!white,colframe=blue!40!white,title=\normalsize {\sc{Solução}},coltitle=black]}{\end{tcolorbox}}

\newenvironment{augmatrix}{\left(\begin{array}}{\end{array}\right)}

\DeclareMathOperator{\Ker}{Ker}
\DeclareMathOperator{\im}{Im}






%------ Desenha ângulos retos no espaço ---------------------------------------------------------------------
%--- Parâmetros (A,B,C,t,s)
%--- sendo A, B, C pontos no epaço, e t e s números reais entre 0 e 1.
%--- t é a fração do segmento AB, e s é a fração do segmento BC utilizadas para contruir o quadradinho.
%------------------------------------------------------------------------------------------------------------
\newcommand\drawanguloreto[5]{
  \draw[-] ($#2 - #4*#2 + #4*#1$)  -- ($#2 - #4*#2 + #4*#1 - #5*#2 + #5*#3$) -- ($#2 - #5*#2 + #5*#3$);
}

\tikzset{
  % style to apply some styles to each segment of a path
  on each segment/.style={
    decorate,
    decoration={
      show path construction,
      moveto code={},
      lineto code={
        \path [#1]
        (\tikzinputsegmentfirst) -- (\tikzinputsegmentlast);
      },
      curveto code={
        \path [#1] (\tikzinputsegmentfirst)
        .. controls
        (\tikzinputsegmentsupporta) and (\tikzinputsegmentsupportb)
        ..
        (\tikzinputsegmentlast);
      },
      closepath code={
        \path [#1]
        (\tikzinputsegmentfirst) -- (\tikzinputsegmentlast);
      },
    },
  },
  % style to add an arrow in the middle of a path
  mid arrow/.style={postaction={decorate,decoration={
        markings,
        mark=at position .5 with {\arrow[#1]{stealth}}
      }}},
}
%------------------------------------------------------------------------------------------------------------




%--------  AJUSTANDO O TAMANHO DAS PÁGINAS -----------------------------------------------------
\addtolength{\textwidth}{4 cm}
\addtolength{\textheight}{3 cm}
\addtolength{\oddsidemargin}{-2 cm}
\addtolength{\evensidemargin}{-2 cm}
\addtolength{\topmargin}{-3 cm}



%-------- NUMERAÇÃO DE DEFINIÇÕES, TEOREMAS, ETC...  ---------------------------------------------
\newtheorem{qst}{Questão}
\declaretheorem[numbered=no, name=Questão Bônus]{bonus}
\newtheorem{enunc}{Enunciado}[exn]

%--------  TÍTULO E DATA   ----------------------------------------------------------
\author{Segunda Prova - GAAL}
\date{04 de Abril de 2019}
\title{}





%-------------------------------------------------------------------------------------------
%-------------------------------------------------------------------------------------------
%-----    INÍCIO DO TEXTO   ----------------------------------------------------------------
%-------------------------------------------------------------------------------------------
%-------------------------------------------------------------------------------------------
\begin{document}
\begin{center}
	{\Large{\sc Segunda Prova - GAAL}}
\end{center}

	
Em todas as questões abaixo, sempre que encontrar uma solução você deve mostrar que ela é, de fato, uma solução.

\begin{qst}
	Considere a matriz real $A=\begin{pmatrix}
	1 & -1\\-2&2
	\end{pmatrix}$ e faça o que se pede.
	\begin{enumerate}[a)]
		\item Calcule $\Ker A$ e $\im A$ e faça um esboço gráfico desses subespaços de $\R^2$.
		\item Exiba (se possível) um conjunto de geradores l.i. para $\Ker A$ e para $\im A$.
		\item Sem fazer contas, exiba o conjunto solução do sistema
		\[\begin{pmatrix}
		3&-3\\-6&6
		\end{pmatrix}\begin{pmatrix}
		x\\y
		\end{pmatrix}=\begin{pmatrix}
		6\\9
		\end{pmatrix}.\] 
	\end{enumerate}
\end{qst}

\begin{qst}
	Considere as retas em $\R^3$ dadas por $$q=\{v\in\R^3\mid v=\lambda (1,1,1)+(1,2,3)\}$$ $$r=\{v\in\R^3\mid v=\lambda (1,0,1)+(2,2,1)\}$$ $$s=\{v\in\R^3\mid v=\lambda (5,5,5)\}$$$$t=\{v\in\R^3\mid v=\lambda (-1,0,-1)+(-2,1,0)\}$$ e faça o que se pede:
	\begin{enumerate}[a)]
		\item Classifique-as quanto a paralelas, concorrentes ou reversas.
		\item Calcule as interseções dos pares de retas que forem concorrentes.
	\end{enumerate}
\end{qst}

\begin{qst}
	Considere os planos $$\pi_1=\{v\in \R^3\mid v=\lambda(1,1,1)+\mu(1,0,-1),\mbox{ com }\lambda\mbox{ e }\mu\in \R\}$$ $$\pi_2=\{(x,y,z)\in\R^3\mid x-2y+7z=0\}.$$
	\begin{enumerate}[a)]
		\item Calcule a interseção desses planos.
		\item Exiba um conjunto de geradores l.i. dessa interseção.
	\end{enumerate}
\end{qst}

\begin{qst}
	Considere a matriz real $2\times 2$ $A$ cujo núcleo é a reta $y=2x$ e cuja imagem é o eixo $Y$.
	\begin{enumerate}[a)]
		\item Exiba um conjunto de geradores $l.i.$ para o núcleo e a imagem de $A$.
		\item Responda, sem fazer contas: $A$ é invertível? Justifique sua resposta.
		\item Exiba uma possível expressão para $A$ que seja compatível com o enunciado.
	\end{enumerate}
\end{qst}

\begin{bonus}
	Considere $v,u\in\R^2$ vetores não-nulos arbitrário e faça o que se pede:
	\begin{enumerate}[a)]		
		\item Faça um esboço gráfico de $v$, $u$, $v+u$ e $v-u$.
		\item Calcule $\lVert v-u\rVert^2$.
		\item Compare o resultado acima com a Lei dos Cossenos (ver abaixo).
		\item Obtenha uma fórmula que relacione o ângulo $\theta$ entre $v$ e $u$ e o produto interno $\langle v,u\rangle$.
		\item Conclua mostrando que $v$ e $u$ são l.i. se, e somente se, $\cos\theta\neq 0$.
	\end{enumerate}
\end{bonus}

\textbf{\underline{Lei dos Cossenos:}}

Qualquer triângulo
\definecolor{qqwuqq}{rgb}{0.,0.39215686274509803,0.}
\definecolor{zzttqq}{rgb}{0.6,0.2,0.}
\definecolor{ududff}{rgb}{0.30196078431372547,0.30196078431372547,1.}
\begin{tikzpicture}[line cap=round,line join=round,>=triangle 45,x=.2cm,y=.2cm]
\clip(2.02,-1.86) rectangle (10.68,4.74);
\draw [shift={(2.48,-0.34)}] (0,0) -- (-2.0972181819517335:0.6) arc (-2.0972181819517335:56.19204008995808:0.6) -- cycle;
\draw  (5.48,4.14)-- (2.48,-0.34);
\draw  (2.48,-0.34)-- (9.58,-0.6);
\draw  (9.58,-0.6)-- (5.48,4.14);
\begin{scriptsize}
\draw (3.76,2.25) node {$c$};
\draw (6.06,-0.8) node {$b$};
\draw (7.82,2.15) node {$a$};
\draw (3.52,0.07) node {$\theta$};
\end{scriptsize}
\end{tikzpicture}satisfaz a seguinte relação: $a^2=b^2+c^2-2bc\cos\theta$.

\end{document}