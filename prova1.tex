\documentclass[a4paper,10pt]{article}
\usepackage{amsfonts,amsmath,amstext,amssymb,amsthm,color}
\usepackage{thmtools}
\usepackage[brazil,portuguese]{babel}
\usepackage[utf8]{inputenc}
\usepackage[T1]{fontenc}
\usepackage{enumerate}
\usepackage{pgf}
\usepackage{tikz,tikz-cd}
\usepackage{arydshln}
\usepackage{tcolorbox}
\usepackage[bottom=3cm]{geometry}








\tcbuselibrary{breakable}
\usepackage{graphicx,graphpap}
\usetikzlibrary{calc,intersections,through,backgrounds,positioning,decorations.pathreplacing,decorations.markings}





%------- DEFINIÇÕES DE COMANDOS UTILIZADOS -----------------------------------
\def\R{\mathbb R}
\def\C{\mathbb C}
\def\N{\mathbb N}
\def\eqmod{\!\!\!\mod}
\def\herm#1{\langle #1\rangle}
\def\hmod#1{\parallel #1\parallel}
\newcommand{\dps}{\displaystyle}
\newcommand{\bn}{\bigskip\noindent}
\newcommand{\mb}{\mathbb}
\newcommand{\mc}{\mathcal}
\newcommand{\mf}{\mathfrak}
\newcommand{\mtt}{\mathtt}
\def\ang{{\rm ang}}
\def\id{{\rm id}}
\def\sen{{\rm sen\ }}
\def\diag{{\rm diag}}
\def\dist{{\rm dist}}
\def\sdo{\raisebox{.06cm}{$\bigcirc$\hspace{-0.38cm}\raisebox{0.0cm}{$\bot$}}\,}
\def\spen{{\rm span}}
\def\rectanglepath{-- ++(1cm,0cm) -- ++(0cm,1cm) -- ++(-1cm,0cm) -- cycle}
\newcommand{\del}{\partial}
\newcounter{exn}
\setcounter{exn}{1}
\newcommand{\exn}{\theexn\stepcounter{exn}}
\newcommand{\rin}{\rotatebox[origin=c]{-90}{\Large $\in$}}
\newcommand{\rsubset}{\rotatebox[origin=c]{-90}{\Large $\subset$}}
\newcommand{\Lsubset}{\rotatebox[origin=c]{0}{\Large $\subset$}}
\renewcommand{\hom}{\mathrm{Hom}}

\newenvironment{sol}{\begin{tcolorbox}[breakable,colback=blue!5!white,colframe=blue!40!white,title=\normalsize {\sc{Solução}},coltitle=black]}{\end{tcolorbox}}

\newenvironment{augmatrix}{\left(\begin{array}}{\end{array}\right)}






%------ Desenha ângulos retos no espaço ---------------------------------------------------------------------
%--- Parâmetros (A,B,C,t,s)
%--- sendo A, B, C pontos no epaço, e t e s números reais entre 0 e 1.
%--- t é a fração do segmento AB, e s é a fração do segmento BC utilizadas para contruir o quadradinho.
%------------------------------------------------------------------------------------------------------------
\newcommand\drawanguloreto[5]{
  \draw[-] ($#2 - #4*#2 + #4*#1$)  -- ($#2 - #4*#2 + #4*#1 - #5*#2 + #5*#3$) -- ($#2 - #5*#2 + #5*#3$);
}

\tikzset{
  % style to apply some styles to each segment of a path
  on each segment/.style={
    decorate,
    decoration={
      show path construction,
      moveto code={},
      lineto code={
        \path [#1]
        (\tikzinputsegmentfirst) -- (\tikzinputsegmentlast);
      },
      curveto code={
        \path [#1] (\tikzinputsegmentfirst)
        .. controls
        (\tikzinputsegmentsupporta) and (\tikzinputsegmentsupportb)
        ..
        (\tikzinputsegmentlast);
      },
      closepath code={
        \path [#1]
        (\tikzinputsegmentfirst) -- (\tikzinputsegmentlast);
      },
    },
  },
  % style to add an arrow in the middle of a path
  mid arrow/.style={postaction={decorate,decoration={
        markings,
        mark=at position .5 with {\arrow[#1]{stealth}}
      }}},
}
%------------------------------------------------------------------------------------------------------------




%--------  AJUSTANDO O TAMANHO DAS PÁGINAS -----------------------------------------------------
\addtolength{\textwidth}{4 cm}
\addtolength{\textheight}{3 cm}
\addtolength{\oddsidemargin}{-2 cm}
\addtolength{\evensidemargin}{-2 cm}
\addtolength{\topmargin}{-3 cm}



%-------- NUMERAÇÃO DE DEFINIÇÕES, TEOREMAS, ETC...  ---------------------------------------------
\newtheorem{qst}{Questão}
\declaretheorem[numbered=no, name=Questão Bônus]{bonus}
\newtheorem{enunc}{Enunciado}[exn]

%--------  TÍTULO E DATA   ----------------------------------------------------------
\author{Primeira Prova - GAAL}
\date{04 de Abril de 2019}
\title{}





%-------------------------------------------------------------------------------------------
%-------------------------------------------------------------------------------------------
%-----    INÍCIO DO TEXTO   ----------------------------------------------------------------
%-------------------------------------------------------------------------------------------
%-------------------------------------------------------------------------------------------
\begin{document}
\begin{center}
	{\Large{\sc Primeira Prova - GAAL}}
\end{center}


Em todas as questões abaixo, sempre que encontrar uma solução você deve mostrar que ela é, de fato, uma solução.

\begin{qst}
	Considere o sistema linear
	\[\begin{pmatrix}
	3&0&-1&0\\4&1&-1&-2\\0&1&0&-1
	\end{pmatrix}\begin{pmatrix}
	x\\y\\z\\w
	\end{pmatrix}=\begin{pmatrix}
	0\\0\\0
	\end{pmatrix}\]e faça o que se pede:
	\begin{enumerate}[a)]
		\item Exiba a matriz aumentada desse sistema.
		\item Exiba a matriz escalonada desse sistema.
		\item Calcule o conjunto solução desse sistema.
		\item Exiba uma solução do sistema, e verifique que ela é, de fato, uma solução.
	\end{enumerate}
\end{qst}

\begin{qst}
	Considere a matriz $A=\begin{pmatrix}
	1&1&1\\2&5&-2\\1&7&-7
	\end{pmatrix}$ e faça o que se pede:
	\begin{enumerate}[a)]
		\item Resolva o sistema linear homogêneo $AX=0$.
		\item Dada a matriz coluna $X_1=\begin{pmatrix}
		1\\1\\1
		\end{pmatrix}$ encontre uma matriz $B$ tal que $X_1$ seja solução do sistema $AX=B$.
		\item Encontre, sem fazer contas, outra solução qualquer do sistema $AX=B$ e mostre que ela é solução.
	\end{enumerate}
\end{qst}

\begin{qst}
	Dadas as matrizes $A=\begin{pmatrix}
	1&2&3\\1&1&2\\0&1&2
	\end{pmatrix}$ e $B=\begin{pmatrix}
	5\\7\\-2
	\end{pmatrix}$, faça o que se pede:
	\begin{enumerate}[a)]
		\item Calcule $A^{-1}$.
		\item Calcule $A^{-1}B$.
		\item Exiba o conjunto solução do sistema linear $AX=B$.
	\end{enumerate}
\end{qst}

\begin{qst}
	Dadas as matrizes $A=\begin{pmatrix}
	1&1\\0&1
	\end{pmatrix}$ e $B=\begin{pmatrix}
	0&1\\1&0
	\end{pmatrix}$, faça o que se pede:
	
	\begin{enumerate}[a)]
		\item Calcule $AB$ e $BA$ e compare os resultados obtidos.
		\item Escolha $A$ ou $B$. Você seria capaz de encontrar uma matriz $C$ que comute com a matriz que você escolheu (por exemplo, se você escolher a matriz $A$, uma matriz $C$ tal que $AC=CA$)? Explique seu raciocínio.
	\end{enumerate}
\end{qst}

\begin{bonus}
	Considere a reação
	\[x{\rm Fe_3O_4}+y{\rm CO}\rightarrow z{\rm FeO}+w{\rm CO_2},\]em que $x,y,z,w\in \R$ são coeficientes de balanceamento, e faça o que se pede:
	\begin{enumerate}[a)]
		\item Escreva as equações que relacionam as quantidades de cada substância na reação - uma equação para o ferro, uma equação para o oxigênio e uma equação para o carbono (por exemplo, como temos $3x$ átomos de ferro do lado esquerdo e $z$ átomos de ferro do lado direito, isso nos dá a equação $3x=z$).
		\item Use as equações que obtidas no item anterior para montar um sistema linear homogêneo.
		\item Resolva o sistema obtido no item anterior.
		\item Escolha uma solução não-trivial do sistema e verifique que ela é um balanceamento da reação.
		\item Conclua descrevendo um procedimento sistemático para balanceamento de reações.
	\end{enumerate}
\end{bonus}
\end{document}