\documentclass[a4paper,12pt]{article}
\usepackage{amsfonts,amsmath,amstext,amssymb,amsthm,color}
\usepackage[brazil,portuguese]{babel}
\usepackage[utf8]{inputenc}
\usepackage[T1]{fontenc}
\usepackage{enumerate}
\usepackage{pgf}
\usepackage{tikz,tikz-cd}









\usepackage{graphicx,graphpap}
\usetikzlibrary{calc,intersections,through,backgrounds,positioning,decorations.pathreplacing,decorations.markings}





%------- DEFINIÇÕES DE COMANDOS UTILIZADOS -----------------------------------
\def\R{\mathbb R}
\def\C{\mathbb C}
\def\N{\mathbb N}
\def\eqmod{\!\!\!\mod}
\def\herm#1{\langle #1\rangle}
\def\hmod#1{\parallel #1\parallel}
\newcommand{\dps}{\displaystyle}
\newcommand{\bn}{\bigskip\noindent}
\newcommand{\mb}{\mathbb}
\newcommand{\mc}{\mathcal}
\newcommand{\mf}{\mathfrak}
\newcommand{\mtt}{\mathtt}
\def\ang{{\rm ang}}
\def\id{{\rm id}}
\def\sen{{\rm sen\ }}
\def\diag{{\rm diag}}
\def\dist{{\rm dist}}
\def\sdo{\raisebox{.06cm}{$\bigcirc$\hspace{-0.38cm}\raisebox{0.0cm}{$\bot$}}\,}
\def\spen{{\rm span}}
\def\rectanglepath{-- ++(1cm,0cm) -- ++(0cm,1cm) -- ++(-1cm,0cm) -- cycle}
\newcommand{\del}{\partial}
\newcounter{exn}
\setcounter{exn}{1}
\newcommand{\exn}{\theexn\stepcounter{exn}}
\newcommand{\rin}{\rotatebox[origin=c]{-90}{\Large $\in$}}
\newcommand{\rsubset}{\rotatebox[origin=c]{-90}{\Large $\subset$}}
\newcommand{\Lsubset}{\rotatebox[origin=c]{0}{\Large $\subset$}}
\newcommand{\pint}[2]{\langle #1,#2\rangle}
	
\renewcommand{\hom}{\mathrm{Hom}}

\newenvironment{sol}{\noindent\normalsize {\sc Solução:}}

\DeclareMathOperator{\Ker}{Ker}
\DeclareMathOperator{\im}{Im}






%------ Desenha ângulos retos no espaço ---------------------------------------------------------------------
%--- Parâmetros (A,B,C,t,s)
%--- sendo A, B, C pontos no epaço, e t e s números reais entre 0 e 1.
%--- t é a fração do segmento AB, e s é a fração do segmento BC utilizadas para contruir o quadradinho.
%------------------------------------------------------------------------------------------------------------
\newcommand\drawanguloreto[5]{
  \draw[-] ($#2 - #4*#2 + #4*#1$)  -- ($#2 - #4*#2 + #4*#1 - #5*#2 + #5*#3$) -- ($#2 - #5*#2 + #5*#3$);
}

\tikzset{
  % style to apply some styles to each segment of a path
  on each segment/.style={
    decorate,
    decoration={
      show path construction,
      moveto code={},
      lineto code={
        \path [#1]
        (\tikzinputsegmentfirst) -- (\tikzinputsegmentlast);
      },
      curveto code={
        \path [#1] (\tikzinputsegmentfirst)
        .. controls
        (\tikzinputsegmentsupporta) and (\tikzinputsegmentsupportb)
        ..
        (\tikzinputsegmentlast);
      },
      closepath code={
        \path [#1]
        (\tikzinputsegmentfirst) -- (\tikzinputsegmentlast);
      },
    },
  },
  % style to add an arrow in the middle of a path
  mid arrow/.style={postaction={decorate,decoration={
        markings,
        mark=at position .5 with {\arrow[#1]{stealth}}
      }}},
}
%------------------------------------------------------------------------------------------------------------




%--------  AJUSTANDO O TAMANHO DAS PÁGINAS -----------------------------------------------------
\addtolength{\textwidth}{4 cm}
\addtolength{\textheight}{3 cm}
\addtolength{\oddsidemargin}{-2 cm}
\addtolength{\evensidemargin}{-2 cm}
\addtolength{\topmargin}{-3 cm}



%-------- NUMERAÇÃO DE DEFINIÇÕES, TEOREMAS, ETC...  ---------------------------------------------
\newtheorem{df}{Definição}[subsection]
\newtheorem{thm}[df]{Teorema}
\newtheorem{cor}[df]{Corolário}
\newtheorem{prop}[df]{Proposição}
\newtheorem{lemma}[df]{Lema}
\newtheorem{conjec}[df]{Conjectura}
\newtheorem{exerc}[df]{Exercício(s)}
\newtheorem{qst}{Questão}
\newtheorem{ex}[df]{Exemplo(s)}
\newtheorem{enunc}{Enunciado}[exn]

%--------  TÍTULO E DATA   ----------------------------------------------------------
\author{Prova Suplementar - GAAL}
\date{02 de Julho de 2019}
\title{}





%-------------------------------------------------------------------------------------------
%-------------------------------------------------------------------------------------------
%-----    INÍCIO DO TEXTO   ----------------------------------------------------------------
%-------------------------------------------------------------------------------------------
%-------------------------------------------------------------------------------------------
\begin{document}
\maketitle

Caso tenha perdido alguma prova, faça somente as questões referentes à prova perdida.

Caso tenha perdido mais de uma prova ou deseje substituir uma prova, escolha uma das provas abaixo e faça somente ela.

Quem fizer questões de mais de uma prova terá sua prova \textbf{anulada}.

Em todas as questões abaixo, sempre que encontrar uma solução você deve mostrar que ela é, de fato, uma solução.

\section*{Primeira Prova}

\begin{qst}
	Considere a matriz $A=\begin{pmatrix}
	1&2&3&1\\1&3&0&1\\1&0&2&1
	\end{pmatrix}$ e faça o que se pede:
	\begin{enumerate}[a)]
		\item Resolva o sistema linear homogêneo $AX=0$.
		\item Exiba uma solução particular do sistema linear homogêneo $AX=0$.
		\item Dada a matriz coluna $X_1=\begin{pmatrix}
		1\\1\\1\\-2
		\end{pmatrix}$ encontre uma matriz $B$ tal que $X_1$ seja solução do sistema $AX=B$.
		\item Encontre, sem fazer contas, outra solução qualquer do sistema $AX=B$ e mostre que ela é solução.
	\end{enumerate}
\end{qst}

\begin{qst}
	Sejam $A,B,P\in M_n(\R)$ matrizes reais quadradas $n\times n$ tais que $A=PBP^{-1}$. Mostre que $A$ é inversível se, e somente se, $B$ é inversível (dica: Uma matriz $T$ é inversível se, e somente se, $\det T\neq 0$).
\end{qst}
\pagebreak
\section*{Segunda Prova}
\setcounter{qst}{0}
\begin{qst}
	Considere os planos $$\pi_1=\{v\in \R^3\mid v=\lambda(1,1,1)+\mu(1,-1,1),\mbox{ com }\lambda\mbox{ e }\mu\in \R\}$$ $$\pi_2=\{(x,y,z)\in\R^3\mid x-2y+7z=0\}.$$
	\begin{enumerate}[a)]
		\item Calcule a interseção desses planos.
		\item Exiba um conjunto de geradores l.i. dessa interseção.
	\end{enumerate}
\end{qst}

\begin{qst}
	\begin{enumerate}[a)]
		\item Explique em palavras o que é um conjunto de vetores l.d. e l.i.
		\item Explique em palavras o que significa um subespaço ser gerado por uma coleção de vetores e o que é um conjunto de geradores para um subespaço.
		\item Verifique se o conjunto $\{(1,4,2), (1,1,1), (2,-1,1)\}\subseteq\R^3$ é l.d. ou l.i. e se é ou não um conjunto de geradores de $\R^3$. Faça o mesmo para o conjunto $\{(1,-1),(4,3)\}\subseteq\R^2$.
	\end{enumerate}
\end{qst}

\section*{Terceira Prova}
\setcounter{qst}{0}
\begin{qst}	
	Dada a equação \[x^2-6xy-7y^2+10x+2y+9=0\]faça o que se pede:
	\begin{enumerate}[a)]
		\item Reescreva equação acima em forma matricial.
		\item Encontre os autovalores e autovetores associados à parte de grau 2 da equação acima.
		\item Identifique a matriz de rotação associada a essa equação.
		\item Faça a mudança de coordenadas que diagonaliza a forma matricial obtida no item (a).
		\item Identifique a translação associada a essa equação.
		\item Faça uma translação de forma que a equação obtida no item (d) esteja na forma padrão.
		\item Identifique qual a cônica descrita pela equação.
		\item Faça um esboço dessa cônica.
	\end{enumerate}
\end{qst}

\begin{qst}
	\begin{enumerate}[a)]
		\item Explique, em palavras, o que os autovetores e autovalores de uma matriz quadrada nos dizem sobre a matrix.
		\item Calcule os autovetores e autovalores da matriz $A=\begin{pmatrix}
		3&1\\1&4
		\end{pmatrix}$.
		\item Considere $C=\{v\in \R^2\mid \lVert v\rVert = 1\}$, ou seja, $C$ é o círculo de raio 1 centrado na origem. Esboce a imagem de $C$ quando aplicamos a matriz $A$ acima a cada um de seus pontos.
	\end{enumerate}
\end{qst}

\end{document}