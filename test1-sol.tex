\documentclass[a4paper,12pt]{article}
\usepackage{amsfonts,amsmath,amstext,amssymb,amsthm,color}
\usepackage[brazil,portuguese]{babel}
\usepackage[utf8]{inputenc}
\usepackage[T1]{fontenc}
\usepackage{enumerate}
\usepackage{pgf}
\usepackage{tikz,tikz-cd}
\usepackage{arydshln}
\usepackage{tcolorbox}








\tcbuselibrary{breakable}
\usepackage{graphicx,graphpap}
\usetikzlibrary{calc,intersections,through,backgrounds,positioning,decorations.pathreplacing,decorations.markings}





%------- DEFINIÇÕES DE COMANDOS UTILIZADOS -----------------------------------
\def\R{\mathbb R}
\def\C{\mathbb C}
\def\N{\mathbb N}
\def\eqmod{\!\!\!\mod}
\def\herm#1{\langle #1\rangle}
\def\hmod#1{\parallel #1\parallel}
\newcommand{\dps}{\displaystyle}
\newcommand{\bn}{\bigskip\noindent}
\newcommand{\mb}{\mathbb}
\newcommand{\mc}{\mathcal}
\newcommand{\mf}{\mathfrak}
\newcommand{\mtt}{\mathtt}
\def\ang{{\rm ang}}
\def\id{{\rm id}}
\def\sen{{\rm sen\ }}
\def\diag{{\rm diag}}
\def\dist{{\rm dist}}
\def\sdo{\raisebox{.06cm}{$\bigcirc$\hspace{-0.38cm}\raisebox{0.0cm}{$\bot$}}\,}
\def\spen{{\rm span}}
\def\rectanglepath{-- ++(1cm,0cm) -- ++(0cm,1cm) -- ++(-1cm,0cm) -- cycle}
\newcommand{\del}{\partial}
\newcounter{exn}
\setcounter{exn}{1}
\newcommand{\exn}{\theexn\stepcounter{exn}}
\newcommand{\rin}{\rotatebox[origin=c]{-90}{\Large $\in$}}
\newcommand{\rsubset}{\rotatebox[origin=c]{-90}{\Large $\subset$}}
\newcommand{\Lsubset}{\rotatebox[origin=c]{0}{\Large $\subset$}}
\renewcommand{\hom}{\mathrm{Hom}}

\newenvironment{sol}{\begin{tcolorbox}[breakable,colback=blue!5!white,colframe=blue!40!white,title=\normalsize {\sc{Solução}},coltitle=black]}{\end{tcolorbox}}

\newenvironment{augmatrix}{\left(\begin{array}}{\end{array}\right)}






%------ Desenha ângulos retos no espaço ---------------------------------------------------------------------
%--- Parâmetros (A,B,C,t,s)
%--- sendo A, B, C pontos no epaço, e t e s números reais entre 0 e 1.
%--- t é a fração do segmento AB, e s é a fração do segmento BC utilizadas para contruir o quadradinho.
%------------------------------------------------------------------------------------------------------------
\newcommand\drawanguloreto[5]{
  \draw[-] ($#2 - #4*#2 + #4*#1$)  -- ($#2 - #4*#2 + #4*#1 - #5*#2 + #5*#3$) -- ($#2 - #5*#2 + #5*#3$);
}

\tikzset{
  % style to apply some styles to each segment of a path
  on each segment/.style={
    decorate,
    decoration={
      show path construction,
      moveto code={},
      lineto code={
        \path [#1]
        (\tikzinputsegmentfirst) -- (\tikzinputsegmentlast);
      },
      curveto code={
        \path [#1] (\tikzinputsegmentfirst)
        .. controls
        (\tikzinputsegmentsupporta) and (\tikzinputsegmentsupportb)
        ..
        (\tikzinputsegmentlast);
      },
      closepath code={
        \path [#1]
        (\tikzinputsegmentfirst) -- (\tikzinputsegmentlast);
      },
    },
  },
  % style to add an arrow in the middle of a path
  mid arrow/.style={postaction={decorate,decoration={
        markings,
        mark=at position .5 with {\arrow[#1]{stealth}}
      }}},
}
%------------------------------------------------------------------------------------------------------------




%--------  AJUSTANDO O TAMANHO DAS PÁGINAS -----------------------------------------------------
\addtolength{\textwidth}{4 cm}
\addtolength{\textheight}{3 cm}
\addtolength{\oddsidemargin}{-2 cm}
\addtolength{\evensidemargin}{-2 cm}
\addtolength{\topmargin}{-3 cm}



%-------- NUMERAÇÃO DE DEFINIÇÕES, TEOREMAS, ETC...  ---------------------------------------------
\newtheorem{df}{Definição}[subsection]
\newtheorem{thm}[df]{Teorema}
\newtheorem{cor}[df]{Corolário}
\newtheorem{prop}[df]{Proposição}
\newtheorem{lemma}[df]{Lema}
\newtheorem{conjec}[df]{Conjectura}
\newtheorem{exerc}[df]{Exercício(s)}
\newtheorem{qst}{Questão}
\newtheorem{ex}[df]{Exemplo(s)}
\newtheorem{enunc}{Enunciado}[exn]

%--------  TÍTULO E DATA   ----------------------------------------------------------
\author{Solução do ``1º Testinho'' - GAAL}
\date{28 de Março de 2019}
\title{}





%-------------------------------------------------------------------------------------------
%-------------------------------------------------------------------------------------------
%-----    INÍCIO DO TEXTO   ----------------------------------------------------------------
%-------------------------------------------------------------------------------------------
%-------------------------------------------------------------------------------------------
\begin{document}
\maketitle

Em todas as questões abaixo, sempre que encontrar uma solução você deve mostrar que ela é, de fato, uma solução.

\begin{qst}
Dado o sistema linear
\[AX=0\] em que $A\in M_n(\R)$ e $X$ é uma matriz coluna de variáveis, você seria capaz de apresentar uma solução óbvia do sistema sem fazer contas?
\end{qst}
\begin{sol}
	Como uma ``solução'' é uma matriz coluna $C$ tal que $AC=B$, o sistema linear $AX=0$ sempre admite solução $X=0$, já que toda matriz vezes a matriz coluna $0$ é $0$.
\end{sol}

\begin{qst}
Considere o sistema linear
\[\begin{pmatrix}
1&3&-1&5\\5&15&-10&40\\0&0&3&-9
\end{pmatrix}\begin{pmatrix}x\\y\\z\\w
\end{pmatrix}=\begin{pmatrix}
-7\\-45\\6
\end{pmatrix}\]e faça o que se pede:
\begin{enumerate}[a)]
	\item Exiba a matriz aumentada desse sistema.
	\item Exiba a matriz escalonada desse sistema.
	\item Calcule o conjunto solução desse sistema.
	\item Exiba uma solução do sistema, e verifique que ela é, de fato, uma solução.
\end{enumerate}
\end{qst}
\begin{sol}
	\begin{enumerate}[a)]		
		\item A matriz aumentada de um sistema consiste da matriz de coeficientes acrescida da matriz de resultados. Assim, a matriz aumentada do sistema é
		\[\begin{augmatrix}{cccc:c}
			1&3&-1&5&-7\\
			5&15&-10&40&-45\\
			0&0&3&-9&6
		\end{augmatrix}.\]
		\item Vamos escalonar essa matriz:
		\[\begin{array}{cc}
			\begin{augmatrix}{cccc:c}
			1&3&-1&5&-7\\
			5&15&-10&40&-45\\
			0&0&3&-9&6
			\end{augmatrix}&\rightsquigarrow\begin{augmatrix}{cccc:c}
			1&3&-1&5&-7\\
			0&0&-5&15&-10\\
			0&0&3&-9&6
			\end{augmatrix}\\&\rightsquigarrow\begin{augmatrix}{cccc:c}
			1&3&-1&5&-7\\
			0&0&1&-3&2\\
			0&0&3&-9&6
			\end{augmatrix}\\&\rightsquigarrow\begin{augmatrix}{cccc:c}
			1&3&0&2&-5\\
			0&0&1&-3&2\\
			0&0&0&0&0
			\end{augmatrix}
		\end{array}\] e ver que a forma escalonada da matriz desse sistema é
		\[\begin{augmatrix}{cccc:c}
		1&3&0&2&-5\\
		0&0&1&-3&2\\
		0&0&0&0&0
		\end{augmatrix}.\]
		\item Agora, para achar uma solução, usamos a forma escalonada acima para obter que $x+3y+w=-5$ e $z-3w=2$. Como temos duas equações e quatro variáveis, teremos duas variáveis livres.
		
		Apesar da apresentação do conjunto sistema depender de quais variáveis livres você vai escolher, o conjunto em si independe dessa escolha. Então vamos escolher $y$ e $w$ livres, ou seja, $x=-3y-w-5$ e $z=3w+2$.
	
		Assim, o conjunto solução é da forma
		\[S=\{(x,y,z,w)\in \R^4\mid x=-3y-w-5,z=3w+2\},\]que também podemos expressar por
		\[S=\left\{\begin{pmatrix}
			-3y-w-5\\y\\3w+2\\w
			\end{pmatrix}\in M_{4\times 1}(\R)\right\}\]ou ainda
			\[S=\left\{\lambda\begin{pmatrix}
			-3\\1\\0\\0
			\end{pmatrix}+\mu\begin{pmatrix}
			-1\\0\\3\\1
			\end{pmatrix}+\begin{pmatrix}
			-5\\0\\2\\0
			\end{pmatrix}\in M_{4\times 1}(\R)\mid\lambda,\mu\in \R\right\}.\]
			
		\item Finalmente, escolhendo valores para as variáveis livres podemos encontrar soluções do sistema. Por exemplo, escolhendo $\lambda=\mu=0$, vemos que $\begin{pmatrix}
		-5\\0\\2\\0
		\end{pmatrix}$ é uma solução. Vamos testar:
		\[\begin{pmatrix}
		1&3&-1&5\\5&15&-10&40\\0&0&3&-9
		\end{pmatrix}\begin{pmatrix}
		-5\\0\\2\\0
		\end{pmatrix}=\begin{pmatrix}
		-5+0-2+0\\-25+0-20+0\\0+0+6
		\end{pmatrix}=\begin{pmatrix}
		-7\\-45\\6
		\end{pmatrix},\]ou seja, $\begin{pmatrix}
		5\\0\\2\\0
		\end{pmatrix}$ é, de fato, uma solução.
	\end{enumerate}
\end{sol}

\begin{qst}
	Resolva o sistema linear
	\[\begin{pmatrix}
	2&1\\5&3
	\end{pmatrix}\begin{pmatrix}
	x\\y
	\end{pmatrix}=\begin{pmatrix}
	2\\-3
	\end{pmatrix}\]sem escalonar e sem fazer substituição (dica: use matrizes inversas).
\end{qst}
\begin{sol}
	Sabemos que a matriz inversa de uma matriz da forma $\begin{pmatrix}
	a&b\\c&d
	\end{pmatrix}$ é a matriz $\dfrac{1}{ad-bc}\begin{pmatrix}
	d&-b\\-c&a
	\end{pmatrix}$. Assim, a inversa da matriz acima é $\dfrac{1}{6-5}\begin{pmatrix}
	3&-1\\-5&2
	\end{pmatrix}=\begin{pmatrix}
	3&-1\\-5&2
	\end{pmatrix}$. Com isso, podemos calcular as soluções do sistema:
	\[\begin{array}{rcl}
	\begin{pmatrix}
	2&1\\5&3
	\end{pmatrix}\begin{pmatrix}
	x\\y
	\end{pmatrix}&=&\begin{pmatrix}
	2\\-3
	\end{pmatrix}\\
	\begin{pmatrix}
	3&-1\\-5&2
	\end{pmatrix}\left(\begin{pmatrix}
	2&1\\5&3
	\end{pmatrix}\begin{pmatrix}
	x\\y
	\end{pmatrix}\right)&=&\begin{pmatrix}
	3&-1\\-5&2
	\end{pmatrix}\begin{pmatrix}
	2\\-3
	\end{pmatrix}\\ \left(
	\begin{pmatrix}
	3&-1\\-5&2
	\end{pmatrix}\begin{pmatrix}
	2&1\\5&3
	\end{pmatrix}\right)\begin{pmatrix}
	x\\y
	\end{pmatrix}&=&\begin{pmatrix}
	6+3\\-10-6
	\end{pmatrix}\\\begin{pmatrix}
	1&0\\0&1
	\end{pmatrix}\begin{pmatrix}
	x\\y
	\end{pmatrix}&=&\begin{pmatrix}
	9\\-16
	\end{pmatrix}\\\begin{pmatrix}
	x\\y
	\end{pmatrix}&=&\begin{pmatrix}
	9\\-16
	\end{pmatrix}
	\end{array},\] ou seja, a única solução do sistema é $x=9$ e $y=-16$.
	
	Contudo, caso não soubéssemos qual a inversa da matriz do sistema, poderíamos calculá-la resolvendo o seguinte sistema:
	\[\begin{pmatrix}
	2&1\\5&3
	\end{pmatrix}X=\begin{pmatrix}
	1&0\\0&1
	\end{pmatrix}.\]
	
	\[\begin{array}{rl}
	\begin{augmatrix}{cc:cc}
	2&5&1&0\\1&3&0&1
	\end{augmatrix}&\rightsquigarrow\begin{augmatrix}{cc:cc}
	1&\dfrac{5}{2}&\dfrac{1}{2}&0\\&\\
	1&3&0&1
	\end{augmatrix}\rightsquigarrow\begin{augmatrix}{cc:cc}
	1&\dfrac{5}{2}&\dfrac{1}{2}&0\\&\\
	0&\dfrac{1}{2}&-\dfrac{1}{2}&1
	\end{augmatrix}\\&\rightsquigarrow\begin{augmatrix}{cc:cc}
	1&\dfrac{5}{2}&\dfrac{1}{2}&0\\&\\
	0&1&-1&2
	\end{augmatrix}\rightsquigarrow\begin{augmatrix}{cc:cc}
	1&0&3&-5\\&\\
	0&1&-1&2
	\end{augmatrix}
	\end{array}\]ou seja, $X=\begin{pmatrix}
	3&-5\\-1&2
	\end{pmatrix}$ é inversa da matriz do sistema, e podemos simplesmente repetir o raciocínio anterior para resolver o sistema.
\end{sol}
\begin{qst}
	Sem fazer contas, explique porque o sistema abaixo não possui solução:
	\[\begin{pmatrix}
		1&0\\0&1\\9&8\\\pi&\sqrt{2}\\0&0		
	\end{pmatrix}\begin{pmatrix}
	x\\y
\end{pmatrix}=\begin{pmatrix}
1\\1\\17\\\pi+\sqrt{2}\\1278
\end{pmatrix}.\]

Você consegue modificar apenas uma linha do sistema de forma que ele passe a ter solução? Se sim, exiba tal modificação e calcule a solução. Se não, diga por quê é impossível.
\end{qst}
\begin{sol}
	Claramente o sistema não possui solução, pois a última linha nos diz que se $(x,y)$ é solução, então $0x+0y=1278$, mas claramente $0x+0y=0$ e $0\neq 1278$. Então não existe solução.
	
	Contudo, poderíamos notar que, a menos da última linha, o sistema tem uma solução única: $(1,1)$. Então, podemos remover a última linha, e isso tornaria o sistema solucionável. Podemos também trocar a última linha por $(1,1277)$ ou de maneira geral, $(a,1278-a)$. Isso faria com que $a+(1278-a)=1278$, então a solução $(1,1)$ continua sendo solução.
	
	Poderíamos também simplesmente zerar toda a última linha, e certamente $0+0=0$.
\end{sol}

\end{document}