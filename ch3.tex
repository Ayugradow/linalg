\chapter{Spaaaaaace!}
\section{Introduction}

Now that we're reasonably comfortable with $\R^2$ (don't worry, we'll come back to it) we're gonna take the next step: Consider the set $\R^3$ and study what are its vectors, how they behave etc.

But, as it's going to become apparent very soon, this is essentially \textbf{not different at all} from what we've already been doing for $\R^2$ - some proofs are actually literally the same, without changing anything.

As a consequence, this chapter shall be much shorter than the previous one. Here's the basic schema of this chapter:

\begin{itemize}
	\item First, we're gonna define $\R^3$ and stablish its properties. At this point we'll notice how it's basically $\R^2$ all over again, with some few minor changes;
	\item Then we're gonna prove whatever is new for $\R^3$ and state which results from $\R^2$ still hold.
\end{itemize}

With that said, let us begin!

\newpage
\subsection{A small step for mankind... Kinda}

By definition, $\R^3$ is the set of all ordered triples of real numbers - like $(1,2,3)$ or $(0,\pi, -4)$ etc. We won't go into as much detail as we did for $\R^2$, but we can prove the following result:

\begin{prop}
	The Euclidean space $E_*$ with distinguished point $*$ is in bijection with $\R^3$.
\end{prop}
\begin{cor}
	Every point in $\R^3$ can be thought of as either a point in the space, or a vector from the origin to its endpoint.
\end{cor}

The idea is pretty simple - we can think of any ordered triple $(x,y,z)\in \R^3$ as a set of ``coordinates'' in a grid system which tells us how to move away from the distinguished point $*$: Move $x$ steps away from $*$ in a certain direction, move $y$ steps away from $*$ in a different direction and then $z$ steps away from $*$ in a third direction.

And we can, as before, define:

\begin{df}
	Given two vectors $v=(v_1,v_2,v_3),u=(u_1,u_2,u_3)\in \R^3$ we define their \textbf{sum} $v+u$ to be the unique vector given by
	\[v+u:=(v_1+u_1,v_2+u_2,v_3+u_3).\]
\end{df}

The intuition here is also very similar as it was for $\R^2$: $0,v,u$ define a unique triangle in the space and, by axiom, there's a unique plane containing this triangle. Therefore, $v+u$ is the parallelogram which is contained in that plane and has $0v$ and $0u$ as its sides.

\begin{prop}
	The addition of vectors in $\R^3$ is associative, commutative, has identity element and inverses.
\end{prop}
\begin{proof}
	Let $v,u,w\in \R^3$ be given by $v=(v_1,v_2,v_3)$, $u=(u_1,u_2,u_3)$ and $w=(w_1,w_2,w_3)$. Then:
	
	\begin{itemize}
		\item \begin{align*}
			v+(u+w)&=(v_1,v_2,v_3)+((u_1,u_2,u_3)+(w_1,w_2,w_3))\\
			&=(v_1,v_2,v_3)+(u_1+w_1,u_2+w_2,u_3+w_3)\\
			&=(v_1+u_1+w_1,v_2+u_2+w_2,v_3+u_3+w_3)\\
			&=(v_1+u_1,v_2+u_2+v_3+u_3)+(w_1,w_2,w_3)\\
			&=((v_1,v_2,v_3)+(u_2,u_2,u_3))+(w_1,w_2,w_3)=(v+u)+w
		\end{align*} so it is associative;
		
		\item \begin{align*}
			v+u&=(v_1,v_2,v_3)+(u_1,u_2,u_3)\\
			&=(v_1+u_1,v_2+u_2,v_3+u_3)\\
			&=(u_1+v_1,u_2+v_2,u_3+v_3)\\
			&=(u_1,u_2,u_3)+(v_1,v_2,v_3)=u+v
		\end{align*}so it is commutative;
		
		\item  \begin{align*}
			v+0&=(v_1,v_2,v_3)+(0,0,0)\\
			&=(v_1+0,v_2+0,v_3+0)\\
			&=(v_1,v_2,v_3)=v\\
			&=(0+v_1,0+v_2,0+v_3)\\
			&=(0,0,0)+(v_1,v_2,v_3)=0+v
		\end{align*}so $0$ is the identity element;
		
		\item \begin{align*}
			v+(-v)&=(v_1,v_2,v_3)+(-v_1,-v_2,-v_3)\\
			&=(v_1-v_1,v_2-v_2,v_3-v_3)=0
		\end{align*}so it has inverses.
	\end{itemize}

This ends the proof.
\end{proof}

\begin{df}
	Let us define a few important subsets of $\R^3$:
	\begin{itemize}
		\item The set $\{0\}\times \R\times \R$ will be called the $\mathds{YZ}$-plane;
		\item The set $\R\times\{0\}\times \R$ will be called the $\mathds{XZ}$-plane;
		\item The set $\R\times \R\times \{0\}$ will be called the $\mathds{XY}$-plane;
		\item The set $\R\times\{0\}\times \{0\}$ will be called the $\mathds{X}$-axis;
		\item The set $\{0\}\times \R\times\{0\}$ will be called the $\mathds{Y}$-axis;
		\item The set $\{0\}\times\{0\}\times \R$ will be called the $\mathds{Z}$-axis.
	\end{itemize}
\end{df}

\begin{df}
	Let $v=(v_1,v_2,v_3)\in \R^3$ and $\lambda\in \R$. We define the \textbf{scalar multiplication} of $v$ and $\lambda$ to be the vector $\lambda v\in\R^3$ given by
	\[\lambda v:=(\lambda v_1,\lambda v_2,\lambda v_3).\]
\end{df}

\begin{df}
	Given any vectors $v,u\in\R^3$ we define:
	\begin{itemize}
		\item $\R v:=\{w\in \R^3\mid w=\lambda v\mbox{ for some }\lambda\in \R\}$ the \textbf{line through zero containing $v$};
		\item $\R v+\R u:=\{w\in \R^3\mid w=\lambda v+\mu u\mbox{ for some }\lambda,\mu\in \R\}$ the \textbf{plane through zero containing $v$ and $u$}. 
	\end{itemize}
\end{df}