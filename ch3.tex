\chapter{Spaaaaaace!}
\section{Introduction}

Now that we're reasonably comfortable with $\R^2$ (don't worry, we'll come back to it) we're gonna take the next step: Consider the set $\R^3$ and study what are its vectors, how they behave etc.

But, as it's going to become apparent very soon, this is essentially \textbf{not different at all} from what we've already been doing for $\R^2$ - some proofs are actually literally the same, without changing anything.

As a consequence, this chapter shall be much shorter than the previous one. Here's the basic schema of this chapter:

\begin{itemize}
	\item First, we're gonna define $\R^3$ and stablish its properties. At this point we'll notice how it's basically $\R^2$ all over again, with some few minor changes;
	\item Then we're gonna prove whatever is new for $\R^3$ and state which results from $\R^2$ still hold.
\end{itemize}

With that said, let us begin!

\newpage
\subsection{A small step for mankind... Kinda}

By definition, $\R^3$ is the set of all ordered triples of real numbers - like $(1,2,3)$ or $(0,\pi, -4)$ etc. We won't go into as much detail as we did for $\R^2$, but we can prove the following result:

\begin{prop}
	The Euclidean space $E_*$ with distinguished point $*$ is in bijection with $\R^3$.
\end{prop}
\begin{cor}
	Every point in $\R^3$ can be thought of as either a point in the space, or a vector from the origin to its endpoint.
\end{cor}