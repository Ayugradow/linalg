\chapter{Real Linear Algebra}
\section{Introduction}
\subsection{First notions and definitions}

To start working with vector spaces we first need to understand that the perspective is going to change a bit from the previous chapter. We're leaving the domain of \textbf{set theory} and jumping right in the domain of \textbf{algebra}.

\textit{Algebra }is the domain of mathematics that deals with operations and their properties.

\begin{df}
	Given any set $X$, a \textbf{(binary) operation} on $X$ is a function $f:X\times X \to X$.
\end{df}

\begin{ex}
	Let $\N$ be the set of natural numbers, as before. We have a few operations here:
	\[f,g,h:\N\times\N \to \N\]
	\[(n,m)\mapsto f(n,m):=n+m\]
	\[(n,m)\mapsto g(n,m):=nm\]
	\[(n,m)\mapsto h(n,m):=n^m\]and some of these operations have some properties that the others don't.
	
	For instance, all three functions satisfy the following property:
	
	\begin{itemize}
		\item Let $\phi$ be an operation on $X$. There is some $n_e\in X$ such that $\phi(n,n_e)=n$ for all $n\in X$.
	\end{itemize}

	In the case of $f$, if we choose $n_e:=0$, we see that $f(n,0)=n+0=n$, no matter which $n\in \N$ we chose, so $f$ satisfies the property above.
	
	In the case of $g$, if we choose $n_e:=1$, we see that $g(n,1)=n\cdot1=n$, no matter which $n\in\N$ we chose, so $g$ satisfies the property above.
	
	Finally, in the case of $h$, if we choose $n_e:=1$, we see that $h(n,1)=n^1=n$, no matter which $n\in\N$ we chose, so $h$ satisfies the property above.
	
	\bigskip
	Next up is the property:
	
	\begin{itemize}
		\item Let $\phi$ be an operation on $X$. There is some $n_e\in X$ such that $\phi(n_e,n)=n$ for all $n\in X$.
	\end{itemize}

	What can we say about $f,g,h$ in this case? Well, it's easy to see that for $f$ and $g$ it still holds true - and it does so for the same value of $n_e$ as before.
	
	However, for $h$ it fails. For instance, is there some number $x\in\N$ such that $h(x,2)=2$? Well, by definition of $h$ we would need to have $x^2=2$ and so $x=\sqrt{2}$ which is not in $\N$ - this tells us that there's no such $x\in\N$. It follows that this property fails for $h$.
	
	\bigskip
	Summing up all of these together, we get the following property:
	\begin{itemize}
		\item (Identity element) Let $\phi$ be an operation on $X$. There is some $n_e\in X$ such that $\phi(n,n_e)=n=\phi(n_e,n)$ for all $n\in X$.
	\end{itemize}
	And we see that $f$ and $g$ have what's called an \textit{identity element} - it's an element $n_e$ such that if you fix it in any input of your operation, then your operation is just the identity function.
	
	\bigskip
	Consider now the following property:
	\begin{itemize}
		\item (Associativity) Let $\phi$ be an operation on $X$. Then, for all $n,m,l\in X$ we have that $\phi(\phi(n,m),l)=\phi(n,\phi(m,l))$.
	\end{itemize}

	In the case of $f$ we can check
	\[f(f(n,m),l)=f(n+m,l)=(n+m)+l=n+(m+l)=f(n,m+l)=f(n,f(m,l))\]and see that $f$ is associative.
	
	In the case of $g$ we can check
	\[g(g(n,m),l)=g(nm,l)=(nm)l=n(ml)=g(n,ml)=g(n,g(m,l))\]and see that $g$ is associative.
	
	However, for $h$, once again, this property fails: For instance, let us compare $h(h(2,2),3)$ and $h(2,h(2,3))$:
	\[h(h(2,2),3)=h(2^2,3)=(2^2)^3=4^3=64\]
	\[h(2,h(2,3))=h(2,2^3)=2^{(2^3)}=2^8=256\]so they are clearly different, and $h$ is not associative.
	
	\bigskip
	One more:
	\begin{itemize}
		\item (Commutativity) Let $\phi$ be an operation on $X$. Then, for all $n,m\in X$ we have that $\phi(n,m)=\phi(m,n)$.
	\end{itemize}

	In the case of $f$ we can easily see that $f(n,m)=n+m=m+n=f(m,n)$.
	
	Similarly for $g$, we see that $g(n,m)=nm=mn=g(m,n)$.
	
	But, once again, $h(2,3)=8\neq 9=h(3,2)$, so $h$ is not commutative.
	
	\bigskip
	These are the most common operations in $\N$ and some of their properties. Now, let us show something that is \textbf{not} an operation:
	
	Consider the functions $$f',g':\N\times\N\to\N$$\[(n,m)\mapsto f'(n,m):=n-m\]
	\[(n,m)\mapsto g'(n,m):=n/m.\]
	
	Notice that I've just lied to you - these are \textbf{not} functions. To see that, take $f'$ and apply it on $(3,1)$. By definition of function, $f'(3,1)$ should lie on $\N$, the codomain of $f'$. But, by definition of $f'$, we see that $f'(3,1)=3-1=-2$, which is \textbf{not} in $\N$.
	
	Similarly, $g'$ isn't a function for the same reason: It should take, for instance, $(1,2)$ to a natural number - but it doesn't. It takes $(1,2)$ to $g'(1,2)=1/2$ which, once more, is not a natural number.
	
	However, for \textit{some} specific values of the input, $f'$ and $g'$ really have outputs in $\N$. For that reason, they are called \textbf{partial operations} and, sadly, won't be studied in this text, since we're mostly concerned with proper operations.
	
	If, however, you'd like to learn more about partial operations, you should click \href{https://ncatlab.org/nlab/show/groupoid}{here} or Google for ``groupoid'' - which is precisely the mathematical notion of a set with an associative partial operation.
\end{ex}

\begin{df}
	Given an operation $\phi:X\times X\to X$ we will say that
	\begin{itemize}
		\item (Identity element) $\phi$ admits an \textbf{identity element} if there is some $e\in X$ such that $\phi(x,e)=x=\phi(e,x)$ for all $x\in X$. In this case, $e$ is called an \textbf{identity element};
		\item (Associativity) $\phi$ is \textbf{associative} if $\phi(x,\phi(y,z))=\phi(\phi(x,y),z)$ for all $x,y,z\in X$;
		\item (Commutative) $\phi$ is \textbf{commutative} if $\phi(x,y)=\phi(y,x)$ for all $x,y\in X$;
		\item (Inverse element) $\phi$ admits \textbf{inverse elements} if for all $x\in X$ there is some $y\in X$ such that $\phi(x,y)=e=\phi(y,x)$ for some identity element $e\in X$.
	\end{itemize}
\end{df}

\begin{df}
	Let $X$ be a set with two operations, $f,g:X\times X\to X$. We say that \textbf{$f$ distributes over $g$ on the left} (resp. \textbf{on the right}) if $$f(x,g(y,z))=g(f(x,y),f(x,z))$$ (resp. $$f(g(x,y),z)=g(f(x,z),f(y,z)))$$ for all $x,y,z\in X$.
	
	If $f$ distributes over $g$ on both sides, we simply say that \textbf{$f$ distributes over $g$}.
\end{df}
\begin{ex}
	Following up on the previous example, we see that $g$ (the multiplication) distributes over $f$ (the addition):
	\[g(n,f(m,l))=g(n,m+l)=n(m+l)=nm+nl=f(nm,nl)=f(g(n,m),g(n,l))\]
	\[g(f(n,m),l)=g(n+m,l)=(n+m)l=nl+ml=f(nl,ml)=f(g(n,l),g(m,l))\]but $f$ doesn't distribute (on either side!) over $g$:
	\[1+(1\cdot 1)=1+1=2\neq 4=2\cdot 2=(1+1)\cdot(1+1)\]
	\[(1\cdot 1)+1=1+1=2\neq 4=2\cdot 2=(1+1)\cdot(1+1)\]
\end{ex}

All this talk now brings us to a very specific definition:

\begin{df}
	Let $X$ be a set with two operatios $A,M:X\times X\to X$. We will say that $(X,A,M)$ is a \textbf{field} if
	\begin{multicols}{3}
		\begin{enumerate}[(1)]
			\item $A$ is associative;
			\item $A$ is commutative;
			\item $A$ has an identity element;
			\item $A$ has inverses;
			\item $M$ is associative;
			\item $M$ is commutative;
			\item $M$ has an identity element;
			\item $M$ has inverses (excluding the additive identities);
			\item $M$ distributes over $A$.
		\end{enumerate}
	\end{multicols}
	

In this case, we call $A$ and $M$, respectively, the field's \textbf{addition} and \textbf{multiplication} operations, and denote them simply by $x+y:=A(x,y)$ and $xy:=M(x,y)$ for all $(x,y)\in X\times X$.
\end{df}

\begin{prop}
	The set $\R$ of real numbers with the usual addition and multiplication is a field.
\end{prop}
\begin{proof}
	This is immediate, since for every $x,y,z\in\R$ we have:
	\begin{multicols}{3}
		\begin{enumerate}[(1)]
			\item $x+(y+z)=(x+y)+z$;
			\item $x+y=y+x$;
			\item $x+0=0+x=x$;
			\item $x+(-x)=(-x)+x=0$;
			\item $x(yz)=(xy)z$;
			\item $xy=yx$;
			\item $x\cdot 1=1\cdot x=x$;
			\item $xx^{-1}=x^{-1}x=1$ if $x\neq 0$;
			\item $x(y+z)=xy+xz$ and $(x+y)z=xz+yz$.
		\end{enumerate}
	\end{multicols}	
\end{proof}

\begin{ex}
	Notice, however, that the sets $\N$ and $\Z$, of the naturals and integers, respectively, are \textbf{not} fields: $\N$ doesn't have either additive or multiplicative inverses (so it fails properties (4) and (8)), and $\N$ doesn't have multiplicative inverses (so it fails property (8)).
	
	On the other hand, it's easy to see that $\Q$, the set of rational numbers, is indeed a field. It is actually constructed to be, in some sense, ``the smallest field which extends $\Z$/$\N$''.
	
	Finally, the set $\C$ of complex numbers is also a field if you define the inverse of $z=x+iy$ to be $z^{-1}:=\dfrac{x-iy}{x^2+y^2}$. Indeed:
	
	\[zz^{-1}=(x+iy)\left(\frac{x-iy}{x^2+y^2}\right)=\frac{x^2+y^2}{x^2+y^2}=1\]so it is indeed an inverse for $z$.
\end{ex}

Let us show some properties of fields:

\begin{lemma}
	Let $(k,+,\cdot)$ be a field. Then the following hold:
	\begin{enumerate}[(a)]
		\item There's a unique additive identity;
		\item For each $x\in k$, there's a unique additive inverse;
		\item There's a unique multiplicative identity;
		\item For each $x\in k$, there's a unique multiplicative inverse;
		\item Let $0$ be an additive identity of $k$. Then $0x=0$ for all $x\in k$.
		\item Let $1$ be a multiplicative identity of $k$. Then $-x=(-1)x$, where $(-1)+1=0$.
	\end{enumerate}
\end{lemma}
\begin{proof}
	\begin{enumerate}[(a)]
		\item Let $0$ and $0'$ be two additive identities of $k$. Then
		\[0=0+0'=0'\]where the leftmost equality holds since $0'$ is additive identity, and the rightmost equality holds since $0$ is additive identity, and so $0=0'$.
		
		\item Given $x\in k$, let $x'$ and $x''$ be two additive inverses to $x$. Then
		\[x'=x'+0=x'+(x+x'')=(x'+x)+x''=0+x''=x''\]and so $x'=x''$.
		
		\item Let $1$ and $1'$ be two multiplicative identities of $k$. Then
		\[1=1\cdot1'=1'\]where the leftmost equality holds since $1'$ is a multiplicative identity, and the rightmost equality holds since $1$ is a multiplicative identity, so $1=1'$.
		
		\item Given $x\in k$, let $x''$ and $x''$ be two multiplicative inverses to $x$. Then
		\[x'=x'\cdot1=x'(xx'')=(x'x)x''=1\cdot x''=x''\]and so $x'=x''$.
		
		\item Given $x\in k$, we have that
		\[0x=(0+0)x=0x+0x\]since $k$ is a field and $0$ is the additive identity. Let $y\in k$ be the additive inverse of $0x$.
		
		Then, since the above is true, we can see that $(0x)+y=(0x+0x)+y$ is also true. But the LHS is just 0, since $y$ is the additive inverse of $0x$, and the RHS is just $(0x+0x)+y=0x+(0x+y)=0x+0=0x$, so the above equation evaluates to $0=0x$.
		
		\item Given $x\in k$, we have that $0x=(1+(-1))x$ since $1+(-1)=0$. But now, by the distributive property of fields we see that $0x=(1+(-1))x=(1)x+(-1)x$.
		
		But $0x=0$ and $1x=x$, so this is just $0=x+(-1)x$. Since additive inverses are unique, we see that $(-1)x=-x$.
	\end{enumerate}
\end{proof}

This result basically tells us that every field is ``similar'' to $\R$, in some sense.

\begin{rmk}
	The reason why we require that the multiplication has inverses for all elements \textit{except for 0} is precisely because of item (e) above. Since $0x=0$ for all $x$, if we could have some $0^{-1}$, then $0=00^{-1}=1$ so we would have $0=1$.
	
	But since $1x=x$ for all $x$, this would imply that $x=1x=0x=0$, so \textbf{every element of the field would have to be 0 for it to be consistent}.
	
	In other words, the only set that satisfies all the properties of a field and also has a multiplicative inverse to 0 is the set $\{0\}$.
	
	In fact:
\end{rmk}

\begin{prop}
	The set $1=\{0\}$ with addition and multiplication being equal and given by $0+0=0\cdot 0=0$ is a field. Its multiplicative and additive identities are $0$, who is also the inverse of $0$.
\end{prop}
\begin{proof}
	There's literally nothing to prove.
\end{proof}

Finally, to end this section, let us give some examples of fields that aren't $1$, $\Q$, $\R$ or $\C$.

\begin{ex}
	Let $p\in \N$ be a prime number (that is, there are only two ways to write $p=nm$: $n=p, m=1$ and $n=1, m=p$). Consider the set $p\in\N$ - that is, $p=\{0,1,2,\cdots,p-1\}$. We will give a field structure to $p$ as follows:
	
	For any $x,y\in p$, define:
	\begin{itemize}
		\item $x+y$ is the remainder of the division of $x+y$ in $\N$ by $p$;
		\item $xy$ is the remainder of the division of $xy$ in $\N$ by $p$.
	\end{itemize}

	We claim that $p$ with those two operations is a field, which will be denoted by either $\Z_p$, $\Z/p\Z$ or $\mathds F_p$.
	
	For instance, let us do some computations with $p=3$.
	
	In this case, $p=\{0,1,2\}$, and so we have the following tables of operations:
	
	\begin{center}
		\begin{tabu}{|c|[2pt]c|c|c|}
			\hline+&0&1&2\\
			\tabucline[2pt]{-} 0&0&1&2\\
			\hline 1&1&2&0\\
			\hline 2&2&0&1\\
			\hline 
		\end{tabu} and \begin{tabu}{|c|[2pt]c|c|c|}
			\hline $\times$&0&1&2\\
			\tabucline[2pt]{-} 0&0&0&0\\
			\hline 1&0&1&2\\
			\hline 2&0&2&1\\
			\hline 
		\end{tabu}
	\end{center}
	
	It is, then, readily seen that $0$ and $1$ are, respectively, the additive and multiplicative identities of $\mathds{F}_3$.
	
	We can also see that $1+2=0$ so 1 and 2 are additive inverses to each other. Similarly, we see that $1\cdot 1=1=2\cdot 2$ so both 1 and 2 are multiplicative inverses to themselves.
	
	This shows that $\mathds{F}_3$ is a field.
	
	Building similar tables of operations we can prove that any $\mathds{F}_p$ is a field.
	
	Let us now show the necessity of $p$ being a prime.
	
	\bigskip
	Let $4=\{0,1,2,3\}$. Let's try building the same operations:
	
	\begin{center}
		\begin{tabu}{|c|[2pt]c|c|c|c|}
			\hline
			+&0&1&2&3\\\tabucline[2pt]{-}
			0&0&1&2&3\\\hline
			1&1&2&3&0\\\hline
			2&2&3&0&1\\\hline
			3&3&0&1&2\\\hline 
		\end{tabu} and \begin{tabu}{|c|[2pt]c|c|c|c|}
			\hline
			$\times$&0&1&2&3\\\tabucline[2pt]{-}
			0&0&0&0&0\\\hline
			1&0&1&2&3\\\hline
			2&0&2&0&2\\\hline
			3&0&3&2&1\\\hline
		\end{tabu}
	\end{center}but this shows that $2$ doesn't have any multiplicative inverses: $2\cdot 0=0$, $2\cdot 1=2$, $2\cdot 2=0$ and $2\cdot 3=2$.
	
	But by definition of a field, the only element that has no multiplicative inverse is 0. But clearly $2\neq 0$ (since $1+2\neq 1$), so $\Z/4\Z$ cannot be a field.
	
	This happens precisely because $4$ can be written as $4=nm$ in \textit{three} different ways: $4=4\cdot1=1\cdot 4=2\cdot 2$.
	
	Since this isn't supposed to be a course on field theory, we won't go into much detail on how to prove that $\Z/n\Z$ is a field if, and only if, $n$ is prime.
\end{ex}

\newpage
\subsection{Real vector fields}

Let us start this section with the set that will be the focus of most, if not all, of this chapter: $\R^2$.

By definition, $\R^2=\R\times \R$ is the set of ordered pairs of real numbers.

\begin{df}
	We're going to define the \textbf{addition} $A:\R^2\times \R^2\to \R^2$ to be given by $A((x,y),(z,w)):=(x+z,y+w)$ for any $(x,y),(z,w)\in\R^2$.
\end{df}

\begin{prop}
	The addition $A$ we've just defined satisfies the following properties:
	\begin{enumerate}[(i.)]
		\item $A$ is associative;
		\item $A$ is commutative;
		\item $A$ admits an identity element;
		\item $A$ admits inverses.
	\end{enumerate}
\end{prop}
\begin{proof}
	Choose any three elements $(a,b),(c,d),(e,f)\in\R^2$. Then:
	\begin{enumerate}[(i.)]
		\item \begin{align*}
			A\left(A\left((a,b),(c,d)\right),(e,f)\right)&=A((a+c,b+d),(e,f))\\
			&=((a+c)+e,(b+d)+f)\\
			&=(a+(c+e),b+(d+f))\\
			&=A((a,b),(c+e,d+f))=A((a,b),A((c,d),(e,f))),
		\end{align*}so $A$ is associative;
		
		\item \begin{align*}
		A((a,b),(c,d))&=(a+c,b+d)=(c+a,d+b)=A((c,d),(a,b))
		\end{align*}so $A$ is commutative;
		
		\item $A((a,b),(0,0))=(a+0,b+0)=(a,b)=(0+a,0+b)=A((0,0),(a,b))$ so $A$ has identity $(0,0)$;
		
		\item $A((a,b),(-a,-b))=(a-a,b-b)=(0,0)=(-a+a,-b+b)=A((-a,-b),(a,b))$ so $A$ has inverses.
	\end{enumerate}
\end{proof}

\begin{rmk}
	From now on, we're gonna denote $A((a,b),(c,d))$ by $(a,b)+(c,d)$ for any $(a,b),(c,d)\in \R^2$ since, by the preceding proposition, it behaves well-enough like number addition.
\end{rmk}

And it seems there's not much else we can do with $\R^2$ for now.

To proceed with our studies, then, we're gonna need a new approach.

\begin{df}
	We'll denote $E_*$ the \textbf{pointed Euclidean plane}. That is, $E$ is the set of all points in the Euclidean plane, and $*$ is a distinguished point.
\end{df}

\begin{ex}
	For instance,
	
	\[\definecolor{ududff}{rgb}{0.30196078431372547,0.30196078431372547,1.}
	\begin{tikzpicture}[line cap=round,line join=round,>=triangle 45,x=1.0cm,y=1.0cm]
	\clip(-0.5,-2.) rectangle (7.,6.);
	\begin{scriptsize}
	\draw [fill=ududff] (0.94,0.96) circle (2.5pt);
	\draw[color=ududff] (1.08,1.33) node {$A$};
	\draw [fill=ududff] (5.06,2.92) circle (2.5pt);
	\draw[color=ududff] (5.2,3.29) node {$B$};
	\end{scriptsize}
	\end{tikzpicture}\] we can think of the set $E_A$ (whose elements are all points in the plane, including $A$ and $B$, but distinguishing $A$) and the set $E_B$ (whose elements are all points in the plane, including $A$ and $B$, but distinguishing $B$).
\end{ex}

\begin{df}
	Given a pointed Euclidean plane $E_*$, we define a \textit{vector in $E_*$} to be any directed segment starting in $*$.
\end{df}

\begin{ex}
	Continuing the above example,
	\[
	\definecolor{ududff}{rgb}{0.30196078431372547,0.30196078431372547,1.}
	\begin{tikzpicture}[line cap=round,line join=round,>=triangle 45,x=1.0cm,y=1.0cm]
	\clip(-3.18,-2.52) rectangle (6.,4.75);
	\draw [->] (0.94,0.96) -- (2.2,3.56);
	\draw [->] (0.94,0.96) -- (-2.32,1.66);
	\draw [->] (0.94,0.96) -- (-0.06,-1.7);
	\draw [->] (0.94,0.96) -- (5.5,0.1);
	\draw [->] (0.94,0.96) -- (4.34,2.82);
	\draw [->] (5.06,2.92) -- (4.34,2.82);
	\draw [->] (5.06,2.92) -- (5.5,0.1);
	\begin{scriptsize}
	\draw [fill=ududff] (0.94,0.96) circle (2.5pt);
	\draw[color=ududff] (0.72,1.43) node {$A$};
	\draw [fill=ududff] (4.34,2.82) circle (2.5pt);
	\draw [fill=ududff] (5.5,0.1) circle (2.5pt);
	\draw [fill=ududff] (-0.06,-1.7) circle (2.5pt);
	\draw [fill=ududff] (-2.32,1.66) circle (2.5pt);
	\draw [fill=ududff] (2.2,3.56) circle (2.5pt);	
	\draw [fill=ududff] (5.06,2.92) circle (2.5pt);
	\draw[color=black] (1.46,2.47) node {$s$};
	\draw[color=black] (-0.66,1.17) node {$v$};
	\draw[color=black] (0.62,-0.23) node {$w$};
	\draw[color=black] (3.28,0.79) node {$u$};
	\draw[color=black] (2.64,2.15) node {$t$};
	\draw[color=ududff] (5.2,3.29) node {$B$};
	\draw[color=black] (4.5,2.8) node {$t$};
	\end{scriptsize}
	\end{tikzpicture}
	\]we see that $s,t,u,v,w$ are vectors in $E_A$, but not in $E_B$.
\end{ex}