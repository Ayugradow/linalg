\documentclass[numbers]{report}
\setcounter{tocdepth}{4}
\usepackage[numbers]{natbib}
\usepackage{amsfonts,amsmath,amstext,amssymb,color}
\usepackage{amsthm,thmtools,thm-restate}
\usepackage{arydshln}
\usepackage{tcolorbox}
\usepackage{nameref}
\usepackage{hyperref}
\usepackage{cleveref}
\usepackage{enumerate}
\usepackage{chngcntr}
% \usepackage{graphicx,graphpap}
\usepackage{multicol} % This is so we can have multiple columns of text side-by-side
\columnsep=20pt % This is the amount of white space between the columns in the poster
\raggedbottom
%\usepackage{titlesec}
%\titleformat{\chapter}[display]   
%{\normalfont\huge\bfseries}{\chaptertitlename\ \thechapter}{20pt}{\Huge}   
%\titlespacing*{\chapter}{0pt}{-50pt}{40pt}
\usepackage{dsfont}
\usepackage{pgf,tikz,pgfplots}
\pgfplotsset{compat=1.15}
\usepackage{mathrsfs}
\usetikzlibrary{arrows}





\hypersetup{
	colorlinks=true,       % false: boxed links; true: colored links
	linkcolor=[rgb]{0,0,0.5},          % color of internal links (change box color with linkbordercolor)
	citecolor=[rgb]{0.1,0.05,0.67},        % color of links to bibliography
	filecolor=magenta,      % color of file links
	urlcolor=cyan           % color of external links
}


\tcbuselibrary{breakable}
\usetikzlibrary{calc,intersections,through,backgrounds,positioning,decorations.pathmorphing,decorations.pathreplacing,decorations.markings}

\tikzstyle{printersafe}=[decoration={snake,amplitude=0pt}]






%------- DEFINIÇÕES DE COMANDOS UTILIZADOS -----------------------------------
\def\R{\mathbb R}
\def\C{\mathbb C}
\def\N{\mathbb N}
\def\Z{\mathbb{Z}}
\def\eqmod{\!\!\!\mod}
\def\herm#1{\langle #1\rangle}
\def\hmod#1{\parallel #1\parallel}
\def\L{\Lambda}
\def\G{\Gamma}
\newcommand{\dps}{\displaystyle}
\newcommand{\bn}{\bigskip\noindent}
\newcommand{\mb}{\mathbb}
\newcommand{\mc}{\mathcal}
\newcommand{\mf}{\mathfrak}
\newcommand{\mtt}{\mathtt}
\def\ang{{\rm ang}}
\def\id{{\rm id}}
\def\sen{{\rm sen\ }}
\def\diag{{\rm diag}}
\def\dist{{\rm dist}}
\def\End{{\rm End}}
\def\sdo{\raisebox{.06cm}{$\bigcirc$\hspace{-0.38cm}\raisebox{0.0cm}{$\bot$}}\,}
\def\spen{{\rm span}}
\def\rectanglepath{-- ++(1cm,0cm) -- ++(0cm,1cm) -- ++(-1cm,0cm) -- cycle}
\newcommand{\del}{\partial}
\newcommand{\epi}{\twoheadrightarrow}
\newcommand{\mono}{\hookrightarrow}
\newcommand{\iso}{\cong}
\newcommand{\isomor}{\stackrel{\iso}{\to}}
\newcounter{exn}
\setcounter{exn}{1}
\newcommand{\exn}{\theexn\stepcounter{exn}}
\newcommand{\rin}{\rotatebox[origin=c]{-90}{\Large $\in$}}
\newcommand{\rsubset}{\rotatebox[origin=c]{-90}{\Large $\subset$}}
\newcommand{\Lsubset}{\rotatebox[origin=c]{0}{\Large $\subset$}}

\newcommand{\ladj}{\dashv}
\newcommand{\radj}{\vdash}
\renewcommand{\mod}{\mathrm{Mod}}
\newcommand{\bimod}[2]{#1\text{-}#2\text{-}\mathrm{Bimod}}
\newcommand{\comod}{\mathrm{Comod}}
\newcommand{\contra}{\mathrm{Contra}}
\newcommand{\bicomod}[2]{(#1,#2)\text{-}\mathrm{Bicomod}}
\newcommand{\cat}[1]{\mathrm{#1}}
\newcommand{\comA}[1]{\mathrm{Hom}_{\text{-}#1}}
\newcommand{\com}[1]{\mathrm{Hom}_{#1\text{-}}}
\renewcommand{\hom}{\mathrm{Hom}}
\newcommand{\mor}{\mathrm{Mor}}
\newcommand{\pcmod}[1]{\mathrm{PCmod}\text{-}#1}
\newcommand{\pcbimod}[2]{(#1,#2)\text{-}\mathrm{PCbimod}}
\newcommand{\codim}{\mathrm{codim}}
\newcommand{\arrow}[1]{\stackrel{#1}{\longrightarrow}}
\newcommand{\qf}[1]{#1\text{-}\mathrm{comod}}
\newcommand{\Cohom}[1]{\mathrm{Cohom}_#1}
\newcommand{\sqf}[1]{\mathrm{comod_s}\text{-}{#1}}
\newcommand{\inj}[1]{\mathrm{inj}\text{-}{#1}}
\newcommand{\proj}[1]{\mathrm{proj}\text{-}{#1}}
\newcommand{\ninj}[1]{\mathrm{comod}_{\mathcal{I}}\text{-}{#1}}
\newcommand{\vect}[1]{\mathrm{Vect}\text{-}{#1}}
\newcommand{\restrict}{\!\!\restriction}
\newcommand{\cprod}{\mathbin{\rotatebox[origin=c]{45}{\(\square\)}}}
\newcommand{\coch}{\mathrm{Coch}}
\newcommand{\ch}{\mathrm{Ch}}
\newcommand{\ho}{\mathrm{Ho}}

\newcommand{\abs}[1]{\lvert#1\rvert}

\newcommand{\inprod}[1]{\langle #1\rangle}
\newcommand{\ses}[5]{\[0\arrow{}#1\arrow{#4}#2\arrow{#5}#3\arrow{}0\]}

\DeclareMathOperator{\add}{add}
\DeclareMathOperator{\dual}{D}
\DeclareMathOperator{\cohom}{Cohom}
\DeclareMathOperator{\coend}{Coend}
\DeclareMathOperator{\coext}{Coext}
\DeclareMathOperator{\im}{Im}
\DeclareMathOperator{\tr}{Tr}
\DeclareMathOperator{\Ker}{Ker}
\DeclareMathOperator{\rad}{Rad}
\DeclareMathOperator{\corad}{Corad}
\DeclareMathOperator{\coker}{coker}
\DeclareMathOperator{\Coker}{Coker}
\DeclareMathOperator{\gen}{Gen}
\DeclareMathOperator*{\cotensor}{\mathbin{\square}}
\DeclareMathOperator{\ctens}{\widehat{\otimes}}
\DeclareMathOperator{\card}{card}
\DeclareMathOperator*{\tensor}{\mathbin{\otimes}}
\DeclareMathOperator*{\bigcprod}{\Large \cprod}

\newenvironment{sol}{\noindent\normalsize {\sc Solução:}}

\newenvironment{ex}{\begin{tcolorbox}[breakable,colback=blue!5!white,colframe=white!80!black,title=\normalsize {\sc EXEMPLO(S):},coltitle=black]}{\end{tcolorbox}}

\newcounter{exerc}
\numberwithin{exerc}{section}
\renewcommand*{\theexerc}{(\arabic{chapter}.\arabic{section}.\arabic{exerc})}
\newenvironment{exerc}{\stepcounter{exerc}\begin{tcolorbox}[breakable,colback=blue!5!white,colframe=blue!40!white,title=\normalsize {\sc\textbf{Exercício} \theexerc},coltitle=black]}{\end{tcolorbox}}

\newenvironment{cdiag}{\begin{displaymath}\begin{tikzcd}}{\end{tikzcd}\end{displaymath}}
\newenvironment{multicdiag}{\begin{equation*}\begin{tikzcd}
}{\end{tikzcd}\end{equation*}}

\newenvironment{augmatrix}{\left(\begin{array}}{\end{array}\right)}



%------ Desenha ângulos retos no espaço ---------------------------------------------------------------------
%--- Parâmetros (A,B,C,t,s)
%--- sendo A, B, C pontos no epaço, e t e s números reais entre 0 e 1.
%--- t é a fração do segmento AB, e s é a fração do segmento BC utilizadas para contruir o quadradinho.
%------------------------------------------------------------------------------------------------------------
\newcommand\drawanguloreto[5]{
	\draw[-] ($#2 - #4*#2 + #4*#1$)  -- ($#2 - #4*#2 + #4*#1 - #5*#2 + #5*#3$) -- ($#2 - #5*#2 + #5*#3$);
}

\tikzset{
	% style to apply some styles to each segment of a path
	on each segment/.style={
		decorate,
		decoration={
			show path construction,
			moveto code={},
			lineto code={
				\path [#1]
				(\tikzinputsegmentfirst) -- (\tikzinputsegmentlast);
			},
			curveto code={
				\path [#1] (\tikzinputsegmentfirst)
				.. controls
				(\tikzinputsegmentsupporta) and (\tikzinputsegmentsupportb)
				..
				(\tikzinputsegmentlast);
			},
			closepath code={
				\path [#1]
				(\tikzinputsegmentfirst) -- (\tikzinputsegmentlast);
			},
		},
	},
	% style to add an arrow in the middle of a path
	mid arrow/.style={postaction={decorate,decoration={
				markings,
				mark=at position .5 with {\arrow[#1]{stealth}}
	}}},
}
%------------------------------------------------------------------------------------------------------------




%--------  AJUSTANDO O TAMANHO DAS PÁGINAS -----------------------------------------------------
\addtolength{\textwidth}{4 cm}
\addtolength{\textheight}{3 cm}
\addtolength{\oddsidemargin}{-2 cm}
\addtolength{\evensidemargin}{-2 cm}
\addtolength{\topmargin}{-3 cm}
\setlength{\fboxsep}{4pt}




%-------- NUMERAÇÃO DE DEFINIÇÕES, TEOREMAS, ETC...  ---------------------------------------------

\declaretheorem[name=Definição,refname={definição,definições},Refname={Definição,Definições}]{df}
\numberwithin{df}{section}
\renewcommand*{\thedf}{\arabic{chapter}.\arabic{section}.\arabic{df}}
\declaretheorem[numberlike=df,name=Teorema,refname={teorema,teoremas},
Refname={Teorema,Teoremas}]{theorem}
\declaretheorem[sibling=df,name=Corolário,refname={corolário,corolários},
Refname={Corolário,Corolários}]{cor}
\declaretheorem[sibling=df,name=Lema,refname={lema,lemas},
Refname={Lema,Lemas}]{lemma}
\declaretheorem[sibling=df,name=Proposição,refname={proposição,proposições},
Refname={Proposição,Proposições}]{prop}
\declaretheorem[sibling=df,name=Propriedade,refname={propriedade,propriedades},
Refname={Propriedade,Propriedades}]{pp}
%\declaretheorem[name=Exercício,refname={exercício,exercícios},Refname={Exercício,Exercícios}]{exerc}
\declaretheorem[sibling=df,name=Observação,refname={observação,observações},Refname={Observação,Observações}]{rmk}


% \makeatletter
% \let\c@equation\c@df
% \let\theequation\thedf
% \makeatother
\renewcommand*{\theequation}{E(\arabic{section}.\arabic{equation})}







%-------------------------------------------------------------------------------------------
%-------------------------------------------------------------------------------------------
%-----    INÍCIO DO TEXTO   ----------------------------------------------------------------
%-------------------------------------------------------------------------------------------
%-------------------------------------------------------------------------------------------
\usepackage[brazil,portuguese]{babel}
\usepackage[utf8]{inputenc}
%--------  TÍTULO E DATA   ----------------------------------------------------------
\title{Notas de Aula de GAAL}

\author{Ricardo Souza}
\begin{document}
	\maketitle

\tableofcontents
\chapter*{Introdução}

Este texto foi escrito como material auxiliar para um curso de GAAL ministrado em 2019/1.
\chapter{Matrizes}
\section{Definições e Propriedades Básicas}

Matrizes são simplesmente o nome matemático dado a tabelas de valores. Por exemplos, podemos ter matrizes numéricas
\[\begin{pmatrix}
1 & 2 & 9\\
-8 & 5 & \pi
\end{pmatrix}, \begin{pmatrix}
0 & 0\\
0 & 0
\end{pmatrix},
\begin{pmatrix}
1 & 0 & 0\\
0 & 5 & 0\\
0 & 0 & -16
\end{pmatrix},\cdots\] ou de qualquer outra natureza, realmente
\[\begin{pmatrix}
\text{Geometria} & \text{Analítica}\\
\text{Álgebra} & \text{Linear}
\end{pmatrix},\begin{pmatrix}
\sqcup & \triangle & \triangledown\\
\circledcirc & \nabla & \sum
\end{pmatrix},\cdots.\]

A princípio pode parecer meio arbitrário estudar esses objetos - e mais ainda definir operações e fazer matemática com eles. A motivação para isso ficará em um capítulo posterior. Neste momento, vamos dar uma motivação bem ``rasa'':

\begin{ex}
	Suponha que cinco amigos, Ana, Bernardo, Carlos, Diogo e Eliza resolveram sair para comemorar o aniversário de Carlos no boliche. Eles acabaram jogando três partidas e obtiveram os seguintes resultados:
	\[\bordermatrix{
	&\text{Ana} & \text{Bernando} & \text{Carlos} & \text{Diogo} &\text{Eliza}\cr
	\text{Primeiro jogo} & 101 & 96 & 99 & 87 & 123\cr
	\text{Segundo jogo} & 95 & 100 & 110 & 80 & 102\cr
	\text{Terceiro jogo} & 90 & 103 & 80 & 86 & 110}.\]Inconformado com o resultado, Diogo praticou bastante. Alguns meses depois, em seu aniversário, convidou os amigos para repetirem a jogatina, e obtiveram os seguintes resultados:
	\[\bordermatrix{
		&\text{Ana} & \text{Bernando} & \text{Carlos} & \text{Diogo} &\text{Eliza}\cr
		\text{Primeiro jogo} & 97 & 87 & 90 & 150 & 103\cr
		\text{Segundo jogo} & 80 & 105 & 100 & 170 & 98\cr
		\text{Terceiro jogo} & 120 & 110 & 80 & 127 & 115}.\] Surpresos com o resultados, eles resolveram computar quem tinha o maior total de pontos, combinando os dois aniversários. Com isso, eles obtiveram
	\[\bordermatrix{
		&\text{Ana} & \text{Bernando} & \text{Carlos} & \text{Diogo} &\text{Eliza}\cr
		\text{Primeiro jogo} & 101+97 & 96+87 & 99+90 & 87+150 & 123+103\cr
		\text{Segundo jogo} & 95+80 & 100+105 & 110+100 & 80+170 & 102+98\cr
		\text{Terceiro jogo} & 90+120 & 103+110 & 80+80 & 86+127 & 110+115}\]\[=\bordermatrix{
		&\text{Ana} & \text{Bernando} & \text{Carlos} & \text{Diogo} &\text{Eliza}\cr
		 & 198 & 183 & 189 & 237 & 226\cr
		 & 175 & 205 & 210 & 250 & 200\cr
		 & 210 & 213 & 160 & 213 & 225}\]
\end{ex}

\subsection{Somas e Produtos por Números}

\begin{df}
	Denotaremos por $M_{n\times m}(\R)$ o \textbf{conjunto das matrizes com $n$ linhas e $m$ colunas, e com entradas em $\R$}. Caso $n=m$ diremos que nossas matrizes são \textbf{quadradas} e notaremos simplesmente $M_n(\R)$.
	
	De maneira análoga, denotaremos o elemento na linha $i$, coluna $j$ de uma matriz $M$ por $M_{i,j}$.
\end{df}

\begin{ex}
	A matriz \[M=\begin{pmatrix}
	1 & 2 & 9\\
	-8 & 5 & \pi
	\end{pmatrix}\]claramente pertence a $M_{2\times 3}(\R)$. Além disso, temos: $M_{1,1}=1,M_{1,2}=2,M_{1,3}=9,M_{2,1}=-8,M_{2,2}=5$ e $M_{2,3}=\pi$.
	
	\bigskip
	Reciprocamente, se dissermos que uma matriz $N\in M_{3\times 2}(\R)$ tem como entradas $N_{1,1}=0=N_{3,2},N_{1,2}=1,N_{2,1}=-1,N_{2,2}=\pi$ e $N_{3,1}=-\pi$, podemos recuperar a matriz $N$:
	\[N=\begin{pmatrix}
	0 & 1\\
	-1 & \pi\\
	-\pi &0
	\end{pmatrix}\]
\end{ex}

Desse exemplo tiramos uma informação muito importante: \textbf{uma matriz está unicamente determinada por seus elementos} - isto é, dada uma matriz $M$, então existe uma única coleção de elementos $M_{i,j}$; e dada qualquer coleção $a_{i,j}$ existe uma única matriz $A$ cujos elementos são exatamente $A_{i,j}=a_{i,j}$.

Isso pode parecer trivial, mas nos permite, por exemplo, criar a seguinte definição:

\begin{df}
	Dadas duas matrizes $M,N\in M_{n\times m}(\R)$, definimos a \textbf{soma de $M$ e $N$} como sendo uma matriz $M+N$ dada por
	\[(M+N)_{i,j}:= M_{i,j}+N_{i,j}.\]
\end{df}
\begin{rmk}
	Aqui, o símbolo ``$x:=y$'' significa ``estamos definindo $x$ como sendo igual a $y$'' ou ``$x=y$ por definição''.
\end{rmk}
\begin{rmk}
	Note que a definição acima faz todo sentido: para cada par de índices $i,j$, $M_{i,j}$ e $N_{i,j}$ são números reais (que nós já sabemos somar!) e portanto $M_{i,j}+N_{i,j}$ também é um número real. Nós estamos, então, coletando todas as somas, variando $i$ e $j$, e chamando essa matriz, cujos elementos são somas dos elementos de $M$ e $N$, de $M+N$.
\end{rmk}
\begin{exerc}
	Do jeito que definimos, só sabemos somar matrizes que têm o mesmo número de linhas e colunas. Tente criar uma definição de soma de matrizes que funcione para quaisquer duas matrizes. Por que não usamos uma definição desse tipo?
\end{exerc}

Similarmente a como fizemos com a soma, definindo elemento a elemento, podemos também definir multiplicação por números:

\begin{df}
	Seja $a\in \R$ um número real qualquer e $M\in M_{n\times m}(\R)$ uma matriz $n\times m$ real. Definimos o \textbf{produto de $a$ cópias de $M$} como sendo a matriz $aM$ dada por
	\[(aM)_{i,j}:=a\cdot M_{i,j}.\]
\end{df}

Novamente, pelo mesmo argumento acima, isso faz sentido, porque como $a$ e cada $M_{i,j}$ são números reais (que nós sabemos multiplicar!), então $aM_{i,j}$ também é um número real.

\begin{ex}
	Sejam $M=\begin{pmatrix}
	1 & -8\\
	2 & 5\\
	9 & \pi
	\end{pmatrix}$ e $N=\begin{pmatrix}
	0 & 1\\
	-1 & \pi\\
	-\pi &0
	\end{pmatrix}$ matrizes reais, e $a=\pi$. Então, podemos computar:
	\begin{alignat*}{3}
	&(M+N)_{1,1}=M_{1,1}+N_{1,1}=1+0=1\quad&\quad&(M+N)_{1,2}=M_{1,2}+N_{1,2}=-8+1=-7\\
	&(M+N)_{2,1}=M_{2,1}+N_{2,1}=2+(-1)=1\quad&\quad&(M+N)_{2,2}=M_{2,2}+N_{2,2}=5+\pi\\
	&(M+N)_{3,1}=M_{3,1}+N_{3,1}=9+(-\pi)=9-\pi\quad&\quad&(M+N)_{3,2}=M_{3,2}+N_{3,2}=\pi+0=\pi
	\end{alignat*}e escrever
	\[M+N=\begin{pmatrix}
	1 & -7\\
	1 & 5+\pi\\
	9-\pi & \pi
	\end{pmatrix}.\]
	\tcblower
	Similarmente, podemos computar
	\begin{alignat*}{3}
	&(\pi M)_{1,1}=\pi\cdot 1=\pi\quad&\quad &(\pi M)_{1,2}=\pi\cdot (-8)=-8\pi\\
	&(\pi M)_{2,1}=\pi\cdot 2 = 2\pi\quad&\quad &(\pi M)_{2,2}=\pi\cdot 5 = 5\pi\\
	&(\pi M)_{3,1}=\pi\cdot 9 = 9\pi\quad&\quad 	&(\pi M)_{3,2}=\pi\cdot \pi=\pi^2
	\end{alignat*}e escrever
	\[\pi M=\begin{pmatrix}
	\pi & -8\pi\\
	2\pi & 5\pi\\
	9\pi &\pi^2
	\end{pmatrix}.\]
\end{ex}

\begin{exerc}
	Calcule, usando os dados do exemplo acima, $aN$ e $a(M+N)$. Em seguida, calcule $(M+M)+N$ e $2M+N$. O que podemos dizer dessas duas últimas matrizes?
\end{exerc}

\subsection{Produtos(?)}

Agora que sabemos somar matrizes e multiplicar matrizes por números, o mais natural seria definirmos uma multiplicação de matrizes. Poderíamos, intuitivamente, nos inspirar nas construções acima e definir que dadas duas matrizes $M,N\in M_{n\times m}(\R)$, o produto delas será uma matriz $M\times N$ dada por
\[(M\times N)_{i,j}:=M_{i,j}\cdot N_{i,j}\]o que faria sentido, já que para cada par de índices $i,j$, ambos $M_{i,j}$ e $N_{i,j}$ são números reais - o que já sabemos multiplicar. Além disso, essa operação de produto teria excelentes propriedades: Seria comutativa, associativa, teria inverso, teria elemento neutro...

\underline{Por que então não definimos a multiplicação de matrizes assim?}

A resposta simples é que o produto que vamos definir, apesar de não parecer intuitivo, é o que mais faz sentido quando consideramos as aplicações de matrizes que veremos mais à frente. 

\begin{df}[Multiplicação Clássica]\label{df:produto-classico}
	Dadas duas matrizes $M\in M_{n\times m}(\R)$ e $N\in M_{m\times l}(\R)$, definimos o \textbf{produto de $M$ com $N$} como sendo uma matriz $MN$ dada por
	\[(MN)_{i,j}:=M_{i,1}N_{1,j}+M_{i,2}N_{2,j}+\cdots+M_{i,m}N_{m,j}.\]
\end{df}

\begin{rmk}
	Novamente, convém notar que essa definição faz todo sentido, porque para cada trio de índices $i,j$ e $k$, $M_{i,k}$ e $N_{k,j}$ são números reais (que nós sabemos multiplicar!) e portanto $M_{i,k}N_{k,j}$ também é um número real; e como $(MN)_{i,j}$ é uma soma de números reais, $(MN)_{i,j}$ é, em si, um número real. Além disso,  essa definição explica a necessidade de o número de colunas de $M$ ser o número de linhas de $N$: esse número é exatamente o número de termos da soma.
\end{rmk}

Essa é a multiplicação de matrizes que já conhecemos desde sempre. Contudo, como comentamos previamente, ela ``não faz sentido'' - parece que surge do nada, e é desnecessariamente complicada.

Visando ``facilitar'' esse processo, vamos gastar um tempo tentando criar uma intuição de onde surge essa multiplicação de matrizes. Mas não se assuste - o resultado que vamos chegar é ``o mesmo''. A única diferença é que vamos explicar cada passo e tentar justificar essa definição. 

Sendo assim, continuem calculando produtos de matrizes como sempre fizeram, mas atentem para os raciocínios que virão a seguir para entender de onde esses produtos vêm.

\bigskip
\begin{df}
	Definimos por $e_{i,j}^n\in M_n(\R)$ a matriz dada por:
	\[(e_{i,j}^n)_{r,s}:=\begin{cases}
	1\text{ se } r=i \text{ e } s=j\\
	0\text{ caso contrário }
	\end{cases}.\]
\end{df}

\begin{ex}
	A matriz $e^2_{1,2}\in M_{2}(\R)$ é a matriz $2\times 2$ que tem $1$ na linha 1, coluna 2, e $0$ em todo o resto - ou seja,\[e^2_{1,2}=\begin{pmatrix}
	0 & 1\\
	0 & 0
	\end{pmatrix}.\] Já a matriz $e^3_{1,2}\in M_3(\R)$ é a matriz $3\times 3$ que tem $1$ na linha 1, coluna 2, e $0$ em todo o resto, ou seja, \[e^3_{1,2}=\begin{pmatrix}
	0 & 1&0\\
	0 & 0&0\\
	0&0&0
	\end{pmatrix}.\]
\end{ex}

Mais à frente, vamos estender essa definição para matrizes de tamanho arbitrário - não necessariamente quadradas. Por enquanto, vamos nos abster às matrizes quadradas para facilitar a vida.

\begin{df}
	Definimos por $E^n_{i,j}:M_{n\times m}(\R)\to M_{n\times m}(\R)$ a \textbf{função} dada por:
	\[\bordermatrix{&&&&\cr
		&M_{1,1} & M_{1,2} & \cdots & M_{1,m}\cr
		& M_{2,1}& M_{2,2} & \cdots & M_{2,m}\cr
		& \vdots & \vdots & \ddots & \vdots\cr		
		j\text{-ésima linha}& M_{j,1} & M_{j,2} & \cdots & M_{j,m}\cr
		& \vdots & \vdots & \ddots & \vdots\cr
		&M_{n,1} & M_{n,2} & \cdots & M_{n,m}	
	}\mapsto
	\bordermatrix{
	&&&&\cr
	&0&0&\cdots&0\cr
	&0&0&\cdots&0\cr
	& \vdots & \vdots & \ddots & \vdots\cr
	i\text{-ésima linha}& M_{j,1} & M_{j,2} & \cdots & M_{j,m}\cr
	& \vdots & \vdots & \ddots & \vdots\cr
	&0&0&\cdots&0
	}.
	\]
\end{df}

Ou seja, $E^n_{i,j}$ é a função que leva matrizes de $n$ linhas (e qualquer quantidade de colunas) em matrizes de $n$ linhas (e a mesma quantidade de colunas) simplesmente colando uma cópia da $j$-ésima linha da matriz original na $i$-ésima linha da imagem, e preenchendo o resto com $0$s.

\begin{ex}
	A função $E^{3}_{1,2}$ aplicada na matriz $M=\begin{pmatrix}
	1 & -8\\
	2 & 5\\
	9 & \pi
	\end{pmatrix}$ nos dá a matriz \(E^3_{1,2}(M)=\bordermatrix{
	&&\cr
	& 2 & 5\cr
	& 0 &0\cr
	&0 &0
	}\) em que nós simplesmente copiamos a segunda linha de $M$ e colocamos na primeira linha da matriz nova, preenchendo o resto com $0$s.

	Analogamente, se $N=\begin{pmatrix}
	0 & 1\\
	-1 & \pi\\
	-\pi & 0
	\end{pmatrix}$, então $E^3_{1,2}(N)=\begin{pmatrix}
	-1 & \pi\\
	0 & 0\\
	0 & 0
	\end{pmatrix}$ em que nós simplesmente copiamos a segunda linha de $N$ e colocamos na primeira linha da matriz nova, preenchendo o resto com $0$s.
\end{ex}

\begin{rmk}
	Aqui, o ``expoente'' da função simplesmente serve para indicar a quantidade de linhas das matrizes que você quer computar. Nos exemplos acima, o 3 no expoente de $E^3_{1,2}$ simplesmente indica que a função $E^3_{1,2}$ só pode ser aplicada em matrizes com três linhas.
\end{rmk}
\begin{exerc}
	Usando os dados do exemplo acima, calcule $E^3_{1,1}, E^3_{1,3}, E^3_{2,1}, E^3_{2,2}, E^3_{2,3},E^3_{3,1}, E^3_{3,2}$ e $E^3_{3,3}$ aplicadas tanto em $M$ quanto em $N$.
	
	O que podemos dizer sobre $E^3_{i,j}$ quando $i=j$? Ou seja, calcule $E^3_{1,1},E^3_{2,2}$ e $E^3_{3,3}$ de $M$ e $N$. O que essas matrizes têm de especial?
\end{exerc}
\begin{exerc}
	Usando o exemplo acima, existe algum par de índices $i,j$ tal que $E^3_{i,j}(E^3_{i,j}(M))=E^3_{i,j}(M)$? E para $N$?
\end{exerc}

Com isso temos nossa primeira noção de multiplicação de matrizes:

\begin{df}
	Dadas uma matriz $M\in M_{n\times m}(\R)$ qualquer e $e^n_{i,j}\in M_n(\R)$, definimos o \textbf{produto de $e^n_{i,j}$ e $M$} como sendo a matriz $e^n_{i,j}M$ dada por
	\[e^n_{i,j}M:=E^n_{i,j}(M).\]
\end{df}

\begin{ex}
	Ainda com os dados do exemplo anterior, podemos agora computar o produto
	\[\begin{pmatrix}
	0 & 1 & 0\\
	0&0&0\\
	0&0&0
	\end{pmatrix}\cdot\begin{pmatrix}
	1 & -8\\
	2 & 5\\
	9 &\pi
	\end{pmatrix}=e^3_{1,2}M:=E^3_{1,2}(M)=\begin{pmatrix}
	2 & 5\\0&0\\0&0
	\end{pmatrix}\]e
		\[\begin{pmatrix}
	0 & 1 & 0\\
	0&0&0\\
	0&0&0
	\end{pmatrix}\cdot\begin{pmatrix}
	0 & 1\\
	-1 & \pi\\
	-\pi &0
	\end{pmatrix}=e^3_{1,2}N:=E^3_{1,2}(N)=\begin{pmatrix}
	-1 &\pi\\0&0\\0&0
	\end{pmatrix}.\]
\end{ex}

\begin{rmk}
	Vamos mostrar, com alguns exemplos, que nossa definição até agora coincide com a definição clássica \ref{df:produto-classico}:
	\[\begin{pmatrix}
	0 & 1 & 0\\
	0&0&0\\
	0&0&0
	\end{pmatrix}\cdot\begin{pmatrix}
	1 & -8\\
	2 & 5\\
	9 &\pi
	\end{pmatrix}=\begin{pmatrix}
	0\cdot 1+1\cdot 2+0\cdot 9 & 0\cdot(-8)+1\cdot 5+0\cdot\pi\\
	0\cdot1+0\cdot2+0\cdot 9 & 0\cdot(-8)+0\cdot 5+0\cdot \pi\\
	0\cdot1+0\cdot2+0\cdot 9 & 0\cdot(-8)+0\cdot 5+0\cdot \pi
	\end{pmatrix}=\begin{pmatrix}
	2 & 5\\0&0\\0&0
	\end{pmatrix}\]que é exatamente o resultado que obtivemos aplicando $E^3_{1,2}$ em $M$.
\end{rmk}

\begin{exerc}
	Calcule $E^4_{3,2}$ de uma matriz $M=\begin{pmatrix}
	a & b\\
	c & d\\
	e & f\\
	g & h
	\end{pmatrix}\in M_{4\times 2}(\R)$ genérica. Em seguida, calcule $e^4_{3,2}M$ usando a multiplicação clássica de matrizes e compare os resultados.
\end{exerc}


\bigskip
O próximo passo é considerar matrizes que são \textit{quase} $e^n_{i,j}$. Por exemplo, como vamos multiplicar $\begin{pmatrix}
0 & 8 & 0\\
0 & 0 & 0\\
0&0&0
\end{pmatrix}$ por $M=\begin{pmatrix}
1 & -8\\
2 & 5\\
9 & \pi
\end{pmatrix}$? Bom, podemos perceber que $\begin{pmatrix}
0 & 8 & 0\\
0 & 0 & 0\\
0&0&0
\end{pmatrix}=8\cdot\begin{pmatrix}
0 & 1 & 0\\
0 & 0 & 0\\
0&0&0
\end{pmatrix}=8\cdot e^3_{1,2}$ e que nós já sabemos multiplicar números por matrizes e $e^3_{1,2}$ por $M$.

Dito de outra maneira, queremos calcular $(8\cdot e^3_{1,2})\cdot M$ sendo que já sabemos calcular $e^3_{1,2}\cdot M$. Como a multiplicação de números reais é associativa ($(ab)c=a(bc)$) faz sentido que desejemos que essa operação que vamos definir seja associativa. Assim, podemos simplesmente definir
\[(8\cdot e^3_{1,2})\cdot M:=8\cdot(e^3_{1,2}\cdot M)\]o que faz sentido, já que $8\in \R$, $e^3_{1,2}\cdot M\in M_{n\times m}(\R)$ e \textbf{já sabemos multiplicar matrizes por números}.

De maneira geral, podemos definir:

\begin{df}
	Dado um número real $a\in\ R$, uma matriz real $M\in M_{n\times m}(\R)$ e uma matriz quadrada $e^n_{i,j}\in M_n(\R)$, definimos o \textbf{produto de $ae^n_{ij}$ com $M$} como sendo a matriz $(ae^n_{i,j})M$ dada por
	\[(ae^n_{i,j})M:=a(e^n_{i,j}M).\]
\end{df}

\begin{exerc}
	Usando $M$ e $N$ do exemplo anterior, calcule $\begin{pmatrix}
	0 & 5 & 0\\
	0&0&0\\
	0&0&0
	\end{pmatrix}\cdot M$ e $\begin{pmatrix}
	0 & 0 & 0\\
	0&0&-\pi\\
	0&0&0
	\end{pmatrix}\cdot N$ usando nossa nova definição e compare com a multiplicação clássica de matrizes.
\end{exerc}

\bigskip
Outro tipo de matrizes que são \textit{quase} $e^n_{i,j}$ são matrizes que são \textit{somas} de matrizes $e^n_{i,j}$. Por exemplo, a matriz $A=\begin{pmatrix}
0 & 1 & 1\\
1 & 0 & 0\\
0 & 1 & 0
\end{pmatrix}$ pode ser escrita como
\[A=\begin{pmatrix}
0 & 1 & 1\\
1 & 0 & 0\\
0 & 1 & 0
\end{pmatrix}=\begin{pmatrix}
0 & 1 & 0\\
0&0&0\\
0&0&0
\end{pmatrix}+ \begin{pmatrix}
0 & 0 & 1 \\
0 & 0 & 0 \\
0 & 0 & 0 
\end{pmatrix}  + \begin{pmatrix}
0 & 0 & 0 \\
1 & 0 & 0 \\
0 & 0 & 0 
\end{pmatrix}  + \begin{pmatrix}
0 & 0 & 0 \\
0 & 0 & 0 \\
0 & 1 & 0 
\end{pmatrix}  =e^3_{1,2}+e^3_{1,3}+e^3_{2,1}+e^3_{3,2}.\] Agora, dada uma matriz $M=\begin{pmatrix}
1 & -8\\
2 & 5\\
9 & \pi
\end{pmatrix}$ queremos definir a multiplicação $A\cdot M$. Novamente, vamos nos lembrar que \textbf{já sabemos somar matrizes}, \textbf{já sabemos multiplicar matrizes $e^3_{i,j}$ por $M$} e \textbf{o produto de números reais distribui sobre a soma}. Assim, podemos definir, inspirados por essas três propriedades,
\[AM=(e^3_{1,2}+e^3_{1,3}+e^3_{2,1}+e^3_{3,2})M:=e^3_{1,2}M+e^3_{1,3}M+e^3_{2,1}M+e^3_{3,2}M,\]e calculando
\[e^3_{1,2}M=\begin{pmatrix}
2 & 5\\
0 & 0\\
0 & 0
\end{pmatrix},e^3_{1,3}M=\begin{pmatrix}
9 & \pi\\
0 & 0\\
0 & 0
\end{pmatrix},e^3_{2,1}M=\begin{pmatrix}
0 & 0\\
1 & -8\\
0 & 0
\end{pmatrix},e^3_{3,2}M=\begin{pmatrix}
0 & 0\\
9 & \pi\\
0 & 0
\end{pmatrix}\]vemos que
\[AM=\begin{pmatrix}
2 & 5\\
0 & 0\\
0 & 0
\end{pmatrix}+\begin{pmatrix}
9 & \pi\\
0 & 0\\
0 & 0
\end{pmatrix}+\begin{pmatrix}
0 & 0\\
1 & -8\\
0 & 0
\end{pmatrix}+\begin{pmatrix}
0 & 0\\
9 & \pi\\
0 & 0
\end{pmatrix}=\begin{pmatrix}
11 & 5+\pi\\
10 & \pi-8\\
0 & 0
\end{pmatrix}\]

\begin{exerc}
	Compare o resultado obtido acima com a definição clássica de multiplicação de matrizes.
\end{exerc}

\begin{df}
	Dados um conjunto de pares de índices $\{i_1,j_1\},\{i_2,j_2\},\cdots,\{i_k,j_k\}$ e uma matriz $M\in M_{n\times m}(\R)$, definimos o \textbf{produto da soma de todos os $e^n_{i_r,j_r}$ com $M$} como sendo a matriz $(e^n_{i_1,j_1}+e^n_{i_2,j_2}+\cdots+e^n_{i_k,j_k})M$ dada por
	\[(e^n_{i_1,j_1}+e^n_{i_2,j_2}+\cdots+e^n_{i_k,j_k})M:=e^n_{i_1,j_1}M+e^n_{i_2,j_2}M+\cdots+e^n_{i_k,j_k}M.\]
\end{df}

\bigskip
Agora podemos combinar os dois resultados: Considere a matriz $A=\begin{pmatrix}
0 & 5 & 2\\
-7 & 0 & 0\\
0 & 10 & 0
\end{pmatrix}$ e a matriz $M=\begin{pmatrix}
1 & -8\\
2 & 5\\
9 & \pi
\end{pmatrix}$. Como vamos definir o produto $AM$?

Primeiro vamos decompor $A$ como soma de matrizes \textit{quase} $e^n_{i,j}$
\[A=\begin{pmatrix}
0 & 5 & 2\\
-7 & 0 & 0\\
0 & 10 & 0
\end{pmatrix}=\begin{pmatrix}
0 & 5 & 0\\
0&0&0\\
0&0&0
\end{pmatrix}+ \begin{pmatrix}
0 & 0 & 2 \\
0 & 0 & 0 \\
0 & 0 & 0 
\end{pmatrix}  + \begin{pmatrix}
0 & 0 & 0 \\
-7 & 0 & 0 \\
0 & 0 & 0 
\end{pmatrix}  + \begin{pmatrix}
0 & 0 & 0 \\
0 & 0 & 0 \\
0 & 10 & 0 
\end{pmatrix}\]e em seguida escrever cada uma dessas matrizes como produto de um número por uma matriz $e^n_{i,j}$
\begin{alignat*}{7}
A=\begin{pmatrix}
0 & 5 & 2\\
-7 & 0 & 0\\
0 & 10 & 0
\end{pmatrix}&=&\begin{pmatrix}
0 & 5 & 0\\
0&0&0\\
0&0&0
\end{pmatrix}&+&\begin{pmatrix}
0 & 0 & 2 \\
0 & 0 & 0 \\
0 & 0 & 0 
\end{pmatrix}&+&\begin{pmatrix}
0 & 0 & 0 \\
-7 & 0 & 0 \\
0 & 0 & 0 
\end{pmatrix}&+&\begin{pmatrix}
0 & 0 & 0 \\
0 & 0 & 0 \\
0 & 10 & 0 
\end{pmatrix}\\&=&5e^3_{1,2} \quad\ &+&2e^3_{1,3}\quad&+&(-7)e^3_{2,1}\quad&+&10e^3_{3,2}\quad
\end{alignat*}e finalmente, como já sabemos multiplicar cada pedaço por $M$, vamos definir
\[AM=(5e^3_{1,2}+2e^3_{1,3}+(-7)e^3_{2,1}+10e^3_{3,2})M:=5e^3_{1,2}M+2e^3_{1,3}M+(-7)e^3_{2,1}M+10e^3_{3,2}M.\] Efetuando, vamos obter:
\begin{align*}
AM&=5\begin{pmatrix}
2 & 5\\
0 & 0\\
0 & 0
\end{pmatrix}+2\begin{pmatrix}
9 & \pi\\
0 & 0\\
0 & 0
\end{pmatrix}+(-7)\begin{pmatrix}
0 & 0\\
1 & -8\\
0 & 0
\end{pmatrix}+10\begin{pmatrix}
0 & 0\\
9 & \pi\\
0 & 0
\end{pmatrix}\\
&=\begin{pmatrix}
10 & 25\\
0 & 0\\
0 & 0
\end{pmatrix}+\begin{pmatrix}
18 & 2\pi\\
0 & 0\\
0 & 0
\end{pmatrix}+\begin{pmatrix}
0 & 0\\
-7 & 56\\
0 & 0
\end{pmatrix}+\begin{pmatrix}
0 & 0\\
90 & 10\pi\\
0 & 0
\end{pmatrix}=\begin{pmatrix}
28 & 25+2\pi\\
83 & 56+10\pi\\
0 & 0
\end{pmatrix}.
\end{align*}

\begin{rmk}
	Note que qualquer matriz quadrada pode ser escrita como acima - isto é, somando vários produtos de números com matrizes $e^n_{i,j}$. Isso nos permite finalmente definir produtos quase totalmente gerais:
\end{rmk}

\begin{df}
	Seja $A\in M_n(\R)$ uma matriz quadrada e $M\in M_{n\times m}(\R)$ qualquer. Definimos o \textbf{produto de $A$ com $M$} como sendo a matriz $AM$ dada por
	\begin{align*}
	AM&=(A_{1,1}e^n_{1,1}+A_{1,2}e^n_{1,2}+\cdots+A_{1,n}e^n_{1,n}+A_{2,1}e^n_{2,1}+\cdots A_{n,n}e^n_{n,n})M\\&:=A_{1,1}e^n_{1,1}M+A_{1,2}e^n_{1,2}M+\cdots+A_{1,n}e^n_{1,n}M+A_{2,1}e^n_{2,1}M+\cdots A_{n,n}e^n_{n,n}M
	\end{align*}
\end{df}

\begin{ex}
	Dada a matriz quadrada $A=\begin{pmatrix}
	a & b\\
	c & d
	\end{pmatrix}$ e uma matriz qualquer $M=\begin{pmatrix}
a' &b' & c'\\
d' & e' & f'
\end{pmatrix}$, temos que $AM$ é simplesmente
\[\begin{pmatrix}
a & 0\\
0 & 0
\end{pmatrix}\begin{pmatrix}
a' &b' & c'\\
d' & e' & f'
\end{pmatrix}+\begin{pmatrix}
0 & b\\
0 & 0
\end{pmatrix}\begin{pmatrix}
a' &b' & c'\\
d' & e' & f'
\end{pmatrix}+\begin{pmatrix}
0 & 0\\
c & 0
\end{pmatrix}\begin{pmatrix}
a' &b' & c'\\
d' & e' & f'
\end{pmatrix}+\begin{pmatrix}
0 & 0\\
0 & d
\end{pmatrix}\begin{pmatrix}
a' &b' & c'\\
d' & e' & f'
\end{pmatrix}\]
\[=\begin{pmatrix}
aa' &ab' & ac'\\
0 & 0 & 0
\end{pmatrix}+\begin{pmatrix}
bd' & be' & bf'\\
0&0&0
\end{pmatrix}+\begin{pmatrix}
0&0&0\\
ca' &cb' & cc'
\end{pmatrix}+\begin{pmatrix}
0&0&0\\
dd' & de' & df'
\end{pmatrix}\]
\[=\begin{pmatrix}
aa'+bd' &ba'+be' & ac'+bf'\\
ca'+dd' & cb'+de' & cc'+df'
\end{pmatrix}\]
\end{ex}

\begin{rmk}
	Em matemática, quando vamos escrever uma soma muito grande, ou com muitos termos, costumamos usar um símbolo especial para isso - $\sum$ - o \textbf{somatório}. Ele funciona da seguinte maneira: Ao invés de escrever $x_1+x_2+\cdots+x_n$ vamos escrever
	\[\sum_{i=1}^n x_i.\] A ideia é a seguinte: $i$ denota um índice variável, e os números $1$ e $n$ que aparecem abaixo e acima, respectivamente, do somatório indicam qual o valor mínimo e máximo que $i$ pode assumir, sempre variando de 1 em 1.
\end{rmk}

\begin{ex}
	Podemos escrever a soma de todos os números naturais entre 1 e $n$ usando \(\sum_{i=1}^ni,\) por exemplo,
	\[1+2+3+4=\sum_{i=1}^4i.\]
	
	Podemos escrever a soma dos quadrados dos números naturais entre 1 e $n$ como \(\sum_{i=1}^ni^2,\) por exemplo,
	\[1+4+9+16+25=1^2+2^2+3^2+4^2+5^2=\sum_{i=1}^5i^2.\]
	
	Podemos escrever a soma dos $n$ primeiros números ímpares e pares, respectivamente, como $\sum_{i=1}^n 2i-1\text{ e }\sum_{i=1}^n2i,$ por exemplo,
	\[1+3+5+7+9+11 = (2\cdot1-1)+(2\cdot2-1)+(2\cdot3-1)+(2\cdot3-1)+(2\cdot4-1)+(2\cdot5-1)+(2\cdot6-1) = \sum_{i=1}^62i-1\]\[2+4+6+8+10+12+14 = (2\cdot 1)+(2\cdot2)+(2\cdot3)+(2\cdot4)+(2\cdot5)+(2\cdot6)+(2\cdot7 )= \sum_{i=1}^72i\]
	
	Podemos escrever o produto clássico (\ref{df:produto-classico}) das matrizes $M\in M_{n,m}(\R)$ e $N\in M_{m,l}(\R)$ como $$(MN)_{i,j}=\sum_{k=1}^{m}M_{i,k}N_{k,j}.$$
\end{ex}

Com isso, por exemplo, podemos tornar a definição de produto mais compacta:
\begin{df}
	Seja $A\in M_n(\R)$ uma matriz quadrada e $M\in M_{n\times m}(\R)$ qualquer. Definimos o \textbf{produto de $A$ com $M$} como sendo a matriz $AM$ dada por
	\begin{align*}
	AM&=\left(\sum_{j=1}^n\sum_{i=1}^nA_{i,j}e^n_{i,j}\right)M\\&:=\sum_{j=1}^n\sum_{i=1}^n\left(A_{i,j}e^n_{i,j}M\right)
	\end{align*}
\end{df}

\bigskip

Finalmente, para encerrar esta seção, vamos aprender a multiplicar matrizes de tamanho qualquer, usando o raciocínio acima.

\begin{df}
	Vamos denotar por $e_{i,j}^{n,m}$ a matriz $n\times m$ dada por
	\[(e^{n,m}_{i,j})_{r,s}:=\begin{cases}
	1,\text{ se } r=i \text{ e } s=j\\
	0, \text{ caso contrário}.
	\end{cases}\]
\end{df}

Ou seja, é apenas uma generalização das nossas já familiares matrizes $e^n_{i,j}$ para matrizes não-quadradas com a mesma propriedade - elas têm 1 na entrada $i,j$ e 0 em todas as outras entradas.

Novamente, podemos definir agora funções que vão realizar nossas multiplicações:
\begin{df}
	Definimos por $E^{l,n}_{i,j}:M_{n\times m}(\R)\to M_{l\times m}(\R)$ a \textbf{função} dada por:
	\[\bordermatrix{&&&&\cr
		&M_{1,1} & M_{1,2} & \cdots & M_{1,m}\cr
		& M_{2,1}& M_{2,2} & \cdots & M_{2,m}\cr
		& \vdots & \vdots & \ddots & \vdots\cr		
		j\text{-ésima linha}& M_{j,1} & M_{j,2} & \cdots & M_{j,m}\cr
		& \vdots & \vdots & \ddots & \vdots\cr
		&M_{n,1} & M_{n,2} & \cdots & M_{n,m}	
	}\mapsto
	\bordermatrix{
		&&&&\cr
		&0&0&\cdots&0\cr
		&0&0&\cdots&0\cr
		& \vdots & \vdots & \ddots & \vdots\cr
		i\text{-ésima linha}& M_{j,1} & M_{j,2} & \cdots & M_{j,m}\cr
		& \vdots & \vdots & \ddots & \vdots\cr
		&0&0&\cdots&0
	}.
	\]
\end{df}

\begin{rmk}
	Assim como antes, o expoente em $E^{l,n}_{i,j}$ significa apenas que estamos transformando matrizes de $n$ linhas em matrizes de $l$ linhas.
\end{rmk}

\begin{ex}
	Dada uma matriz $M=\begin{pmatrix}
	5 & -8 & \sqrt 2\\
	4 & 4 & 0
	\end{pmatrix}\in M_{2\times 3}(\R)$, temos que
	\[E^{3,2}_{1,1}(M)=\begin{pmatrix}
	5 & -8 & \sqrt2\\
	0&0&0\\
	0&0&0
	\end{pmatrix}\quad\quad E^{3,2}_{1,2}(M)=\begin{pmatrix}
	4 & 4&0\\
	0&0&0\\
	0&0&0
	\end{pmatrix}\]
	\[E^{3,2}_{2,1}(M)=\begin{pmatrix}
	0&0&0\\
	5 & -8 & \sqrt2\\	
	0&0&0
	\end{pmatrix}\quad\quad E^{3,2}_{2,2}(M)=\begin{pmatrix}
	0&0&0\\
	4 & 4 & 0\\	
	0&0&0
	\end{pmatrix}\]
	\[E^{3,2}_{3,1}(M)=\begin{pmatrix}
	0&0&0\\
	0&0 & 0\\	
	5 & -8 & \sqrt2
	\end{pmatrix}\quad\quad E^{3,2}_{3,2}(M)=\begin{pmatrix}
	0&0&0\\
	0&0&0\\	
	4 & 4 & 0
	\end{pmatrix}\]
\end{ex}

Com isso, vamos simplesmente imitar o procedimento anterior:
\begin{df}
	Dada uma matriz $M\in M_{n\times m}(\R)$, $l\in \N$ e qualquer par de índices $i,j$ com $i\leq l$ e $j\leq n$, definimos o produto de $e^{l,n}_{i,j}$ com $M$ como sendo a matriz $e^{l,n}_{i,j}M$ dada por
	\[e^{l,n}_{i,j}M:=E^{l,n}_{i,j}(M).\]
\end{df}

\begin{ex}
	Continuando o exemplo acima, em que $M=\begin{pmatrix}
	5 & -8 & \sqrt 2\\
	4 & 4 & 0
	\end{pmatrix}\in M_{2\times 3}(\R)$, temos que
	\[\begin{pmatrix}
	1 & 0\\
	0&0\\
	0&0
	\end{pmatrix}\cdot \begin{pmatrix}
	5 & -8 & \sqrt 2\\
	4 & 4 & 0
	\end{pmatrix}=e^{3,2}_{1,1}M:=E^{3,2}_{1,1}(M)=\begin{pmatrix}
	5 & -8 & \sqrt2\\
	0&0&0\\
	0&0&0
	\end{pmatrix}\]
	\[\begin{pmatrix}
	0&1\\
	0&0\\
	0&0
	\end{pmatrix}\cdot \begin{pmatrix}
	5 & -8 & \sqrt 2\\
	4 & 4 & 0
	\end{pmatrix}=e^{3,2}_{1,2}M:=E^{3,2}_{1,2}(M)=\begin{pmatrix}
	4 & 4&0\\
	0&0&0\\
	0&0&0
	\end{pmatrix}\]
	\[\begin{pmatrix}
	0&0\\
	1&0\\
	0&0
	\end{pmatrix}\cdot \begin{pmatrix}
	5 & -8 & \sqrt 2\\
	4 & 4 & 0
	\end{pmatrix}=e^{3,2}_{2,1}M:=E^{3,2}_{2,1}(M)=\begin{pmatrix}
	0&0&0\\
	5 & -8 & \sqrt2\\	
	0&0&0
	\end{pmatrix}\]
	\[\begin{pmatrix}
	0 & 0\\
	0&1\\
	0&0
	\end{pmatrix}\cdot \begin{pmatrix}
	5 & -8 & \sqrt 2\\
	4 & 4 & 0
	\end{pmatrix}=e^{3,2}_{1,1}M:=E^{3,2}_{2,2}(M)=\begin{pmatrix}
	0&0&0\\
	4 & 4 & 0\\	
	0&0&0
	\end{pmatrix}\]
	\[\begin{pmatrix}
	0&0\\
	0&0\\
	1&0
	\end{pmatrix}\cdot \begin{pmatrix}
	5 & -8 & \sqrt 2\\
	4 & 4 & 0
	\end{pmatrix}=e^{3,2}_{3,1}M:=E^{3,2}_{3,1}(M)=\begin{pmatrix}
	0&0&0\\
	0&0&0\\	
	5 & -8 & \sqrt2
	\end{pmatrix}\]
	\[\begin{pmatrix}
	0&0\\
	0&0\\
	0&1
	\end{pmatrix}\cdot \begin{pmatrix}
	5 & -8 & \sqrt 2\\
	4 & 4 & 0
	\end{pmatrix}=e^{3,2}_{3,2}M:=E^{3,2}_{3,2}(M)=\begin{pmatrix}
	0&0&0\\
	0&0&0\\	
	4 & 4 & 0
	\end{pmatrix}\]
\end{ex}

\begin{exerc}
	Essa multiplicação coincide com a multiplicação clássica de matrizes? Se sim, você consegue mostrar que isso sempre é verdade? Se não, você consegue mostrar um caso onde falha?
\end{exerc}

Feito isso, vamos agora simplesmente repetir o que fizemos antes acima para ensinar a multiplicar matrizes de tamanho ``qualquer''.

\begin{ex}
	Considere as matrizes $M=\begin{pmatrix}
	a&b&c\\
	d&e&f
	\end{pmatrix}\in M_{2\times 3}(\R)$ e $N=\begin{pmatrix}
	\alpha &\beta\\
	\gamma & \delta\\
	\epsilon & \eta	
	\end{pmatrix}\in M_{3\times 2}(\R)$. Para calcular $MN$ podemos fazer como fizemos com as matrizes quadradas:
	
	Primeiro, vamos decompor $M$ em matrizes $e^{l,n}_{i,j}$:
	
	\begin{align*}
	M&=\begin{pmatrix}
	a&b&c\\
	d&e&f
	\end{pmatrix}\\
	&=\begin{pmatrix}
	a&0&0\\
	0&0&0
	\end{pmatrix}+
	\begin{pmatrix}
	0&b&0\\
	0&0&0
	\end{pmatrix}+
	\begin{pmatrix}
	0&0&c\\
	0&0&0
	\end{pmatrix}+
	\begin{pmatrix}
	0&0&0\\
	d&0&0
	\end{pmatrix}+
	\begin{pmatrix}
	0&0&0\\
	0&e&0
	\end{pmatrix}+
	\begin{pmatrix}
	0&0&0\\
	0&0&f
	\end{pmatrix}\\
	&=a\begin{pmatrix}
	1&0&0\\
	0&0&0
	\end{pmatrix}+b
	\begin{pmatrix}
	0&1&0\\
	0&0&0
	\end{pmatrix}+c
	\begin{pmatrix}
	0&0&1\\
	0&0&0
	\end{pmatrix}+d
	\begin{pmatrix}
	0&0&0\\
	1&0&0
	\end{pmatrix}+e
	\begin{pmatrix}
	0&0&0\\
	0&1&0
	\end{pmatrix}+f
	\begin{pmatrix}
	0&0&0\\
	0&0&1
	\end{pmatrix}
	\end{align*}e cada uma dessas matrizes $e^{l,n}_{i,j}$ sabemos multiplicar por $N$:
	\begin{align*}
		\begin{pmatrix}
		1&0&0\\
		0&0&0
		\end{pmatrix}\begin{pmatrix}
		\alpha &\beta\\
		\gamma & \delta\\
		\epsilon & \eta	
		\end{pmatrix}:=\begin{pmatrix}
		\alpha &\beta\\
		0 & 0	
		\end{pmatrix}\quad\quad&\begin{pmatrix}
		0&0&0\\
		1&0&0
		\end{pmatrix}\begin{pmatrix}
		\alpha &\beta\\
		\gamma & \delta\\
		\epsilon & \eta	
		\end{pmatrix}:=\begin{pmatrix}
		0 & 0\\
		\alpha &\beta
		\end{pmatrix}\\		
		\begin{pmatrix}
		0&1&0\\
		0&0&0
		\end{pmatrix}\begin{pmatrix}
		\alpha &\beta\\
		\gamma & \delta\\
		\epsilon & \eta	
		\end{pmatrix}:=\begin{pmatrix}
		\gamma &\delta\\
		0 & 0	
		\end{pmatrix}\quad\quad&\begin{pmatrix}
		0&0&0\\
		0&1&0
		\end{pmatrix}\begin{pmatrix}
		\alpha &\beta\\
		\gamma & \delta\\
		\epsilon & \eta	
		\end{pmatrix}:=\begin{pmatrix}
		0 & 0\\
		\gamma &\delta
		\end{pmatrix}\\		
		\begin{pmatrix}
		0&0&1\\
		0&0&0
		\end{pmatrix}\begin{pmatrix}
		\alpha &\beta\\
		\gamma & \delta\\
		\epsilon & \eta	
		\end{pmatrix}:=\begin{pmatrix}
		\epsilon &\eta\\
		0 & 0	
		\end{pmatrix}\quad\quad&\begin{pmatrix}
		0&0&0\\
		0&0&1
		\end{pmatrix}\begin{pmatrix}
		\alpha &\beta\\
		\gamma & \delta\\
		\epsilon & \eta	
		\end{pmatrix}:=\begin{pmatrix}
		0 & 0\\
		\epsilon &\eta
		\end{pmatrix}
	\end{align*}e cada uma dessas matrizes sabemos multiplicar por números:
	\begin{align*}
	a\begin{pmatrix}
	\alpha &\beta\\
	0 & 0	
	\end{pmatrix}=\begin{pmatrix}
	a\alpha &a\beta\\
	0 & 0	
	\end{pmatrix} \quad\quad& d\begin{pmatrix}
	0 & 0\\
	\alpha &\beta
	\end{pmatrix}=\begin{pmatrix}
	0 & 0\\
	d\alpha&d\beta
	\end{pmatrix}\\
	b\begin{pmatrix}
	\gamma &\delta\\
	0 & 0	
	\end{pmatrix}=\begin{pmatrix}
	b\gamma &b\delta\\
	0 & 0	
	\end{pmatrix} \quad\quad& e\begin{pmatrix}
	0 & 0\\
	\gamma &\delta
	\end{pmatrix}=\begin{pmatrix}
	0 & 0\\
	e\gamma &e\delta
	\end{pmatrix}\\
	c\begin{pmatrix}
	\epsilon &\eta\\
	0 & 0	
	\end{pmatrix}=\begin{pmatrix}
	c\epsilon &c\eta\\
	0 & 0	
	\end{pmatrix} \quad\quad& f\begin{pmatrix}
	0 & 0\\
	\epsilon &\eta
	\end{pmatrix}=\begin{pmatrix}
	0 & 0\\
	f\epsilon &f\eta
	\end{pmatrix}
	\end{align*}e como todas têm o mesmo tamanho, nós sabemos somar todas elas:
	\begin{gather*}	
	\begin{pmatrix}
	a\alpha &a\beta\\
	0 & 0	
	\end{pmatrix}+\begin{pmatrix}
	b\gamma &b\delta\\
	0 & 0	
	\end{pmatrix}+\begin{pmatrix}
	c\epsilon &c\eta\\
	0 & 0	
	\end{pmatrix}+\begin{pmatrix}
	0 & 0\\
	d\alpha&d\beta
	\end{pmatrix}+\begin{pmatrix}
	0 & 0\\
	e\gamma &e\delta
	\end{pmatrix}+\begin{pmatrix}
	0 & 0\\
	f\epsilon &f\eta
	\end{pmatrix}\\
	=\begin{pmatrix}
	a\alpha+b\gamma+c\epsilon & a\beta+b\delta+c\eta\\
	d\alpha+e\gamma+f\epsilon & d\beta+e\delta+f\eta
	\end{pmatrix}
	\end{gather*}e finalmente, vemos chamar esse resultado de $MN$.
\end{ex}

Finalmente podemos definir o produto de matrizes compatíveis:
\begin{df}
	Sejam $M\in M_{n\times m}(\R)$ e $N\in M_{m\times l}(\R)$. Definimos o produto de $M$ com $N$ como sendo a matriz $MN$ dada por
	\[MN:=\sum_{j=1}^m\sum_{i=1}^n\left(M_{i,j}e^{n,m}_{i,j}N\right)\]
\end{df}

\begin{exerc}
	Calcule, usando os dados do exemplo acima, $NM$, usando o método que preferir (isto é, o método clássico, ou o que estamos definindo) e compare o resultado que você obtiver com o $MN$ calculado acima.
\end{exerc}
\begin{exerc}
	Sejam $A$ e $B$ matrizes quaisquer, tais que ambos os produtos $AB$ e $BA$ existem. Isso significa que $AB=BA$? Se sim, tente provar por que isso é verdade. Se não, dê um contra-exemplo.
\end{exerc}

\begin{exerc}
	Tente provar as seguintes propriedades:
	\begin{itemize}
		\item[] \textbf{(Comutatividade da soma)} Para quaisquer duas matrizes $M,N\in M_{n\times m}(\R)$ temos que $M+N=N+M$;
		\item[] \textbf{(Elemento neutro da soma)} Para qualquer matriz $M\in M_{n\times m}(\R)$ existe uma única matriz $Z\in M_{n\times m}(\R)$ tal que $M+Z=M$. Vamos chamar essa matriz de \textbf{zero} e notar por 0;
		\item[] \textbf{(Inverso da soma)} Para qualquer matriz $M\in M_{n\times m}(\R)$ existe uma única matriz $N\in M_{n\times m}(\R)$ tal que $M+N=0$. Vamos chamar essa matriz de \textbf{inversa  aditiva de $M$} e notar por $-M$;
		\item[] \textbf{(Associatividade da soma) }Para quaisquer três matrizes $L,M,N\in M_{n\times m}(\R)$ temos que $L+(M+N)=(L+M)+N$;
		\item[] \textbf{(Comutatividade do produto por número) }Para qualquer número real $a\in \R$ e qualquer matriz $M\in M_{n\times m}(\R)$ temos que $aM=Ma$;
		\item[] \textbf{(Elemento neutro do produto por número)} Para qualquer matriz $M\in M_{n\times m}(\R)$ existe um único número $u\in \R$ tal que $uM=M$; 
		\item[] \textbf{(Associatividade do produto por número)} Para quaisquer dois números $a,b\in \R$ e qualquer matriz $M\in M_{n\times m}(\R)$, temos que $a(bM)=(ab)M$;
		\item[] \textbf{(Distributividade do produto por número sobre a soma)} Para quaisquer duas matrizes $M,N\in M_{n\times m}(\R)$ e qualquer número real $a\in \R$ temos que $a(M+N)=aM+aN$;
		\item[] \textbf{(Distributividade do produto por número sobre a soma de números)} Para quaisquer dois números reais $a,b\in \R$ e qualquer matriz $M\in M_{n\times m}(\R)$ temos que $(a+b)M=aM+bM$;
		\item[] \textbf{(Elemento neutro do produto)} Para qualquer matriz $M\in M_{n\times m}(\R)$ existe uma única matriz $I\in M_m(\R)$ tal que $MI=M=IM$. Vamos chamar essa matriz de \textbf{identidade};
		\item[] \textbf{(Associatividade do produto)} Para quaisquer três matrizes $L\in M_{n\times m}(\R)$, $M\in M_{m\times l}(\R)$ e $N\in M_{l\times k}(\R)$ temos que $(LM)N=L(MN)$;
		\item[] \textbf{(Associatividade do produto com produto por números)} Para quaisquer duas matrizes $M\in M_{n\times m}(\R)$ e $N\in M_{m\times l}(\R)$ e qualquer número real $a\in \R$ temos que $a(MN)=(aM)N$;
		\item[] \textbf{(Distributividade do produto sobre a soma)} Para quaisquer duas matrizes $L,M\in M_{n\times m}(\R)$ qualquer matriz $N\in M_{m\times l}(\R)$ e qualquer matriz $K\in M_{o\times n}(\R)$ temos que $K(L+M)=KL+KM$ e $(L+M)N=LN+MN$.
	\end{itemize}
\end{exerc}
\pagebreak

\section{Sistemas Lineares}

Vamos agora tentar dar uma primeira justificativa para o estudo de matrizes reais: A resolução de sistemas lineares.

\begin{df}
	Um \textbf{sistema linear de $n$ equações em $m$ variáveis} consiste em uma coleção de $n$ equações, em que a $i$-ésima equação é da forma $a_{i,1}x_1+a_{i,2}x_2+\cdots+a_{i,m}x_m=b_i$, em que $\{a_{i,j}\}_{i,j\in \N}$ e $\{b_i\}_{i\in \N}$ são números reais. Vamos denotar sistemas dessa forma por
	\[
	\begin{cases}
	a_{1,1}x_1+a_{1,2}x_2+&\cdots\quad+a_{1,m}x_m=b_1\\
	a_{2,1}x_1+a_{2,2}x_2+&\cdots\quad+a_{2,m}x_m=b_2\\
	&\ \vdots\\
	a_{n,1}x_1+a_{n,2}x_2+&\cdots\quad+a_{n,m}x_m=b_n
	\end{cases}
	\]
\end{df}

\begin{rmk}
	A motivação por trás de chamarmos tais sistemas de \textbf{lineares} ficará para um capítulo posterior.
\end{rmk}

\begin{ex}
	Considere o sistema de equações
	\[\begin{cases}
	5x+4y-7z=0\\
	3x+y-z=9.
	\end{cases}\] Ele é composto de duas equações ($5x+4y-7z=0$ e $3x+y-z=9$), ambas são lineares e ambas têm três variáveis ($x$, $y$ e $z$). Assim, o sistema acima é um sistema linear de duas equações e três incógnitas.
	
	\tcblower
	
	Contudo, o sistema
	\[
	\begin{cases}
	8x+y+z=2\\
	-4x+y+z^2=4
	\end{cases}
	\]não é linear. Ele é composto de duas equações e três variáveis - contudo, a segunda equação \textbf{não é linear}.
\end{ex}

Em outras palavras, sistemas lineares são sistemas de equações onde só aparecem somas de números multiplicados a variáveis - nada de funções trigonométricas, nada de potências, nada de funções exponenciais ou qualquer outro tipo de funções. Apenas multiplicação por números.

Se voltarmos para a definição de sistemas lineares de $n$ equações e $m$ incógnitas, vamos ver que cada equação tem exatamente $m$ números (chamados coeficientes) multiplicando as variáveis. Assim, como temos $n$ equações, temos $n\times m$ coeficientes no total. Isso sugere que podemos pegar um sistema
\[
\begin{cases}
a_{1,1}x_1+a_{1,2}x_2+&\cdots\quad+a_{1,m}x_m=b_1\\
a_{2,1}x_1+a_{2,2}x_2+&\cdots\quad+a_{2,m}x_m=b_2\\
&\ \vdots\\
a_{n,1}x_1+a_{n,2}x_2+&\cdots\quad+a_{n,m}x_m=b_n
\end{cases}
\]e representá-lo por uma matriz, da seguinte forma:

Primeiro, temos uma matriz
\[A=\begin{pmatrix}
a_{1,1} & a_{1,2} & \cdots & a_{1,m}\\
a_{2,1} & a_{2,2} & \cdots & a_{2,m}\\
\vdots & \vdots & \ddots & \vdots\\
a_{n,1} & a_{n,2} & \cdots & a_{n,m}
\end{pmatrix}\]chamada de \textbf{matriz de coeficientes} que faz exatamente o que o nome diz - coleta todos os coeficientes do sistema.

Em seguida, temos uma matriz
\[X=\begin{pmatrix}
x_1\\
x_2\\
\vdots\\
x_m
\end{pmatrix}\]chamada de \textbf{matriz de variáveis} que, novamente, é auto-descritiva: ela contém todas as variáveis do sistema.

Por fim, temos uma matriz
\[B=\begin{pmatrix}
b_1\\
b_2\\
\vdots\\
b_n
\end{pmatrix}\] chamada de \textbf{matriz resultante} que, mais uma vez, é simplesmente a matriz que contém todos os resultados do sistema.

Por que isso é legal? Bom, primeiro vamos notar que $A\in M_{n\times m}(\R)$, $X$ é uma matriz $m\times 1$ com entradas que são variáveis, e que $B\in M_{n,1}(\R)$ . Agora, lembrando da multiplicação de matrizes, não é difícil ver que podemos multiplicar $AX$. Mas o que seria, por exemplo, o elemento na posição $i,j$ de $AX$? Bom, por definição,

\[(AX)_{i,j}=\sum_{l=1}^m A_{i,l}X_{l,j}=\sum_{l=1}^m a_{i,l}X_{l,j}\]e como $AX\in M_{n,1}(\R)$, $j=1$, de forma que podemos continuar:
\[(AX)_{i,j}=(AX)_{i,1}=\sum_{l=1}^m a_{i,l}X_{l,j}=\sum_{l=1}^m a_{i,l}X_{l,1}=\sum_{l=1}^m a_{i,l}x_l,\]e expandindo temos
\[(AX)_{i,1}=a_{i,1}x_1+a_{i,2}x_2+\cdots+a_{i,m}x_m\]que é exatamente $b_i$. Mas $b_i=B_{i,1}$. Disso temos que $(AX)_{i,1}=B_{i,1}$ - ou seja, as matrizes $AX$ e $B$ são iguais em \textbf{todas} as entradas e, portanto, são \textbf{iguais}.

Em outras palavras, ao escrever um sistema usando as matrizes $A$, $X$ e $B$, vemos que o sistema pode ser descrito como a igualdade de matrizes $AX=B$. Ou, dito de outra maneira - se $A'\in M_{n\times m}(\R)$, $X'$ é uma matriz $m\times 1$ com entradas que são variáveis, e $B'\in M_{n,1}(\R)$ são matrizes quaisquer, então uma equação matricial $A'X'=B'$ determina um único sistema linear.

\subsection{Escalonamentos}

Contudo, resolver sistemas lineares não é tarefa fácil. Vamos aqui apresentar uma técnica - chamada \textbf{escalonamento} - para resolver (se possível!) um sistema linear. Antes disso, porém, precisamos entender o que significa \textit{resolver um sistema linear}.

\begin{df}
	Dado um sistema linear $AX=B$, dizemos que \textbf{$C$ é uma solução para o sistema} se $AC=B$.
\end{df}

\begin{ex}
	Dado o sistema linear
	\[\begin{pmatrix}
	1 & 1 & 1\\
	0 & 1 & 1\\
	0 & 0 & 1
	\end{pmatrix}\begin{pmatrix}
	x\\
	y\\
	z
	\end{pmatrix}=\begin{pmatrix}
	6\\
	3\\
	1
	\end{pmatrix},\] temos que uma solução do sistema é a matriz
	\[\begin{pmatrix}
	3\\
	2\\
	1
	\end{pmatrix}.\]
	
	De fato,
	\[\begin{pmatrix}
	1 & 1 & 1\\
	0 & 1 & 1\\
	0 & 0 & 1
	\end{pmatrix}\begin{pmatrix}
	3\\
	2\\
	1
	\end{pmatrix}=\begin{pmatrix}
	3+2+1\\
	2+1\\
	1
	\end{pmatrix}=\begin{pmatrix}
	6\\
	3\\
	1
	\end{pmatrix}.\]
	
	Afirmamos ainda que essa solução é única.
	
	De fato, suponha que temos alguma matrix $\begin{pmatrix}
	a\\b\\c
	\end{pmatrix}$ tal que 
	\[\begin{pmatrix}
	1 & 1 & 1\\
	0 & 1 & 1\\
	0 & 0 & 1
	\end{pmatrix}\begin{pmatrix}
	a\\b\\c
	\end{pmatrix}=\begin{pmatrix}
	6\\
	3\\
	1
	\end{pmatrix}.\] Então teríamos três equações: $a+b+c=6$, $b+c=3$ e $c=1$ Como $c=1$ e $b+c=3$, segue que $b=2$. Similarmente, como $b+c=3$ e $a+b+c=6$, segue que $a=3$ -  ou seja, a matriz dada é, de fato, a única solução do sistema.
\end{ex}

\begin{exerc}
	Encontre uma solução do sistema
	\[\begin{pmatrix}
	1 & 0 & 0\\ 1&1&0\\1&1&1
	\end{pmatrix}\begin{pmatrix}
	x\\y\\z
	\end{pmatrix}=\begin{pmatrix}
	6\\13\\21
	\end{pmatrix}.\] Essa solução é única? Se não, encontre outra. Se sim, prove.
\end{exerc}

Vamos agora lembrar de como resolvíamos sistemas de equações lineares antes:

\begin{ex}
	Considere o sistema
	\[\begin{cases}
	4x+2y=0\\
	-4x+9y=22
	\end{cases}.\] Nós podemos perceber que o coeficiente de $x$ em ambas as equações é o mesmo, com o sinal invertido. Isso nos diz que se somarmos as equações, obteremos uma nova equação com o coeficiente de $x$ sendo 0
	\[
	\begin{array}[b]{lr}
	&4x+2y=0\\
	+&-4x+9y=22\\
	\hline
	&0x+11y=22
	\end{array}
	\]com isso, conseguimos resolver a equação $11y=22$, que tem como única solução $y=2$. Agora, com o valor de $y$ em mãos, podemos escolher qualquer equação original e resolvê-la - por exemplo a primeira
	\[\begin{array}{rl}
		4x+2y=0&\\
		4x+2\cdot2=0 & \text{(usando }y=2\text{)}\\
		4x+4=0&\\
		4(x+1)=0 & \text{(colocando 4 em evidência)}\\
		x+1=0 &\\
		x=-1,
	\end{array}\] e agora afirmamos que a solução do sistema original é $(-1,2)$.
	
	O que fizemos, em suma, foi trocar o sistema
	\[\begin{cases}
	4x+2y=0\\
	-4x+9y=22
	\end{cases}\]pelo sistema
	\[
	\begin{cases}
	4x+2y=0\\
	11y=22,
	\end{cases}\]resolver esse segundo sistema e afirmar que essa solução obtida também é solução do sistema original.
	
	\tcblower
	
	Pensando do ponto de vista de matrizes, vamos chamar
	\[A=\begin{pmatrix}
	4 &2\\-4 &9
	\end{pmatrix}\]a matriz de coeficientes do sistema original e 
	\[A'=\begin{pmatrix}
	4 & 2\\ 0 & 11
	\end{pmatrix}\]a matriz de coeficientes do sistema que obtemos somando as duas equações do sistema original.
	
	Mas como obtivemos a matriz $A'$? Vamos analisar ela por partes:
	
	\begin{itemize}
		\item A linha 1 de $A'$ é simplesmente a linha 1 de $A$;
		\item A linha 2 de $A'$ é simplesmente a soma das linhas 1 e 2 de $A$.
	\end{itemize}

Ou seja, se escrevermos
\[A'=\begin{pmatrix}
4 & 2\\0&0
\end{pmatrix}+\begin{pmatrix}
0&0\\0&11
\end{pmatrix}\]podemos ver que $\left(\begin{smallmatrix}
4&2\\0&0
\end{smallmatrix}\right)=e^2_{1,1}A$. Além disso, como já dissemos, 
\[\begin{pmatrix}
0&0\\0&11
\end{pmatrix}=\begin{pmatrix}
0&0\\4&2
\end{pmatrix}+\begin{pmatrix}
0&0\\-4&9
\end{pmatrix}=e^2_{2,1}A+e^2_{2,2}A,\]ou seja,
\[A'=\begin{pmatrix}
4 & 2\\0&0
\end{pmatrix}+\begin{pmatrix}
0&0\\0&11
\end{pmatrix}=e^2_{1,1}A+e^2_{2,1}A+e^2_{2,2}A=(e^2_{1,1}+e^2_{2,1}+e^2_{2,2})A\]que é simplesmente
\[A'=\begin{pmatrix}
1 & 0\\1 & 1
\end{pmatrix}A.\]Em outras palavras, nós multiplicamos $A$ por uma soma de matrizes $e^n_{i,j}$ e obtivemos uma matriz $A'$ com as mesmas soluções de $A$.
\end{ex}
\begin{ex}
	Similarmente ao exemplo anterior, considere o sistema linear
	\[\begin{cases}
	-7x+24y=8\\
	x-3y=1
	\end{cases}.\]Para resolvê-lo, geralmente somaríamos a linha 1 com sete vezes a linha 2:
	\[
	\begin{array}[b]{lr}
	&-7x+24y=8\\
	+&7(x-3y=1)\\
	\hline
	&0x+3y=15
	\end{array}
	\]e repetindo acima, podemos encontrar $y=5$. Agora, com $y$ em mãos, podemos escolher alguma das equações originais (por exemplo, a segunda) e resolvê-la, obtendo $x=16$, e, portanto, a solução é o par $(16,5)$.
	
	\tcblower
	
	Novamente, temos duas matrizes: A matriz do sistema original
	\[A=\begin{pmatrix}
	-7 & 24\\1&-3
	\end{pmatrix}\]e a matriz que obtivemos somando a primeira linha com sete vezes a segunda linha
	\[A'=\begin{pmatrix}
	1&-3\\0&3
	\end{pmatrix}.\]Será que conseguimos expressar $A'$ como um produto de alguma matriz por $A$ - como fizemos antes?
	
	Vamos começar desmembrando $A'$:
	\[A'=\begin{pmatrix}
	1 & -3 \\0&0
	\end{pmatrix}+\begin{pmatrix}
	0&0\\0&3
	\end{pmatrix}\]em que o primeiro pedaço não é nada mais que uma cópia da segunda linha de $A$ - ou seja, $e^2_{1,2}A$ - e o segundo pedaço é a soma da primeira linha de $A$ com sete vezes a segunda linha de $A$ - ou seja, 
	\[\begin{pmatrix}
	0&0\\0&3
	\end{pmatrix}=\begin{pmatrix}
	0&0\\-7&24
	\end{pmatrix}+7\begin{pmatrix}
	0&0\\1&-3
	\end{pmatrix}=e^2_{2,1}A+7e^2_{2,2}A\]e, portanto, 
	\[A'=e^2_{1,2}A+e^2_{2,1}A+7e^2_{2,2}A=(e^2_{1,2}+e^2_{2,1}+7e^2_{2,2})A,\]ou seja,
	\[A'=\begin{pmatrix}
	0&1\\
	1 & 7
	\end{pmatrix}A,\]como queríamos.
	
	Neste caso, multiplicamos $A$ por uma matriz que era soma de múltiplos de matrizes $e^n_{i,j}$ e ainda obtivemos uma matriz que tem as mesmas soluções.
\end{ex}

\begin{rmk}
	Cuidado! Nem toda multiplicação de $A$ por matrizes $e^n_{i,j}$ preserva soluções. Por exemplo, fazendo $\begin{pmatrix}
	0&0\\1&1
	\end{pmatrix}$ vezes $A=\begin{pmatrix}
	-7 & 24\\1&-3
	\end{pmatrix}$ obtemos a matriz $\begin{pmatrix}
	1 & -3\\1&-3
	\end{pmatrix}$ que claramente não tem as mesmas soluções que $A$.
	
	Em breve veremos condições para que isso não aconteça.
\end{rmk}

Mais à frente veremos exatamente \textit{porquê} isso é verdade. Por agora, nos basta focar nessas operações.

\begin{df}
	Um \textbf{escalonamento} de uma matriz $A\in M_{n\times m}(\R)$ qualquer, é uma matriz $A'$ que pode ser obtida de $A$ por composições das seguintes operações:
	\begin{itemize}
		\item Somar uma linha de $A$ a um múltiplo de outra linha de $A$;
		\item Trocar duas linhas de $A$;
		\item Multiplicar uma linha inteira de $A$ por um mesmo número.
	\end{itemize}
\end{df}

Pelos exemplos acima, podemos ver que sempre podemos realizar escalonamentos de $A$ por multiplicações $XA$, em que $X$ é uma soma de múltiplos de matrizes $e^n_{i,j}$.

\begin{exerc}
	Construa uma matriz $X\in M_2(\R)$ diferente de 0 que seja soma de matrizes $e^2_{i,j}$ tal que $XA$ \textbf{não} seja um escalonamento de $A$ (ou seja, não tenha as mesmas soluções de $A$) para qualquer matriz $A\in M_{2\times m}(\R)$.
\end{exerc}

Finalmente, podemos enunciar o teorema mais forte desta seção:

\begin{theorem}\label{thm:sol-escsol}
	Uma matriz $A$ possui solução se, e somente se, todo escalonamento de $A$ possui solução.
\end{theorem}

Não vamos demonstrar esse teorema agora, mas vamos chamar a atenção para o seguinte corolário:

\begin{cor}\label{cor:sol-id}
	Se, a matriz identidade é um escalonamento de $A$, então $A$ possui solução única.
\end{cor}

\begin{rmk}
	Em matemática, um \textbf{corolário} é um resultado que segue imediatamente de um resultado anterior. Então estamos afirmando que o
	\Cref{cor:sol-id} segue imediatamente do \Cref{thm:sol-escsol}.
\end{rmk}

\begin{exerc}
	Prove o \Cref{cor:sol-id} assumindo que o \Cref{thm:sol-escsol} seja verdade.
\end{exerc}

\bigskip
\subsection{Resolvendo sistemas lineares}

Vamos finalmente aprender a obter soluções usando escalonamentos. Para isso, considere o exemplo abaixo:

\begin{ex}
	Sejam $$A=\begin{pmatrix}
	1&1\\1&1\\1&-1\\1&-1
	\end{pmatrix},\quad B=\begin{pmatrix}
	1\\-1\\1\\-1
	\end{pmatrix}$$ e $AX=B$ um sistema linear. Para escalonar $A$ vamos fazer:
	\begin{alignat*}{4}
	\begin{pmatrix}
	1&1\\1&1\\1&-1\\1&-1
	\end{pmatrix}\quad&\rightsquigarrow\quad\bordermatrix{&&\cr&1&1\cr l_2-l_1&0&0\cr l_3-l_1&0&-2\cr l_4-l_1&0&-2}\quad&\rightsquigarrow\quad\bordermatrix{&&\cr&1&1\cr&0&0\cr\frac{l_3}{-2}&0&1\cr&0&-2}\quad&\rightsquigarrow\quad\bordermatrix{
	&&\cr l_1-l_3&1&0\cr &0&0\cr &0&1\cr l_4+2l_3 & 0 & 0
	}
	\end{alignat*}no primeiro passo multiplicamos $A$ por $P_1=\begin{pmatrix}
	1&0&0&0\\-1&1&0&0\\-1&0&1&0\\-1&0&0&1
	\end{pmatrix}$, no segundo por $P_2=\begin{pmatrix}
	1&0&0&0\\0&1&0&0\\0&0&\frac{1}{-2}&0\\0&0&0&1
	\end{pmatrix}$ e no terceiro por $P_3=\begin{pmatrix}
	1&0&-1&0\\0&1&0&0\\0&0&1&0\\0&0&2&1
	\end{pmatrix}$, obtendo a matrix escalonada $P_3P_2P_1A$.
	
	Se fizermos, agora, $(P_3P_2P_1A)X$ e lembrarmos que o produto de matrizes é associativo, teremos
	\[(P_3P_2P_1A)X=P_3P_2P_1(AX)=P_3P_2P_1B\]onde do lado direito aparece $B$ \underline{escalonada pelas mesmas matrizes que $A$ foi escalonada}. Além disso, a equação $(P_3P_2P_1A)X=P_3P_2P_1B$ nos diz que a solução ($X$) do sistema original continua sendo solução do sistema escalonado. Assim, podemos calcular $P_3P_2P_1B$:
	\[P_1B=\bordermatrix{&\cr l_1&1\cr l_2-l_1&-2\cr l_3-l_1&0\cr l_4-l_1&-2}
	\]
	\[P_2(P_1B)=\bordermatrix{&\cr l_1&1\cr l_2&-2\cr \frac{l_3}{-2}&0\cr l_4&-2}
	\]
	\[P_3(P_2P_1B)=\bordermatrix{&\cr l_1-l_3&1\cr l_2&-2\cr l_3&0\cr l_4+2l_3&-2},\]ou seja,
	\[\begin{pmatrix}
	1&0\\0&0\\0&1\\0&0
	\end{pmatrix}\begin{pmatrix}
	x\\y
	\end{pmatrix}=\begin{pmatrix}
	1\\-2\\0\\-2
	\end{pmatrix}\quad\text{ e }\quad\begin{pmatrix}
	1&1\\1&1\\1&-1\\1&-1
	\end{pmatrix}\begin{pmatrix}
	x\\y
	\end{pmatrix}=\begin{pmatrix}
	1\\-1\\1\\-1
	\end{pmatrix}\]têm as mesmas soluções.
	
	Contudo, o sistema escalonado nos diz que (pela última linha) $0x+0y=-2$, e nós sabemos que $0a=0$ para qualquer $a\in \R$. Em particular, não existe $(x,y)\in \R^2$ tal que $0x+0y=-2$ - ou seja, o sistema escalonado \textbf{não admite solução} e, portanto, o sistema original também não.
\end{ex}

\begin{rmk}
	De maneira geral, sempre que, ao escalonar uma matriz, obtivermos uma linha de zeros sendo igual a algo diferente de zero, podemos parar o escalonamento e concluir, imediatamente, que o sistema não possui soluções.
\end{rmk}

\begin{df}
	Seja $AX=B$ um sistema linear. Denotamos por $\begin{augmatrix}{c:c}
	A&B
	\end{augmatrix}$ a \textbf{matrix aumentada do sistema} obtida de $A$ simplesmente colando uma cópia de $B$ à direita.
\end{df}

\begin{ex}
	No exemplo anterior, em que $A=\begin{pmatrix}
	1&1\\1&1\\1&-1\\1&-1
	\end{pmatrix}$ e $B=\begin{pmatrix}
	1\\-1\\1\\-1
	\end{pmatrix}$, a \textbf{matrix aumentada} é
	\[A\mid B=\left(\begin{array}{c c:c}
	1&1&1\\1&1&-1\\1&-1&1\\1&-1&-1
	\end{array}\right).\]A vantagem de se trabalhar com a matriz aumentada é a seguinte: Ao invés de escalonar $A$ e depois repetir o procedimento em $B$, nós escalonamos ambas as matrizes ao mesmo tempo:
	\begin{alignat*}{4}
	\begin{augmatrix}{cc:c}
	1&1&1\\1&1&-1\\1&-1&1\\1&-1&-1
	\end{augmatrix}\quad&\rightsquigarrow\quad\begin{augmatrix}{cc:c}
	1&1&1\\0&0&-2\\0&-2&0\\0&-2&-2
	\end{augmatrix}\quad&\rightsquigarrow\quad\begin{augmatrix}{cc:c}
	1&1&1\\0&0&-2\\0&1&0\\0&-2&-2
	\end{augmatrix}\quad&\rightsquigarrow\quad\begin{augmatrix}{cc:c}
	1&0&0\\0&0&-2\\0&1&0\\0&0&-2
	\end{augmatrix}.
	\end{alignat*}
	
	Por exemplo, para o sistema linear
	\[\begin{pmatrix}
	1&1\\1&-1
	\end{pmatrix}\begin{pmatrix}
	x\\y
	\end{pmatrix}=\begin{pmatrix}
	1\\1
	\end{pmatrix}\]temos a matriz aumentada $\begin{augmatrix}{cc:c}
	1&1&1\\1&-1&1
	\end{augmatrix}$ que podemos escalonar e obter
	\[\begin{augmatrix}{cc:c}
	1&1&1\\1&-1&1
	\end{augmatrix}\rightsquigarrow\begin{augmatrix}{cc:c}
	1&1&1\\0&-2&0
	\end{augmatrix}\rightsquigarrow\begin{augmatrix}{cc:c}
	1&1&1\\0&1&0
	\end{augmatrix}\rightsquigarrow\begin{augmatrix}{cc:c}
	1&0&1\\0&1&0
	\end{augmatrix}\]e podemos dizer imediatamente que as equações são $1x=1$ e $1y=0$, donde vemos que a única solução do sistema é $(1,0)\in \R^2$.
\end{ex}

\begin{ex}
	Vamos resolver um último sistema antes de avançar:
	\[\begin{pmatrix}
	1 & 2 & 1 & 1\\1&3&-1&2
	\end{pmatrix}\begin{pmatrix}
	x\\y\\z\\w
	\end{pmatrix}=\begin{pmatrix}
	1\\3
	\end{pmatrix},\]cuja matriz aumentada é $\begin{augmatrix}{cccc:c}
	1&2&1&1&1\\
	1&3&-1&2&3
	\end{augmatrix}$. Escalonando obtemos
	\[\begin{augmatrix}{cccc:c}
	1&2&1&1&1\\
	1&3&-1&2&3
	\end{augmatrix}\rightsquigarrow\begin{augmatrix}{cccc:c}
	1&2&1&1&1\\0&1&-2&1&2
	\end{augmatrix}\rightsquigarrow\begin{augmatrix}{cccc:c}
	1&0&5&-1&-3\\0&1&-2&1&2
	\end{augmatrix}\]que nos dá as equações $x+5z-w=-3$ e $y-2z+w=2$. Não temos mais restrições no sistema, então todos os pontos que satisfazem a essas equações são soluções.
	
	Para denotar, então, essa solução, note que podemos isolar $x$ e $y$ acima, obtendo $x=-5z+w-3$ e $y=2z-w+2$. Assim, vemos que \textit{para cada valor distinto de $z$ e $w$ temos uma solução diferente do sistema}. Por exemplo, se $w=z=0$, temos que $x=-3$ e $y=2$. Substituindo, então, o ponto $(-3,2,0,0)$ no sistema original vemos que
	\[(-3)+2(2)+(0)+(0)=4-3=1\]
	\[(-3)+3(2)-1(0)+2(0)=6-3=3\]e, de fato, o ponto $(-3,2,0,0)$ é solução.
	
	De maneira geral, como para cada valor possível de $z$ e $w$ temos uma solução, vamos notar o conjunto de todas as soluções por 
	\[S=\{(-5z+w-3,2z-w+2,z,w)\in\R^4\mid z\in \R,w\in \R\}.\]
\end{ex}

\begin{rmk}
	No caso do exemplo acima dizemos que $z$ e $w$ são \textbf{variáveis livres}. A ideia é que elas não têm um valor fixo, mas são ``livres'' para assumir o valor que quiserem e o sistema continua tendo solução.
\end{rmk}

\begin{exerc}
	Use os exemplos acima para responder: Seja $AX=B$ um sistema $n\times m$, com $n\neq m$. O que podemos dizer sobre a existência de soluções do sistema?
	
	E no caso das matrizes quadradas, i.e., $n=m$?
\end{exerc}

\subsection{Sistemas homogêneos}

\begin{df}
	Um sistema linear $AX=B$ é dito \textbf{homogêneo} se $B=0$.
\end{df}

Poderíamos nos perguntar a importância de destacar sistemas homogêneos dentre todos os sistemas lineares. Contudo, se nos lembrarmos do que já fizemos, vamos ver que um sistema não possui solução se, e somente se, sua forma escalonada possui uma linha de zeros, com a entrada correspondente na matriz resultante diferente de zero - por exemplo, algo da forma
\[\begin{pmatrix}
1&0\\
0&0\\
\end{pmatrix}\begin{pmatrix}
x\\y
\end{pmatrix}=\begin{pmatrix}
1\\1
\end{pmatrix}.\] Mas ao lidar com sistemas homogêneos, todas as entradas da matriz resultante são nulas - ou seja, mesmo que durante o processo de escalonamento alguma linha se anule, ainda assim o sistema continua tendo solução.

\begin{prop}
	Todo sistema linear homogêneo possui solução.
\end{prop}

Um jeito óbvio de ver isso é:

\begin{ex}
	Considere o sistema linear homogêneo
	\[\begin{pmatrix}
	1&7\\8&9\\17&-10\\25&3
	\end{pmatrix}\begin{pmatrix}
	x\\y
	\end{pmatrix}=\begin{pmatrix}
	0\\0
	\end{pmatrix}.\] Ao invés de tentar escalonar o sistema, note que estamos multiplicando a matrix $\begin{pmatrix}
	1&7\\8&9\\17&-10\\25&3
	\end{pmatrix}$ por uma outra matriz, e queremos que o resultado seja $0$. Mas já sabemos que $A0=0$ para qualquer matriz $A$ - ou seja, tomando $X_0=\begin{pmatrix}
	0\\0
	\end{pmatrix}$ vemos que $X_0$ é solução do sistema.
\end{ex}

De fato, todo sistema homogêneo possui a solução $(0,0)$ - por isso ela é chamada de \textbf{solução trivial}.

\begin{ex}
	Vamos voltar a um exemplo que fizemos acima:
	\[\begin{pmatrix}
	1 & 2 & 1 & 1\\1&3&-1&2
	\end{pmatrix}\begin{pmatrix}
	x\\y\\z\\w
	\end{pmatrix}=\begin{pmatrix}
	1\\3
	\end{pmatrix}.\] Já vimos que o conjunto de soluções desse sistema, $S$, é da forma
	\[S=\{(x,y,z,w)\in \R^4\mid x=-5z+w-3,y=2z-w+2\}.\] Vamos, agora, resolver o sistema homogêneo associado ao sistema acima, em que nós simplesmente trocamos a matriz resultante pela matriz de zeros:
	\[\begin{pmatrix}
	1 & 2 & 1 & 1\\1&3&-1&2
	\end{pmatrix}\begin{pmatrix}
	x\\y\\z\\w
	\end{pmatrix}=\begin{pmatrix}
	0\\0
	\end{pmatrix}.\]
	
	Escalonando obtemos
	\[\begin{augmatrix}{cccc:c}
	1&2&1&1&0\\
	1&3&-1&2&0
	\end{augmatrix}\rightsquigarrow\begin{augmatrix}{cccc:c}
	1&2&1&1&0\\0&1&-2&1&0
	\end{augmatrix}\rightsquigarrow\begin{augmatrix}{cccc:c}
	1&0&5&-1&0\\0&1&-2&1&0
	\end{augmatrix}\]que nos dá as equações $x+5z-w=0$ e $y-2z+w=0$. Não temos mais restrições no sistema, então todos os pontos que satisfazem a essas equações são soluções - ou seja, o conjunto de soluções do sistema homogêneo, $S_0$, é da forma
	\[S_0=\{(x,y,z,w)\in\R^4\mid x=-5z+w,y=2z-w\}.\]
	
	Comparando os dois conjuntos, não é difícil ver que se $X_0$ é uma solução do sistema homogêneo, então $X_0+\begin{pmatrix}
	-3\\2\\0\\0
	\end{pmatrix}$ é solução do sistema original.
	
	Por exemplo, escolhendo $X_0=\begin{pmatrix}
	1\\-1\\0\\1
	\end{pmatrix}$ vamos primeiro conferir que $X_0$ é, de fato, solução do sistema homogêneo:
	\[\begin{pmatrix}
		1 & 2 & 1 & 1\\1&3&-1&2
	\end{pmatrix}\begin{pmatrix}
	1\\-1\\0\\1
\end{pmatrix}=\begin{pmatrix}
1\cdot 1+2\cdot(-1)+1\cdot 0+1\cdot 1\\
1\cdot 1+3\cdot(-1)+(-1)\cdot 0+2\cdot 1
\end{pmatrix}=\begin{pmatrix}
1-2+0+1\\1-3+0+2
\end{pmatrix}=\begin{pmatrix}
0\\0
\end{pmatrix}.\] Agora vamos somar $\begin{pmatrix}
-3\\2\\0\\0
\end{pmatrix}$ a $X_0$, obtendo uma nova matriz
\[X=\begin{pmatrix}
1\\-1\\0\\1
\end{pmatrix}+\begin{pmatrix}
-3\\2\\0\\0
\end{pmatrix}=\begin{pmatrix}
-2\\1\\0\\1
\end{pmatrix}.\] Afirmamos que $X$ é solução do sistema original. Vamos conferir:
\[\begin{pmatrix}
1&2&1&1\\1&3&-1&2
\end{pmatrix}\begin{pmatrix}
-2\\1\\0\\1
\end{pmatrix}=\begin{pmatrix}
1\cdot(-2)+2\cdot 1+1\cdot 0+1\cdot1\\
1\cdot(-2)+3\cdot1+(-1)\cdot 0+2\cdot1
\end{pmatrix}=\begin{pmatrix}
-2+2+0+1\\-2+3+0+2
\end{pmatrix}=\begin{pmatrix}
1\\3
\end{pmatrix},\]ou seja, $X$ é de fato uma solução do sistema original.

Será que isso acontece por acaso?
\end{ex}

\begin{prop}
	Dado um sistema linear $AX=B$, uma solução $X_0$ do sistema homogêneo $AX=0$, e $X_1$ uma solução qualquer do sistema $AX=B$, então $X_0+X_1$ é solução de $AX=B$.
\end{prop}
\begin{proof}
	Se $X_0$ é solução do sistema homogêneo, temos que $AX_0=0$. Similarmente, se $X_1$ é solução do sistema $AX=B$, temos que $AX_1=B$. Agora, como o produto de matrizes distribui sobre a soma, temos que
	\[A(X_0+X_1)=AX_0+AX_1=0+B=B,\] ou seja, $X_0+X_1$ também é solução do sistema $AX=B$.
\end{proof}

Mas será que qualquer solução do sistema pode ser calculada assim - sabendo uma solução do sistema homogêneo e uma solução do sistema original? A resposta é sim, e vamos mostrar em seguida, mas antes, vamos fazer um exemplo:

\begin{ex}
	Ainda no exemplo anterior, considere o sistema
	\[\begin{pmatrix}
	1 & 2 & 1 & 1\\1&3&-1&2
	\end{pmatrix}\begin{pmatrix}
	x\\y\\z\\w
	\end{pmatrix}=\begin{pmatrix}
	1\\3
	\end{pmatrix}.\] Já vimos que $X_0=\begin{pmatrix}
	1\\-1\\0\\1
	\end{pmatrix}$ é solução do sistema homogêneo. Vamos escolher qualquer outra solução do sistema original, por exemplo, $X_1=\begin{pmatrix}
	-13\\6\\2\\0
	\end{pmatrix}$ (verifique que isso é uma solução!). Será que existe alguma solução $X_2$ tal que $X_1=X_0+X_2$? Ora, resolver isso é o mesmo que resolver $X_2=X_1-X_0$ - o que nós já sabemos fazer:
	\[X_2=\begin{pmatrix}
	-13\\6\\2\\0
	\end{pmatrix}-\begin{pmatrix}
	1\\-1\\0\\1
	\end{pmatrix}=\begin{pmatrix}
	-14\\7\\2\\-1
	\end{pmatrix}.\] Nos resta mostrar que $X_2$ é solução:
	\[\begin{pmatrix}
	1 & 2 & 1 & 1\\1&3&-1&2
	\end{pmatrix}\begin{pmatrix}
	-14\\7\\2\\-1
	\end{pmatrix}=\begin{pmatrix}
	1\cdot(-14)+2\cdot 7+1\cdot2+1\cdot(-1)\\
	1\cdot(-14)+3\cdot7+(-1)\cdot2+2\cdot(-1)
	\end{pmatrix}=\begin{pmatrix}
	-14+14+2-1\\-14+21-2-2
	\end{pmatrix}=\begin{pmatrix}
	1\\3
	\end{pmatrix},\]como queríamos.
\end{ex}

\begin{prop}
	Dado um sistema linear $AX=B$, uma solução do sistema homogêneo $X_0$ e uma solução do sistema original $X_1$, então existe uma solução do sistema original $X_2$ tal que $X_1=X_0+X_2$.
\end{prop}

Finalmente, vamos caminhar para uma caracterização geral desse tipo de problema:

\begin{ex}
	Mais uma vez, vamos voltar ao sistema 
	\[\begin{pmatrix}
	1 & 2 & 1 & 1\\1&3&-1&2
	\end{pmatrix}\begin{pmatrix}
	x\\y\\z\\w
	\end{pmatrix}=\begin{pmatrix}
	1\\3
	\end{pmatrix}.\] Já vimos que $S_0=\{(x,y,z,w)\in\R^4\mid x=-5z+w,y=2z-w\}$ - ou seja, uma matrix $X_0$ está em $S_0$ se, e somente se, $X_0$ é da forma
	\[X_0=\begin{pmatrix}
	-5z+w\\2z-w\\z\\w
	\end{pmatrix}.\] Mas podemos reescrever isso como
	\[X_0=\begin{pmatrix}
	-5z\\2z\\z\\0
	\end{pmatrix}+\begin{pmatrix}
	w\\-w\\0\\w
	\end{pmatrix},\] que finalmente se torna
	\[X_0=z\begin{pmatrix}
	-5\\2\\1\\0
	\end{pmatrix}+w\begin{pmatrix}
	1\\-1\\0\\1
	\end{pmatrix}.\]Ou seja, para quaisquer valores de $z,w\in \R$ temos uma solução do sistema homogêneo.
	
	Também já vimos que $S=\{(x,y,z,w)\in \R^4\mid x=-5z+w-3,y=2z-w+2\}$, e podemos fazer a mesma análise: $X_1$ é solução se, e somente se, é da forma
	\[X_1=\begin{pmatrix}
	-5z+w-3\\2z-w+2\\z\\w
	\end{pmatrix}=z\begin{pmatrix}
	-5\\2\\1\\0
	\end{pmatrix}+w\begin{pmatrix}
	1\\-1\\0\\1
	\end{pmatrix}+\begin{pmatrix}
	-3\\2\\0\\0
	\end{pmatrix}=X_0+\begin{pmatrix}
	-3\\2\\0\\0
	\end{pmatrix}.\]
	
	Isso nos mostra que qualquer solução $X_1$ do sistema original é da forma $X_0+\begin{pmatrix}
	-3\\2\\0\\0
	\end{pmatrix}$.
	
	Mas o que é a matriz $\begin{pmatrix}
	-3\\2\\0\\0
	\end{pmatrix}$? Se voltarmos a quando resolvemos o sistema pela primeira vez, vamos ver que $-3$ e $2$ é exatamente como fica a matriz resultante após o escalonamento. Os zeros nas linhas de baixo, então, simbolizam o fato de que o sistema não tem informação suficiente para determinar todas as quatro variáveis.
	
	Por fim, note que se $X_0$ e $X'_0$ são soluções do sistema homogêneo, então, pelo que fizemos acima, 
	\[X_0=z\begin{pmatrix}
		-5\\2\\1\\0
	\end{pmatrix}+w\begin{pmatrix}
		1\\-1\\0\\1
	\end{pmatrix}\]e
	\[X'_0=z'\begin{pmatrix}
	-5\\2\\1\\0
	\end{pmatrix}+w'\begin{pmatrix}
	1\\-1\\0\\1
	\end{pmatrix},\] e como $z,z',w.w'$ são números reais, podemos escrever $w'=\frac{ww'}{w}=w\frac{w'}{w}$ e $z'=\frac{zz'}{z}=z\frac{z'}{z}$, ou seja,
	\[X'_0=z'\begin{pmatrix}
	-5\\2\\1\\0
	\end{pmatrix}+w'\begin{pmatrix}
	1\\-1\\0\\1
	\end{pmatrix}=\frac{z'}{z}\left(z\begin{pmatrix}
	-5\\2\\1\\0
	\end{pmatrix}\right)+\frac{w'}{w}\left(w\begin{pmatrix}
	1\\-1\\0\\1
	\end{pmatrix}\right),\]e vemos que duas soluções do sistema homogêneo diferem apenas por multiplicações de números.
\end{ex}

Para obter um grande resultado conclusivo para esta seção, precisaremos avançar mais no curso.

\section{Matrizes Inversas}

Já vimos que as matrizes possuem o que costumamos chama de \textit{identidade multiplicativa} - ou seja, uma matriz tal que toda matriz vezes ela é a própria matriz.

\begin{df}
	A matriz $I_n\in M_n(\R)$ dada por
	\[I_n:=\begin{pmatrix}
	1&0&\cdots&0\\0&1&\cdots&0\\\vdots&\vdots&\ddots&\vdots\\0&0&\cdots&1
	\end{pmatrix}\] é chamada de \textbf{matriz identidade $n\times n$}.
\end{df}

\begin{ex}
	Dada a matriz $A=\begin{pmatrix}
	a&b\\c&d
	\end{pmatrix}\in M_2(\R)$, é fácil ver que
	\[AI_2=I_2A=A.\]
	
	De fato:
	\[AI_2=\begin{pmatrix}
	a&b\\c&d
	\end{pmatrix}\begin{pmatrix}
	1&0\\0&1
	\end{pmatrix}=\begin{pmatrix}
	a\cdot1+b\cdot 0&a\cdot0+b\cdot 1\\c\cdot1+d\cdot0&c\cdot0+d\cdot0
	\end{pmatrix}=\begin{pmatrix}
	a&b\\c&d
	\end{pmatrix}=A\]e
	\[I_2A=\begin{pmatrix}
	1&0\\0&1
	\end{pmatrix}\begin{pmatrix}
	a&b\\c&d
	\end{pmatrix}=\begin{pmatrix}
	1\cdot a+0\cdot c&1\cdot b+0\cdot d\\0\cdot a+ 1\cdot c&0\cdot b+1\cdot d
	\end{pmatrix}=\begin{pmatrix}
	a&b\\c&d
	\end{pmatrix}=A.\]
\end{ex}

Os números reais também têm essa propriedade: Existe um número real (1) tal que para qualquer número real $a$ temos $a\cdot 1=1\cdot a=a$.

Além disso, os números reais têm outra propriedade: Para qualquer número real $a$ existe um (único) número real $a^{-1}$ tal que $a\cdot a^{-1}=a^{-1}\cdot a=1$. Nós chamamos esse número de \textit{inverso de $a$}.

Surge então a pergunta natural: Será que matrizes reais têm inversas?

\begin{rmk}
	Note que se toda matriz quadrada $A\in M_n(\R)$ possui inversa, então qualquer sistema da forma $AX=B$ pode ser resolvido multiplicando ambos os lados por $A^{-1}$:
	\[AX=B\Leftrightarrow A^{-1}(AX)=A^{-1}B\Leftrightarrow(A^{-1}A)X=A^{-1}B\Leftrightarrow X=A^{-1}B.\] Dito de outra forma, se toda matriz $A$ for inversível, todo sistema $AX=B$ teria solução dada por $X=A^{-1}B$.
\end{rmk}

Contudo, como já vimos acima, \textit{nem todo sistema possui solução}. Isso nos diz imediatamente que \textit{nem toda matriz possui inversa}.

Para ver quais são as matrizes que possuem inversa, vamos precisar de dar uma definição formal para elas.

\begin{df}
	Dada uma matriz $A\in M_n(\R)$, dizemos que \textbf{$A$ é inversível} se existe uma matriz $B\in M_n(\R)$ tal que $AB=BA=I_n$.
\end{df}

\begin{exerc}
	Prove que se $B$ e $C$ são duas inversas para $A$, então $B=C$ (dica: comece assim: ``como $I_n$ vezes qualquer matriz é a própria matriz, $B=BI_n$ e $C=I_nC$. Além disso, como $B$ e $C$ são inversos de $A$, temos que $I_n=AC=BA$'').
\end{exerc}

Vamos agora abordar um método para calcular inversos, quando estes existirem:

\begin{ex}
	Considere a matriz\[A=\begin{pmatrix}
	2&5\\1&3
	\end{pmatrix}.\] Queremos achar uma matriz $B$ tal que $AB=I_2$. Mas isso é um sistema linear! Nós temos uma matriz de dados iniciais ($A$) que multiplicada por uma matriz que queremos determinar ($B$) dá uma matriz de resultados ($I_2$). E nós já sabemos resolver sistemas lineares - via escalonamento! Então, vamos lá:
	
	\[\begin{array}{rl}
	\begin{augmatrix}{cc:cc}
	2&5&1&0\\
	1&3&0&1
	\end{augmatrix}&\rightsquigarrow\begin{augmatrix}{cc:cc}
	1&\frac{5}{2}&\frac{1}{2}&0\\
	1&3&0&1
	\end{augmatrix}\\\\&\rightsquigarrow\begin{augmatrix}{cc:cc}
	1&\frac{5}{2}&\frac{1}{2}&0\\
	0&\frac{1}{2}&-\frac{1}{2}&1
	\end{augmatrix}\\\\&\rightsquigarrow\begin{augmatrix}{cc:cc}
	1&\frac{5}{2}&\frac{1}{2}&0\\
	0&1&-1&2
	\end{augmatrix}\rightsquigarrow\begin{augmatrix}{cc:cc}
	1&0&3&-5\\
	0&1&-1&2
	\end{augmatrix}
	\end{array}\]Então estamos dizendo que a matriz inversa de $A$ é a matriz
	\[A^{-1}=\begin{pmatrix}
	3&-5\\-1&2
	\end{pmatrix}.\]
	
	Vamos testar:
	\[AA^{-1}=\begin{pmatrix}
	2&5\\1&3
	\end{pmatrix}\begin{pmatrix}
	3&-5\\-1&2
	\end{pmatrix}=\begin{pmatrix}
	2\cdot3+5\cdot(-1)&2\cdot(-5)+5\cdot2\\
	1\cdot3+3\cdot(-1)&1\cdot(-5)+3\cdot2
	\end{pmatrix}=\begin{pmatrix}
	6-5 & -10+10\\3-3&-5+6
	\end{pmatrix}=\begin{pmatrix}
	1&0\\0&1
	\end{pmatrix}\]e
	\[A^{-1}A=\begin{pmatrix}
	3&-5\\-1&2
	\end{pmatrix}\begin{pmatrix}
	2&5\\1&3
	\end{pmatrix}=\begin{pmatrix}
	3\cdot2+(-5)\cdot1&3\cdot5+(-5)\cdot3\\
	(-1)\cdot2+2\cdot1&(-1)\cdot5+2\cdot3
	\end{pmatrix}=\begin{pmatrix}
	6-5&-5+5\\
	-2+2&-5+6
	\end{pmatrix}=\begin{pmatrix}
	1&0\\0&1
	\end{pmatrix}\]
\end{ex}

\begin{prop}
	Seja $\begin{pmatrix}
	a&b\\c&d
	\end{pmatrix}\in M_2(\R)$ uma matriz. Então a inversa, se existir, é da forma
	\[\frac{1}{ad-bc}\begin{pmatrix}
	d&-b\\
	-c&a
	\end{pmatrix}.\]
\end{prop}

Essa é a primeira vez que vemos o determinante de uma matriz aparecendo. Mais pra frente vamos definir o determinante de outra maneira e ver para que ele serve.

Outro resultado que vamos usar agora, mesmo que não sejamos capazes de mostrar ainda é o seguinte:

\begin{lemma}
	Seja $A\in M_n(\R)$ matriz quadrada. Se existe matriz $B\in M_n(\R)$ tal que $AB=I_n$, então $B$ é inversa de $A$. Similarmente, se existe matriz $C\in M_n(\R)$ tal que $CA=I_n$, então $C$ é inversa de $A$.
\end{lemma}

Em outras palavras, para matrizes, basta checar se o produto em uma ordem dá a identidade, que isso é suficiente para concluir que o produto na outra ordem também dará.

Agora sim, vamos ver um resultado que conseguimos provar:

\begin{lemma}
	Uma matriz $A\in M_n(\R)$ é inversível se, e somente se, $A$ pode ser escalonada em $I_n$.
\end{lemma}
\begin{proof}
	Por um lado, é óbvio que a matriz identidade pode ser escalonada em si mesma (fazendo nada) e que a matriz identidade é inversível: $I_nI_n=I_n$ - de fato, ela é seu próprio inverso. Além disso, já vimos que se $A'$ pode ser obtida de $A$ via escalonamento, então o sistema $AX=B$ tem as mesmas soluções do sistema $A'X=B'$, em que $B'$ é obtida de $B$ pelo mesmo escalonamento que leva $A$ em $A'$. 
	
	Então, se $I_n$ pode ser obtida de $A$ por escalonamento, o sistema $AX=I_n$ tem as mesmas soluções do sistema $I_nX=C$, em que $C$ é obtida de $I_n$ pelo mesmo escalonamento que leva $A$ em $I_n$. Mas $I_nX=C$ nos diz que $C=X$ e, logo, $AC=I_n$. O lema acima agora nos garante que $CA=I_n$ e portanto $A$ é inversível.
	
	\bigskip
	Analogamente, suponha que $A$ é inversível - ou seja, o sistema $AX=I_n$ tem solução $B$ - em outras palavras, $X=B$. Mas podemos re-escrever isso como $I_nX=B$ e ver que esse sistema tem a mesma solução de $AX=I_n$, donde podemos concluir que $I_n$ pode ser obtido de $A$ por escalonamentos.
\end{proof}

Finalmente, vamos encerrar essa seção retomando as matrizes de escalonamento.

Como dissemos anteriormente, nem toda soma de matrizes $e^n_{i,j}$ é um escalonamento.

\begin{prop}
	Toda matriz invertível é um escalonamento.
\end{prop}

\begin{proof}
	Dado um sistema $AX=B$, se $E$ é inversível com inversa $E^{-1}$, tome $Z$ solução de $(EA)X=EB$. Vamos mostrar que $Z$ também é solução de $AX=B$ - e portanto, $E$ é escalonamento.
	
	De fato:
	
	\[AZ=I_n(AZ)=(E^{-1}E)(AZ)=E^{-1}((EA)Z)=E^{-1}(EB)=(E^{-1}E)B=I_nB=B,\]logo $Z$ é solução de $AX=B$ e, portanto, $E$ é um escalonamento.
\end{proof}

Gostaríamos de mostrar mais - gostaríamos de mostrar que todo escalonamento é invertível, mas não temos ferramentas para isso ainda. Contudo, para matrizes $2\times 2$ é fácil:

\begin{ex}
	Os possíveis escalonamentos são combinações de 
	\begin{itemize}
		\item Troca de linhas;
		\item Multiplicação de uma linha por um número;
		\item Soma de uma linha a outra linha.
	\end{itemize}

	Vamos exibir as matrizes que realizam cada operação:
	
	\begin{itemize}
		\item A matriz que troca as duas linhas de uma matriz $2\times 2$ é dada por $\begin{pmatrix}
		0&1\\1&0
		\end{pmatrix}$, de fato:
		\[\begin{pmatrix}
		0&1\\1&0
		\end{pmatrix}\begin{pmatrix}
		a&b\\c&d
		\end{pmatrix}=\begin{pmatrix}
		0\cdot a+1\cdot c&0\cdot b+1\cdot d\\
		1\cdot a+0\cdot c&1\cdot b+0\cdot d
		\end{pmatrix}=\begin{pmatrix}
		c&d\\a&b
		\end{pmatrix}.\]
		
		\item A matriz que multiplica a linha 1 por $\lambda\in \R$ é $\begin{pmatrix}
		\lambda&0\\0&1
		\end{pmatrix}$ e a matriz que multiplica a linha 2 por $\mu\in \R$ é $\begin{pmatrix}
		1&0\\0&\mu
		\end{pmatrix}$. De fato, 
		\[\begin{pmatrix}
			\lambda&0\\
			0&\mu
		\end{pmatrix}\begin{pmatrix}
		a&b\\c&d
		\end{pmatrix}=\begin{pmatrix}
		\lambda\cdot a+0\cdot c&\lambda\cdot b+0\cdot d\\
		0\cdot a+\mu\cdot c&0\cdot b+\mu\cdot d
		\end{pmatrix}=\begin{pmatrix}
		\lambda a&\lambda b\\\mu c&\mu d
		\end{pmatrix}.\]
		\item A matrix que soma a linha 1 na linha 2 é $\begin{pmatrix}
		1&0\\1&1
		\end{pmatrix}$ e a matriz que soma a linha 2 na linha 1 é $\begin{pmatrix}
		1&1\\0&1
		\end{pmatrix}$. De fato,
		\[\begin{pmatrix}
		1&0\\1&1
		\end{pmatrix}\begin{pmatrix}
		a&b\\c&d
		\end{pmatrix}=\begin{pmatrix}
		1\cdot a+0\cdot c&1\cdot b+0\cdot d\\
		1\cdot a+1\cdot c&1\cdot b+1\cdot d
		\end{pmatrix}=\begin{pmatrix}
		a & b\\
		a+c&b+d
		\end{pmatrix}\]e
		\[\begin{pmatrix}
		1&1\\0&1
		\end{pmatrix}\begin{pmatrix}
		a&b\\c&d
		\end{pmatrix}=\begin{pmatrix}
		1\cdot a+1\cdot c&1\cdot b+1\cdot d\\
		0\cdot a+1\cdot c&0\cdot b+1\cdot d
		\end{pmatrix}=\begin{pmatrix}
		a+c & b+d\\
		c&d
		\end{pmatrix}.\]
	\end{itemize}

	Mas todas essas matrizes são inversíveis: 
	\begin{itemize}
		\item O inverso da matriz $\begin{pmatrix}
		0&1\\1&0
		\end{pmatrix}$ é ela mesma, já que trocar as duas linhas duas vezes é a mesma coisa de não fazer nada.
		\item Os inversos das matrizes $\begin{pmatrix}
		\lambda &0\\0&1
		\end{pmatrix}$ e $\begin{pmatrix}
		1&0\\0&\mu
		\end{pmatrix}$ são as matrizes $\begin{pmatrix}
		\lambda^{-1}&0\\0&1
		\end{pmatrix}$ e $\begin{pmatrix}
		1&0\\0&\mu^{-1}
		\end{pmatrix}$, respectivamente, já que o inverso de ``multiplicar uma linha por $x$'' é ``dividir uma linha por $x$''.
		\item Os inversos das matrizes $\begin{pmatrix}
		1&0\\1&1
		\end{pmatrix}$ e $\begin{pmatrix}
		1&1\\0&1
		\end{pmatrix}$ são, respectivamente, as matrizes $\begin{pmatrix}
		1&0\\-1&1
		\end{pmatrix}$ e $\begin{pmatrix}
		1&-1\\0&1
		\end{pmatrix}$, já que o inverso de ``somar a linha $i$ na linha $j$'' é ``subtrair a linha $i$ da linha $j$''.
	\end{itemize}

	Finalmente, note que se $A$ e $B$ são inversíveis, então o produto $AB$ também é, pois $(AB)(B^{-1}A^{-1})=A((BB^{-1})A^{-1})=A(I_nA^{-1})=AA^{-1}=I_n$. Assim, como qualquer escalonamento é produto das matrizes acima, segue que qualquer escalonamento é invertível, já que cada uma delas é.
\end{ex}

\begin{exerc}
	Prove que os inversos que apresentamos acima são, de fato, inversos.
\end{exerc}
\chapter{Vetores em $\R^2$}

\section{Definições e Propriedades Básicas}

Em matemática, dados dois conjuntos $A$ e $B$, nós denotamos por $A\times B$ o conjunto de todos os pares ordenados de elementos de $A$ e $B$.

\begin{ex}
	Se $A=\{a,b,c,d\}$ e $B=\{1,2,3\}$, então \[A\times B=\{(a,1),(a,2),(a,3),(b,1),(b,2),(b,3),(c,1),(c,2),(c,3),(d,1),(d,2),(d,3)\}.\]
	
	Se $C=\{\ltimes,\rtimes\}$, então
	\[C\times B=\{(\ltimes,1),(\ltimes,2),(\ltimes 3),(\rtimes,1),(\rtimes,2),(\rtimes 3)\}.\]
	
	Se $D=\{$calça, camiseta$\}$ e $E=\{$azul, verde, amarela$\}$, então
	\[D\times E=\{\mbox{calça azul, calça verde, calça amarela, camiseta azul, camiseta verde, camiseta amarela}\}.\]
\end{ex}

Assim, o conjunto de todos os pares ordenados de números reais é o conjunto $\R\times \R$. Como o símbolo $\times$ remete a um produto, nós chamamos esse conjunto de $\R^2$.

Vamos representar os elementos de $\R^2$ como pontos no plano da seguinte maneira: Correspondemos o elemento $(a,b)\in \R^2$ ao ponto $P$ dado pelas coordenadas $a$ e $b$, como abaixo:

	\[\definecolor{uuuuuu}{rgb}{0.26666666666666666,0.26666666666666666,0.26666666666666666}
	\begin{tikzpicture}[line cap=round,line join=round,>=triangle 45,x=0.75cm,y=0.75cm]
	\begin{axis}[
	x=0.75cm,y=0.75cm,
	axis lines=middle,
	xlabel=$\R$,
	ylabel=$\R$,
	xmin=-3.22,
	xmax=8.520000000000005,
	ymin=-2.1400000000000032,
	ymax=5.76,
	yticklabels={},	
	xticklabels={}]
	\clip(-3.22,-2.14) rectangle (8.52,5.76);
	\draw [dash pattern=on 3pt off 3pt] (0.,2.)-- (4.,2.);
	\draw [dash pattern=on 3pt off 3pt] (4.,0.)-- (4.,2.);
	\begin{scriptsize}
	\draw [fill=black] (4.,2.) circle (2.5pt);
	\draw[color=black] (4.14,2.37) node {$P$};
	\draw [color=uuuuuu] (0.,2.)-- ++(-2.0pt,0 pt) -- ++(4.0pt,0 pt) ++(-2.0pt,-2.0pt) -- ++(0 pt,4.0pt);
	\draw[color=uuuuuu] (-0.32,2) node {$b$};
	\draw [color=uuuuuu] (4.,0.)-- ++(-2.0pt,0 pt) -- ++(4.0pt,0 pt) ++(-2.0pt,-2.0pt) -- ++(0 pt,4.0pt);
	\draw[color=uuuuuu] (4.,-0.4) node {$a$};
	\end{scriptsize}
	\end{axis}
	\end{tikzpicture}\]
	
Claramente essa correspondência é biunívoca - pois dado um ponto no plano, ele é unicamente determinado por duas coordenadas.

Assim estabelecemos um paralelo entre \textbf{pontos no plano} e \textbf{elementos de $\R^2$}. Por esse motivo, elementos de $\R^2$ são muitas vezes chamados de \textit{pontos} de $\R^2$.

Além disso, podemos fazer outra correspondência:
\[\begin{tikzpicture}[line cap=round,line join=round,>=triangle 45,x=1.0cm,y=1.0cm]
\begin{axis}[
x=1.0cm,y=1.0cm,
axis lines=middle,
xmin=-1,
xmax=6,
ymin=-1,
ymax=4,
xticklabels={},yticklabels={},xlabel=$\R$,ylabel=$\R$,]
\clip(-1.,-1.) rectangle (6.,4.02);
\draw [->] (0.,0.) -- (4.,2.);
\begin{scriptsize}
\draw [fill=black] (4.,2.) circle (2.5pt);
\draw[color=black] (4.14,2.37) node {$P$};
\draw[color=black] (1.88,1.29) node {$v$};
\end{scriptsize}
\end{axis}
\end{tikzpicture}\]

Vamos chamar de \textbf{vetor em $\R^2$} uma seta partindo da origem e terminando em qualquer ponto do plano. A figura acima estabelece uma correspondência biunívoca entre vetores no plano e pontos no plano: Para cada vetor $v$ existe um único ponto final $P$. Similarmente, para cada ponto $P$ existe um único vetor $v$ que termina em $P$.

Juntando tudo isso, nós vemos que \textbf{o conjunto de pontos, pares ordenados e vetores em $\R^2$ são ``a mesma coisa''}. Com isso em mente, daqui para frente vamos usar todos esses significados, usando sempre o significado mais conveniente naquele momento.

\begin{ex}
	\definecolor{ududff}{rgb}{0.30196078431372547,0.30196078431372547,1.}
	\definecolor{xdxdff}{rgb}{0.49019607843137253,0.49019607843137253,1.}
	\[\begin{tikzpicture}[line cap=round,line join=round,>=triangle 45,x=1.0cm,y=1.0cm]
	\begin{axis}[
	x=1.0cm,y=1.0cm,
	axis lines=middle,
	xmin=-1,
	xmax=3,
	ymin=-1,
	ymax=5,
	xtick={1,2},
	ytick={1,2,3,4},
	xticklabels={},
	yticklabels={},
	xlabel=$\R$,
	ylabel=$\R$]
	\clip(-2.02,-2.02) rectangle (3.98,6.04);
	\draw [->] (0.,0.) -- (2.,4.);
	\draw [dash pattern=on 3pt off 3pt] (2.,4.)-- (2.,0.);
	\draw [dash pattern=on 3pt off 3pt] (2.,4.)-- (0.,4.);
	\begin{scriptsize}
	\draw [fill=black] (2.,4.) circle (2.5pt);
	\draw[color=black] (2.14,4.37) node {$P$};
	\draw[color=black] (0.84,2.25) node {$v$};
	\draw  (2.,0.)-- ++(-2.5pt,0 pt) -- ++(5.0pt,0 pt) ++(-2.5pt,-2.5pt) -- ++(0 pt,5.0pt);
	\draw (2,-0.3) node {$2$};
	\draw  (0.,4.)-- ++(-2.5pt,0 pt) -- ++(5.0pt,0 pt) ++(-2.5pt,-2.5pt) -- ++(0 pt,5.0pt);
	\draw (-0.3,4) node {$4$};
	\end{scriptsize}
	\end{axis}
	\end{tikzpicture}\]
	
	A figura acima em $\R^2$ representa simultaneamente os três conceitos: O vetor $v$ tem como ponto final o ponto $P$ cujas coordenadas são $(2,4)$.
\end{ex}

\subsection{Somas e produtos por números}

Dados dois pares ordenados $(a,b)$ e $(c,d)$ em $\R^2$, nós podemos notar que como $a,b,c,d$ são números reais, nós podemos somar $a+c$ e $b+d$, e ambos serão números reais. Assim, faria sentido definir uma operação de soma em $\R^2$ dada por
\[(a,b)+(c,d):=(a+c,b+d).\] Contudo, pelo que já vimos acima, pares, pontos e vetores são ``a mesma coisa''. Será que essa soma tem alguma interpretação via pontos e vetores?

\begin{ex}
	\[\begin{tikzpicture}[line cap=round,line join=round,>=triangle 45,x=1.0cm,y=1.0cm]
	\begin{axis}[
	x=1.0cm,y=1.0cm,
	axis lines=middle,
	xmin=-1,
	xmax=9,
	ymin=-1,
	ymax=6,
	xticklabels={},
	yticklabels={},
	xtick={0,...,8.0},
	ytick={0,...,5.0},]
	\clip(-2.98,-2.02) rectangle (9.04,6.98);
	\draw [->,] (0.,0.) -- (2.,4.) node[sloped,pos=0.5,above] {$v$};
	\draw [,dash pattern=on 4pt off 4pt] (2.,4.)-- (2.,0.);
	\draw [,dash pattern=on 4pt off 4pt] (2.,4.)-- (0.,4.);
	\draw [->,] (0.,0.) -- (6.,1.) node[sloped,pos=0.5,below] {$u$};
	\draw [,dash pattern=on 4pt off 4pt] (6.,1.)-- (6.,0.);
	\draw [,dash pattern=on 4pt off 4pt] (6.,1.)-- (0.,1.);
	\begin{scriptsize}
	\draw [fill=black] (2.,4.) circle (2.5pt);
	\draw[color=black] (2.14,4.37) node {$P$};
	\draw [color=black] (2.,0.)-- ++(-2.5pt,0 pt) -- ++(5.0pt,0 pt) ++(-2.5pt,-2.5pt) -- ++(0 pt,5.0pt);
	\draw[color=black] (2,-0.3) node {$2$};
	\draw [color=black] (0.,4.)-- ++(-2.5pt,0 pt) -- ++(5.0pt,0 pt) ++(-2.5pt,-2.5pt) -- ++(0 pt,5.0pt);
	\draw[color=black] (-0.3,4) node {$4$};
	\draw [fill=black] (6.,1.) circle (2.5pt);
	\draw[color=black] (6.14,1.37) node {$Q$};
	\draw [color=black] (6.,0.)-- ++(-2.5pt,0 pt) -- ++(5.0pt,0 pt) ++(-2.5pt,-2.5pt) -- ++(0 pt,5.0pt);
	\draw[color=black] (6,-0.37) node {$6$};
	\draw [color=black] (0.,1.)-- ++(-2.5pt,0 pt) -- ++(5.0pt,0 pt) ++(-2.5pt,-2.5pt) -- ++(0 pt,5.0pt);
	\draw[color=black] (-0.3,1) node {$1$};
	\end{scriptsize}
	\end{axis}
	\end{tikzpicture}\]Na figura acima temos os pares $(2,4)$ e $(6,1)$ representados pelos pontos $P$ e $Q$ e pelos vetores $v$ e $u$, respectivamente.
	
	\[\definecolor{qqqqff}{rgb}{0.,0.,1.}
	\begin{tikzpicture}[line cap=round,line join=round,>=triangle 45,x=1.0cm,y=1.0cm]
	\begin{axis}[
	x=1.0cm,y=1.0cm,
	axis lines=middle,
	xmin=-1,
	xmax=9,
	ymin=-1,
	ymax=6,
	xticklabels={},
	yticklabels={},
	xtick={0,...,8},
	ytick={0,...,5},]
	\clip(-1.,-1.08) rectangle (9.02,6.);
	\draw [->] (0.,0.) -- (2.,4.) node[sloped,pos=0.5,above] {$v$};
	\draw [,dash pattern=on 3pt off 3pt] (2.,4.)-- (2.,0.);
	\draw [,dash pattern=on 3pt off 3pt] (2.,4.)-- (0.,4.);
	\draw [->] (0.,0.) -- (6.,1.) node[sloped,pos=0.5,below] {$u$};
	\draw [,dash pattern=on 3pt off 3pt] (6.,1.)-- (6.,0.);
	\draw [,dash pattern=on 3pt off 3pt] (6.,1.)-- (0.,1.);
	\draw [->,dotted] (6.,1.) -- (8.,5.) node[sloped,pos=0.5,below] {$v$};
	\draw [->,dotted] (2.,4.) -- (8.,5.) node[sloped,pos=0.5,above] {$u$};
	\draw [->,dash pattern=on 1pt off 1pt on 3pt off 4pt] (0.,0.) -- (8.,5.) node[sloped,pos = 0.5,above] {$v+u$};
	\draw [,dash pattern=on 3pt off 3pt] (8.,5.)-- (8.,0.);
	\draw [,dash pattern=on 3pt off 3pt] (8.,5.)-- (0.,5.);
	\begin{scriptsize}
	\draw [fill=black] (2.,4.) circle (2.5pt);
	\draw[color=black] (2.14,4.37) node {$P$};
	\draw [color=black] (2.,0.)-- ++(-2.5pt,0 pt) -- ++(5.0pt,0 pt) ++(-2.5pt,-2.5pt) -- ++(0 pt,5.0pt);
	\draw[color=black] (2,-0.3) node {$2$};
	\draw [color=black] (0.,4.)-- ++(-2.5pt,0 pt) -- ++(5.0pt,0 pt) ++(-2.5pt,-2.5pt) -- ++(0 pt,5.0pt);
	\draw[color=black] (-0.3,4) node {$4$};
	\draw [fill=black] (6.,1.) circle (2.5pt);
	\draw[color=black] (6.3,1) node {$Q$};
	\draw [color=black] (6.,0.)-- ++(-2.5pt,0 pt) -- ++(5.0pt,0 pt) ++(-2.5pt,-2.5pt) -- ++(0 pt,5.0pt);
	\draw[color=black] (6,-0.37) node {$6$};
	\draw [color=black] (0.,1.)-- ++(-2.5pt,0 pt) -- ++(5.0pt,0 pt) ++(-2.5pt,-2.5pt) -- ++(0 pt,5.0pt);
	\draw[color=black] (-0.3,1) node {$1$};
	\draw [fill=qqqqff] (8.,5.) circle (2.5pt);
	\draw[color=qqqqff] (8.14,5.37) node {$P+Q$};
	\draw [color=black] (8.,0.)-- ++(-2.5pt,0 pt) -- ++(5.0pt,0 pt) ++(-2.5pt,-2.5pt) -- ++(0 pt,5.0pt);
	\draw[color=black] (8,-0.37) node {$8$};
	\draw [color=black] (0.,5.)-- ++(-2.5pt,0 pt) -- ++(5.0pt,0 pt) ++(-2.5pt,-2.5pt) -- ++(0 pt,5.0pt);
	\draw[color=black] (-0.3,5) node {$5$};
	\end{scriptsize}
	\end{axis}
	\end{tikzpicture}\]Então, somando os pares $(2,4)$ e $(6,1)$ obtemos o par $(2+6,4+1)=(8,5)$. Geometricamente, temos a figura acima: Dados os vetores $v$ e $u$, com coordenadas $(2,4)$ e $(6,1)$, respectivamente, o vetor $v+u$ que é obtido simplesmente pela concatenação do vetor $u$ ao vetor $v$.
\end{ex}

\begin{df}
	Dados dois vetores $u,v\in \R^2$ com coordenadas $u=(a,b)$ e $v=(c,d)$, definimos a \textbf{soma de $u$ com $v$} como sendo o vetor $u+v$ dado por
	\[u+v:=(a+c,b+d).\]
\end{df}

\begin{exerc}
	Mostre que essa soma satisfaz as seguintes propriedades:
	\begin{itemize}
		\item (Comutatividade) Para quaisquer vetores $u,v\in R^2$, $u+v=v+u$;
		\item (Associatividade) Para quaisquer três vetores $u,v,w\in \R^2$, $u+(v+w)=(u+v)+w$;
		\item (Existência e unicidade de elemento neutro) Existe um (único) vetor $\overrightarrow{0}\in \R^2$ tal que $v+\overrightarrow{0}=v$ para qualquer $v\in \R^2$;
		\item (Existência e unicidade de inversos) Para qualquer vetor $v\in\R^2$ existe um (único) vetor $-v\in\R^2$ tal que $v+(-v)=\overrightarrow{0}$.
	\end{itemize}
\end{exerc}

Por outro lado, é intuitivamente óbvio que se $(a,b)\in\R^2$ e $\lambda\in \R$, certamente o par $(\lambda a,\lambda b)\in \R^2$. Então parece natural definir
\[\lambda(a,b):=(\lambda a,\lambda b).\] Contudo, será que temos uma boa interpretação geométrica pra isso, como para a soma?

\begin{ex}
	Fixe o número $\lambda=\dfrac{1}{2}\in\R$ e o vetor $v=(4,2)\in\R^2$.
	\definecolor{ududff}{rgb}{0.30196078431372547,0.30196078431372547,1.}
	\definecolor{xdxdff}{rgb}{0.49019607843137253,0.49019607843137253,1.}
	\[\begin{tikzpicture}[line cap=round,line join=round,>=triangle 45,x=1.0cm,y=1.0cm]
	\begin{axis}[
	x=1.0cm,y=1.0cm,
	axis lines=middle,
	xmin=-1,
	xmax=5,
	ymin=-1,
	ymax=3,
	xtick={1,...,4},
	ytick={1,2},]
	\clip(-1,-1) rectangle (6,4);
	\draw [->] (0.,0.) -- (4,2) node[sloped,pos=0.5,above] {$v$};
	\draw [dash pattern=on 3pt off 3pt] (4,2)-- (0,2);
	\draw [dash pattern=on 3pt off 3pt] (4,2)-- (4,0);	
	\begin{scriptsize}
	\draw [fill=black] (4,2) circle (2.5pt);
	\draw[color=black] (4,2.37) node {$P$};
	\end{scriptsize}
	\end{axis}
	\end{tikzpicture}\]O que seria o vetor $u=(2,1)=\left(\dfrac{1}{2}\cdot 4,\dfrac{1}{2}\cdot2\right)=\dfrac{1}{2}\cdot v$?
	
	\[\begin{tikzpicture}[line cap=round,line join=round,>=triangle 45,x=1.0cm,y=1.0cm]
	\begin{axis}[
	x=1.0cm,y=1.0cm,
	axis lines=middle,
	xmin=-1,
	xmax=5,
	ymin=-1,
	ymax=3,
	xtick={1,...,4},
	ytick={1,2},]
	\clip(-1,-1) rectangle (6,4);
	\draw [->] (0.,0.) -- (4,2) node[sloped,pos=0.7,above] {$v$};
	\draw [->,blue] (0.,0.) -- (2,1) node[sloped,pos=0.7,below] {$\frac{1}{2}v$};
	\draw [dash pattern=on 3pt off 3pt] (4,2)-- (0,2);
	\draw [dash pattern=on 3pt off 3pt] (4,2)-- (4,0);
	\draw [dash pattern=on 3pt off 3pt] (2,1)-- (0,1);
	\draw [dash pattern=on 3pt off 3pt] (2,1)-- (2,0);	
	\begin{scriptsize}
	\draw [fill=black] (4,2) circle (2.5pt);
	\draw[color=black] (4,2.37) node {$P$};
	\draw [fill=blue] (2,1) circle (2.5pt);
	\draw[color=black] (2,1.3) node[blue] {$\frac{1}{2}P$};
	\end{scriptsize}
	\end{axis}
	\end{tikzpicture}\]
	
	Então o vetor $\dfrac{1}{2}\cdot v$ é o vetor \textbf{na mesma direção de $v$}, mas \textbf{com tamanho igual a $\dfrac{1}{2}$ vezes o tamanho de $v$}.
\end{ex}

\begin{exerc}	
	O que seria, então multiplicar por um número negativo, por exemplo $\lambda=-1\in\R$? Faça um desenho do que seria uma interpretação geométrica para $(-1)\cdot(4,2)$.
\end{exerc}

\begin{exerc}
	Mostre que a multiplicação de vetores por números satisfaz:
	\begin{itemize}
		\item (Comutatividade) $(\lambda\mu)v=(\mu\lambda)v$ para quaisquer números $\lambda,\mu\in \R$ e vetor $v\in \R^2$;
		\item (Associatividade) $\lambda(\mu v)=(\lambda\mu)v$ e para quaisquer números $\lambda,\mu\in \R$ e vetor $v\in \R^2$;
		\item (Existência e unicidade de elemento neutro) Existe um (único) numero real $u\in \R$ tal que $u v=v$ para qualquer vetor $v\in\R^2$;
		\item (Distributividades) Para quaisquer $\lambda,\mu \in \R$ e $v,u\in\R^2$, temos que $\lambda(v+u)=\lambda v=\lambda u$ e $(\lambda+\mu)v=\lambda v+\mu v$.
	\end{itemize}
\end{exerc}

\section{Vetores e Matrizes}

Antes de estudar mais a fundo os vetores, contudo, precisamos estabelecer um paralelo entre o estudo de matrizes e o estudo de vetores:

Considere a função $[-]:\R^2\to M_{2\times 1}(\R)$ dada por
\[[(x,y)]:=\begin{pmatrix}
x\\y
\end{pmatrix}\in M_{2\times 1}(\R).\] Essa função ``não faz nada'': Ela só nos permite enxergar vetores como matrizes coluna. Similarmente, podemos considerar a função $V:M_{2\times 1}(\R)\to \R^2$ dada por 
\[V\begin{pmatrix}
x\\y
\end{pmatrix}:=(x,y)\in \R^2.\] Essa função também ``não faz nada'': Ela só nos permite enxergar matrizes coluna como vetores. Contudo, note que:

\[V([(x,y)])=V\begin{pmatrix}
x\\y
\end{pmatrix}=(x,y)\]
\[\left[V\begin{pmatrix}
x'\\y'
\end{pmatrix}\right]=[(x',y')]=\begin{pmatrix}
x'\\y'
\end{pmatrix}\]ou seja, \textit{essas funções são inversas}! Isso nos diz que os conjuntos $\R^2$ e $M_{2\times 1}(\R)$ são ``a mesma coisa'', apenas vistos com outros olhos. Por isso, vamos adicionar mais um significado à palavra vetor: Não só pontos no plano, ou pares ordenados em $\R^2$ ou setas partindo da origem, mas também matrizes coluna com duas entradas.

\subsection{Funções lineares}

Nesta seção, então, vamos interpretar vários dos resultados do capítulo anterior, mas agora com o auxílio dos diversos significados de vetor.

Por exemplo, dada uma matriz quadrada $A=\begin{pmatrix}
a&b\\c&d
\end{pmatrix}$, a multiplicação $AX$, em que $X$ é alguma matriz coluna com duas entradas, é da seguinte forma:
\[\begin{pmatrix}
a&b\\c&d
\end{pmatrix}\begin{pmatrix}
x\\y
\end{pmatrix}=\begin{pmatrix}
ax+by\\cx+dy
\end{pmatrix}.\] Então para cada valor de $x$ e $y$, o resultado do produto $AX$ é diferente -  ora, isso nos permite definir:

\begin{df}
	Para cada matriz $A\in M_2(\R)$, definimos $T_A:\R^2\to \R^2$ a função dada por
	\[T_A(x,y):=V(A[(x,y)]).\]
\end{df}

Ou seja, para calcular $T_A(x,y)$, a gente escreve $(x,y)$ como uma matriz coluna $\begin{pmatrix}
x\\y
\end{pmatrix}$, multiplica ele por $A$ e escreve o resultado como um vetor.

\begin{ex}
	Dada a matriz $A=\begin{pmatrix}
	2 & 5\\
	1 & 3
	\end{pmatrix}$, a função $T_A$ é tal que
	\[T_A(x,y)=V(A[(x,y)])=V\left(\begin{pmatrix}
	2 & 5\\
	1 & 3
	\end{pmatrix}\begin{pmatrix}
	x\\y
	\end{pmatrix}\right)=V\left(\begin{pmatrix}
	2x+5y\\x+3y
	\end{pmatrix}\right)=(2x+5y,x+3y)\]em outras palavras, a função $T_A$ é tal que $T_A(x,y)$ é simplesmente
	\[\begin{pmatrix}
	2 & 5\\
	1 & 3
	\end{pmatrix}\begin{pmatrix}
	x\\y
	\end{pmatrix}=\begin{pmatrix}
	2x+5y\\x+3y
	\end{pmatrix}\]escrito em forma de vetor, ou seja $(2x+5y,x+3y)$.
	
	Similarmente,
	\[T_{I_2}(x,y)=V(I_2[(x,y)])=V\left(\begin{pmatrix}
	1&0\\0&1
	\end{pmatrix}\begin{pmatrix}
	x\\y
	\end{pmatrix}\right)=V\left(\begin{pmatrix}
	x\\y
	\end{pmatrix}\right)=(x,y),\]e, analogamente a como fizemos acima, ou seja, $T_{I_2}(x,y)$ é simplesmente
	\[\begin{pmatrix}
	1&0\\0&1
	\end{pmatrix}\begin{pmatrix}
	x\\y
	\end{pmatrix}=\begin{pmatrix}
	x\\y
	\end{pmatrix}\] escrito em forma de vetor, ou seja, $(x,y)$ - em outras palavras, a função $T_{I_2}$ é a função identidade que leva todo vetor nele mesmo.
\end{ex}

Como essas funções são dadas por matrizes, será que elas têm as propriedades de matrizes?

\begin{prop}
	Dados uma matriz $A\in M_2(\R)$, qualquer número $\lambda \in \R$ e dois vetores $v,u\in \R^2$, temos:
	\[T_A(\lambda v)=\lambda T_A(v)\]
	\[T_A(v+u)=T_A(v)+T_A(u).\]
\end{prop}

\begin{exerc}
	Prova a proposição acima (dica: calcule explicitamente $T_A(\lambda v)$ e $\lambda T_A(v)$ e veja que são a mesma coisa. Idem para as somas.).
\end{exerc}

Essas funções são bastante interessantes. Elas nos dão um ``dicionário'' que nos permite ver o mundo dos vetores como se fosse o mundo das matrizes, e vice-e-versa. Seria, então, interessante se a gente conseguisse um método de determinar quando uma função $f:\R^2\to \R^2$ é dessa forma. 

\begin{ex}
	A função $f:\R^2\to \R^2$ dada por $f(x,y)=(x,x^2)$ \textbf{não} é dada por uma matriz. Podemos ver isso usando a proposição acima: Se $f$ fosse igual a $T_A$ para alguma matriz $A\in M_2(\R)$, $f(v+u)$ seria igual a $f(v)+f(u)$. Mas
	\[f(0,0)=(0,0^2)=(0,0),\]enquanto
	\[f(1,0)+f(-1,0)=(1,1^2)+(-1,1^2)=(0,2).\] Agora, note que $(0,0)=(1,0)+(-1,0)$. Então nós acabamos de mostrar que $f((1,0)+(-1,0))\neq f(1,0)+f(-1,0)$, ou seja, $f$ \textbf{não pode ser dada por $T_A$}.
\end{ex}

\begin{prop}
	Uma função $f:\R^2 \to \R^2$ é dada por uma matriz se, e somente se, existem números reais $a,b,c,d\in \R$ tais que $f(x,y)=(ax+by,cx+dy)$.
\end{prop}
\begin{proof}
	Suponha que $f$ é dada por uma matriz, ou seja, $f=T_A$ para alguma matriz $A=\begin{pmatrix}
	a&b\\c&d
	\end{pmatrix}\in M_2(\R)$. Então, para qualquer vetor $(x,y)\in \R^2$ temos:
	\[f(x,y)=T_A(x,y)=V(A([x,y]))=V\left(\begin{pmatrix}
	a&b\\c&d
	\end{pmatrix}\begin{pmatrix}
	x\\y
	\end{pmatrix}\right)=V\left(\begin{pmatrix}
	ax+by\\cx+dy
	\end{pmatrix}\right)=(ax+by,cx+dy),\] como queríamos mostrar.
	
	\bigskip
	Por outro lado, suponha que $f(x,y)=(ax+by,cx+dy)$. Afirmamos que a matriz $A=\begin{pmatrix}
	a&b\\c&d
	\end{pmatrix}$ é tal que $f=T_A$. De fato, 
	\[T_A(x,y)=(ax+by,cx+dy)=f(x,y),\]o que encerra a prova.
\end{proof}

Finalmente, agora vamos resolver um problema que será central nesse curso: Será que toda função $f:\R^2\to \R^2$ que preserva somas e produtos por números é dada por matrizes?

\begin{theorem}
	Seja $f:\R^2\to\R^2$ uma função. Então $f(v+\lambda u)=f(v)+\lambda f(u)$ para todo $\lambda\in \R$ e todos $v,u\in \R^2$ se, e somente se, $f=T_A$ para alguma matriz $A\in M_2(\R)$.
\end{theorem}
\begin{proof}
	Já mostramos que se $f=T_A$, então $f(v+\lambda u)=f(v)+\lambda f(u)$.
	
	Suponha, então, que $f(v+\lambda u)=f(v)+\lambda f(u)$. Vamos mostrar que $f=T_A$ para alguma matriz $A$.
	
	Calculando $f(x,y)$ temos:
	\[\begin{array}{rll}
		f(x,y)&=f(x,0)+f(0,y)&\mbox{(já que $f$ preserva somas)}\\
		&=xf(1,0)+yf(0,1)&\mbox{(já que $f$ preserva multipli-}\\
		&&\mbox{cação por números)}.
	\end{array}\]Sejam então $f(1,0)=(a,b)$ e $f(0,1)=(c,d)$. Assim,
	\[\begin{array}{rl}
	f(x,y)&=xf(1,0)+yf(0,1)\\
	&=x(a,b)+y(c,d)\\
	&=(xa,ab)+(yc,yd)=(xa+yc,xb+yd)
	\end{array}\]e, pela proposição anterior, sabemos que $f=T_A$, onde $A=\begin{pmatrix}
	a&c\\b&d
	\end{pmatrix}$, como queríamos mostrar.
\end{proof}

\begin{df}
	Uma função $f:\R^2\to \R^2$ será dita \textbf{linear} se $f(v+\lambda u)=f(v)+\lambda f(u)$ para quaisquer $\lambda\in \R$ e $v,u\in\R^2$.
\end{df}

Então o teorema acima nos diz que as funções lineares são exatamente as funções ``multiplique o seu vetor por uma matriz''.

\begin{ex}
	Sejam $f,g:\R^2\to R^2$ funções lineares. Será que $g\circ f$ e $f\circ g$ também são lineares?
	
	Por exemplo, se $f=T_A$ e $g=T_B$, com
	\[A=\begin{pmatrix}
	1&3\\
	2&5
	\end{pmatrix}\mbox{ e } B=\begin{pmatrix}
	2&5\\
	1&3
	\end{pmatrix},\] podemos lembrar que $(f\circ g)(x,y)$ é simplesmente
	\[\begin{pmatrix}
	1&3\\
	2&5
	\end{pmatrix}\left(\begin{pmatrix}
	2&5\\
	1&3
	\end{pmatrix}\begin{pmatrix}
	x\\y
	\end{pmatrix}\right)\]escrito em forma de vetor. Contudo, o produto de matrizes é associativo! Então esse produto é a mesma coisa que
	\[\left(\begin{pmatrix}
	1&3\\
	2&5
	\end{pmatrix}\begin{pmatrix}
	2&5\\
	1&3
	\end{pmatrix}\right)\begin{pmatrix}
	x\\y
	\end{pmatrix}=\begin{pmatrix}
	5&14\\9&25
	\end{pmatrix}\begin{pmatrix}
	x\\y
	\end{pmatrix},\] ou seja, a função $f\circ g$ é dada pela matriz $AB=\begin{pmatrix}
	5&14\\9&25
	\end{pmatrix}$.
	
	Similarmente, a função $g\circ f$ leva um vetor $(x,y)$ no vetor dado pelo produto de matrizes
	\[\begin{pmatrix}
	2&5\\
	1&3
	\end{pmatrix}\left(\begin{pmatrix}
	1&3\\
	2&5
	\end{pmatrix}\begin{pmatrix}
	x\\y
	\end{pmatrix}\right)\]escrito em forma de vetor. Contudo, como notamos acima, o produto de matrizes é associativo, então esse produto pode ser escrito como
	\[\left(\begin{pmatrix}
	2&5\\
	1&3
	\end{pmatrix}\begin{pmatrix}
	1&3\\
	2&5
	\end{pmatrix}\right)\begin{pmatrix}
	x\\y
	\end{pmatrix}=\begin{pmatrix}
	12 & 31\\7&18
	\end{pmatrix},\]ou seja, a função $g\circ f$ é dada pela matriz $BA$.
	
	
	Em suma, isso nos diz o seguinte: A definição de multiplicação de matrizes é feita de forma que a composição de função lineares seja a função dada pelo produto das matrizes correspondentes.
	
	Ou, em outras palavras, poderíamos definir $AB$ como sendo a matriz tal que $T_A\circ T_B=T_{AB}$.
\end{ex}

\begin{exerc}
	Mostre, sem utilizar exemplos concretos, que se $f$ e $g$ são lineares, então $f\circ g$ e $g\circ f$ também são (dica: use o exemplo acima).
\end{exerc}

\section{Subespaços}
\subsection{Núcleo e imagem}

Agora vamos introduzir dois conceitos que vão nos ajudar a entender ainda melhor o espaço de vetores e transformações lineares.

\begin{df}
	Dada uma função linear $f:\R^2\to \R^2$ definimos o \textbf{núcleo de $f$} como sendo o conjunto $\Ker f$ dado por
	\[\Ker f:=\{v\in \R^2\mid f(v)=(0,0)\},\]ou seja, são os pontos de $\R^2$ que vão na origem quando aplicamos $f$.
\end{df}

\begin{rmk}
	O símbolo $\Ker$ vem da palavra inglesa \emph{kernel} que significa ``a parte macia de um grão'' ou ``a parte central''.
\end{rmk}

\begin{ex}
	Considere função $f:\R^2\to\R^2$ que leva $(x,y)$ em $(x,0)$, ou seja, $f(x,y)=(x,0)$. Claramente, $f$ é linear (verifique!) e é dada pela matriz $A=\begin{pmatrix}
	1&0\\0&0
	\end{pmatrix}$:
	\[\begin{pmatrix}
	1&0\\0&0
	\end{pmatrix}\begin{pmatrix}
	x\\y
	\end{pmatrix}=\begin{pmatrix}
	x\\0
	\end{pmatrix}.\]
	
	Quem é o núcleo de $f$? São todos os pontos $(x,y)\in \R^2$ tais que $f(x,y)=(0,0)$. Mas $f(x,y)=(x,0)$. Então $(x,0)=f(x,y)=(0,0)$ se, e somente se, $x=0$ -  ou seja,
	\[\Ker f=\{(x,y)\in \R^2\mid x=0\}.\]
	
	Será que isso nos diz alguma coisa sobre a matriz $A$? Bom, resolvendo o sistema $AX=0$ temos:
	\[\begin{pmatrix}
	1&0\\0&0
	\end{pmatrix}\begin{pmatrix}
	x\\y
	\end{pmatrix}=\begin{pmatrix}
	0\\0
	\end{pmatrix}\]ou seja, $x=0$ e $0y=0$. Isso nos diz que
	\[S_0:=\{(x,y)\in\R^2\mid x=0\}.\]
	
	Mas isso é a mesma coisa que $\Ker f$!
	
	Em outras palavras, o núcleo de uma função linear \textbf{nada mais é do que o conjunto solução do sistema homogêneo associado àquela função}.
	
	\tcblower
	Antes de avançarmos, vamos ver o que acontece quando temos uma matriz com uma linha nula:
	\[A=\begin{pmatrix}
	a&b\\0&0
	\end{pmatrix}.\] Nesse caso, o sistema homogêneo
	\[\begin{pmatrix}
	a&b\\0&0
	\end{pmatrix}\begin{pmatrix}
	x\\y
	\end{pmatrix}=\begin{pmatrix}
	0\\0
	\end{pmatrix}\]nos dá a equação $ax+by=0$... E só. Então se ambos $a=b=0$, isso não teria graça - estaríamos dizendo simplesmente que $0+0=0$.
	
	Vamos supor, então, que ou $a$ ou $b$ é diferente de 0 - por exemplo, $a$: Nesse caso, podemos isolar $x$, fazendo $x=\dfrac{-b}{a}y$.
	
	Assim, vemos que para cada possível valor de $y$ podemos encontrar um único valor de $x$ correspondente. Ora, isso nos diz que se uma matriz tem uma linha nula, então certamente o sistema homogêneo tem várias soluções além da trivial - basta dar valores para $y$ na fórmula acima!
\end{ex}

\begin{prop}
	Para qualquer função linear $f:\R^2\to\R^2$, $\Ker f\neq \varnothing$.
\end{prop}
\begin{proof}
	Como $f$ é linear, existe matriz $A$ tal que $f=T_A$. Mas, pelo exemplo acima, vimos que $\Ker f= S_0$, em que $S_0$ é o conjunto de soluções de $AX=0$. Contudo, já vimos que $(0,0)\in S_0$. Segue que $(0,0)\in \Ker f$ e, portanto, $\Ker f\neq\varnothing$.
	
	
	Alternativamente, podemos calcular explicitamente:
	
	\[f(0,0)=f(1,0)+f(-1,0)=f(1,0)-f(1,0)=(0,0)\]e chegar à mesma conclusão.
\end{proof}

Agora vamos usar o núcleo para inferir alguns resultados importantes da álgebra vetorial:

\begin{prop}
	Seja $f:\R^2\to\R^2$ linear. Então $f$ é injetiva se, e somente se, $\Ker f=\{(0,0)\}$.
\end{prop}

\begin{proof}
	Se $f$ é injetiva, então $f(v)=f(u)$ implica $v=u$. Escolha, então, qualquer $v\in \Ker f$ - ou seja, $f(v)=(0,0)$. Mas, pela proposição anterior, sabemos que $f(0,0)=(0,0)$. Agora, como $f$ é injetiva e $v$ e $(0,0)$ têm a mesma imagem, isso implica que $v=(0,0)$. Ora, nós mostramos que qualquer elemento $v$ no núcleo \textbf{tem} que ser $(0,0)$ - segue que o único elemento do núcleo é $(0,0)$.
	
	Por outro lado, suponha que $\Ker f=\{(0,0)\}$. Tome, então, $v,u\in\R^2$ tais que $f(v)=f(u)$. Se mostrarmos que $v=u$, teremos mostrado que $f$ é injetiva. Mas $f$ é linear, então dizer que $f(v)=f(u)$ é a mesma coisa que dizer que $f(v-u)=(0,0)$. Mas estamos supondo que $\Ker f=\{(0,0)\}$ - ou seja, se $f(w)=(0,0)$, então $w=(0,0)$. Como nós temos que $f(v-u)=(0,0)$, nossa hipótese de $\Ker f=\{(0,0)\}$ nos garante que $v-u=(0,0)$. Finalmente, isso é a mesma coisa que $v=u$. Logo, nós concluímos que $f$ é, de fato, injetiva, o que encerra a demonstração.
\end{proof}

\begin{ex}
	Vamos voltar a comparar núcleos e soluções do sistema homogêneo. O resultado que mostramos acima nos diz que uma função linear é injetiva se, e somente se, o núcleo é ``trivial'' - ou seja, o sistema homogêneo só possui a solução trivial $S_0=\{(0,0)\}$. Como podemos interpretar, do ponto de vista de matrizes, a injetividade de uma função linear?
	
	Pela proposição acima, uma função linear é injetiva se, e somente se, o único ponto que vai no zero é o zero. Em termos de matrizes, então, isso significa ``uma matriz $A$ é \textit{injetiva} se $AX=0$ implica $X=0$''. Em outras palavras, uma matriz é injetiva se o sistema homogêneo \textit{possui solução única} (a solução trivial).
	
	Mas já vimos acima que um sistema tem solução única se, e somente se, a forma escalonada da matriz $A$ não possui linhas de zeros.
	
	Juntando tudo isso, vemos que \textit{uma matriz corresponde a uma função linear injetiva se, e somente se, sua forma escalonada \textbf{não possui} linhas de zeros}.
	
	\tcblower
	
	Por exemplo, considere as matrizes 
	\[A=\begin{pmatrix}
	1&0\\0&1
	\end{pmatrix},\quad B=\begin{pmatrix}
	1&1\\
	-1&1
	\end{pmatrix},\quad C=\begin{pmatrix}
	2&3\\
	4&6
	\end{pmatrix}.\] $A$ já está escalonada, e não possui linhas de zeros, então certamente $T_A$ é injetiva.
	
	$B$ ainda não está escalonada:
	\[\begin{pmatrix}
	1&1\\-1&1
	\end{pmatrix}\rightsquigarrow\begin{pmatrix}
	1&1\\0&2
	\end{pmatrix}\rightsquigarrow\begin{pmatrix}
	1&1\\0&1
	\end{pmatrix}\rightsquigarrow\begin{pmatrix}
	1&0\\0&1
	\end{pmatrix}.\] Então $B$ escalonada é $A$, que já vimos que não possui linhas de zeros. Segue que $T_B$ é injetiva.
	
	$C$ também não está escalonada ainda:
	\[\begin{pmatrix}
	2&3\\4&6
	\end{pmatrix}\rightsquigarrow\begin{pmatrix}
	1&3/2\\4&6
	\end{pmatrix}\rightsquigarrow\begin{pmatrix}
	1&3/2\\0&0
	\end{pmatrix}.\] Então $C$ escalonada tem uma linha de zeros, ou seja, $T_C$ \textbf{não} é injetiva.
\end{ex}

A discussão e os exemplos acima sugerem que toda função linear injetiva possui inverso. Isso pode soar estranho a princípio: Por exemplo, esquecendo o adjetivo ``linear'', sabemos que uma função $f$ entre dois conjuntos quaisquer possui inversa se, e somente se, $f$ é injetiva e sobrejetiva. Dito de outra maneira, em geral \textit{não basta uma função ser injetiva para que ela seja inversível}.

Essa é uma das grandes vantagens de funções lineares: Com o acréscimo desse adjetivo ``linear'', podemos mostrar o seguinte resultado:

\begin{theorem}
	Seja $f:\R^2\to \R^2$ uma função linear. Então $f$ é inversível se, e somente se, $f$ é injetiva, o que acontece se, e somente se, $f$ é sobrejetiva.
\end{theorem}

Em outras palavras, no mundo dos vetores e funções lineares, \textit{ser injetivo, sobrejetivo e inversível é \textbf{a mesma coisa}}.

Vamos agora caminhar em direção a demonstra esse resultado. Para isso vamos precisar de alguns conceitos.

\begin{df}
	Seja $S\subset \R^2$ um subconjunto de $\R^2$ tal que para quaisquer $v,u\in S$ e qualquer $\lambda\in\R$ temos que $v+\lambda u\in S$. Então diremos que $S$ é um \textbf{subespaço de $\R^2$}.
\end{df}

\begin{rmk}
	Note que, por definição, se $v\in S$ e $S$ é subespaço de $\R^2$, então $\lambda v$ também está em $S$, para qualquer $\lambda\in \R$. Disso segue que se $S\neq \varnothing$, então $S$ tem infinitos elementos.
\end{rmk}

\begin{exerc}
	Mostre que $\{(0,0)\}$, $\{(x,y)\in \R^2\mid y=\lambda x \mbox{ para algum }\lambda\in \R\}$ e $\R^2$ são subespaços de $\R^2$.
\end{exerc}

Com isso, podemos finalmente ter um bom resultado:

\begin{prop}
	Para qualquer função linear $f:\R^2\to\R^2$, temos que $\Ker f$ é um subespaço de $\R^2$.
\end{prop}

\begin{proof}
	Tome $v,u\in \Ker f$ e $\lambda\in \R$. Vamos mostrar que $f(v+\lambda u)=(0,0)$ e, portanto, $v+\lambda u\in \Ker f$. Calculando:
	\[f(v+\lambda u)=f(v)+f(\lambda u)=f(v)+\lambda f(u),\]já que $f$ é linear. Agora, como $v,u\in \Ker f$, temos que $f(v)=f(u)=(0,0)$, logo
	\[f(v+\lambda u)=f(v)+\lambda f(u)=(0,0)+\lambda (0,0)=(0,0)+(0,0)=(0,0),\]ou seja, $v+\lambda u\in \Ker f$. 
	
	Segue que $\Ker f$ é subespaço de $\R^2$, como queríamos mostrar.
\end{proof}
\begin{cor}
	Seja $f:\R^2\to\R^2$ linear. Então $\Ker f$ ou tem um único ponto ($(0,0)$) ou tem infinitos pontos. 
\end{cor}

\begin{rmk}
	O resultado acima é uma justificativa para a afirmação de que um sistema linear ou não tem solução, ou tem uma solução ou tem infinitas soluções.
\end{rmk}

\begin{exerc}
	Prove a observação acima - ou seja, tome qualquer sistema linear $AX=B$ com $A\in M_2(\R)$ e mostre que o conjunto de soluções ou é vazio, ou tem um único ponto, ou tem infinitos pontos.
\end{exerc}

Temos, por outro lado, outro subespaço associado a uma função linear - a imagem desta:

\begin{df}
	Seja $f:X\to Y$ uma função entre os conjuntos $X$ e $Y$. Denotamos por $\im f$ a \textbf{imagem de $f$} o conjunto dado por 
	\[\im f:=\{v\in Y\mid \exists u\in X\mbox{ tal que }f(u)=v\}.\]
\end{df}

Ou seja, a imagem de uma função são todos os pontos que você pode obter aplicando aquela função.

\begin{ex}
	Considere a função $f:\N\to \N$ dada por $f(n)=n+1$. Qual a imagem de $f$? São todos os números naturais que podem ser escritos como ``algum número natural + 1''. Por exemplo, 2 está na imagem, pois $2=1+1$. Similarmente, 7 está na imagem, pois $7=6+1$. O único número natural que não está na imagem é o 0 - de fato, não existe nenhum número natural $n$ tal que $0=n+1$. Então
	\[\im f=\N-\{0\}=\{1,2,3,4,5,\cdots\}.\]
	\tcblower
	Considere agora a função $\lVert-\rVert:\R^2\to \R$ dada por $\lVert(x,y)\rVert:=\sqrt{x^2+y^2}$. Essa função é chamada de \textbf{norma} - ela mede o tamanho de um vetor em $\R^2$. Qual a imagem dessa função, ou seja, quais são todos os possíveis tamanhos de vetores?
	
	Olhando para a fórmula da função, vemos que a norma é a raiz quadrada (que é sempre não-negativa) da soma de dois números positivos (que é sempre não-negativa), ou seja, a imagem de $\lVert-\rVert$ são todos os números reais não-negativos.
	
	De fato, para qualquer número não-negativo $\lambda\in\R$ podemos exibir um vetor com essa norma: $(\lambda,0)$. Computando a norma de $(\lambda,0)$ temos:
	\[\lVert(\lambda,0)\rVert=\sqrt{\lambda^2+0^2}=\sqrt{\lambda^2}\]e como $\lambda\geq0$, temos finalmente $\sqrt{\lambda^2}=\lambda$, ou seja, existe (pelo menos) um vetor $v$ tal que $\lVert v\rVert=\lambda$ e, portanto, $\lambda\in\im \lVert-\rVert$.
\end{ex}

\begin{prop}
	Para qualquer função linear $f:\R^2\to \R^2$, temos que $\im f$ é um subespaço de $\R^2$.
\end{prop}
\begin{proof}
	Tome $v,u$ em $\im f$ e $\lambda\in \R$ qualquer. Vamos mostrar que $v+\lambda u$ também está na imagem.
	
	Por definição, se $v\in\im f$, então $v$ é imagem de alguém, ou seja, existe $v'\in\R^2$ tal que $f(v')=v$. Similarmente, como $u\in\im f$ ele é imagem de alguém, ou seja, existe $u'\in\R^2$ tal que $f(u')=u$.
	
	Afirmamos que $v+\lambda u$ é imagem de $v'+\lambda u'$ por $f$: De fato, \(f(v'+\lambda u')=f(v')+\lambda f(u')\), já que $f$ é linear. Agora, usando que $v=f(v')$ e $u=f(u')$ temos finalmente que $f(v'+\lambda u')=v+\lambda u$, donde podemos concluir que $v+\lambda u\in\im f$ e, portanto, $\im f$ é subespaço.
\end{proof}
\begin{cor}
	Para qualquer função linear $f:\R^2\to\R^2$, $(0,0)\in\im f$.
\end{cor}

\subsection{Geradores, dependência e independência linear}

Na seção anterior introduzimos o conceito de subespaços de $\R^2$ que são subconjuntos que se comportam ``como $\R^2$'' - ou seja, são fechados para soma e multiplicação por escalar. Será que subespaços herdam alguma outra propriedade interessante de $\R^2$?
\begin{ex}
	Considere qualquer vetor $(x,y)\in \R^2$. Esse vetor pode ser escrito de maneira \textbf{única} como $(x,y)=x(1,0)+y(0,1)$.
	
	Por outro lado, também podemos escrever de maneira única qualquer vetor $(x,y)$ de $\R^2$ como $(x,y)=\dfrac{x+y}{2}(1,1)+\dfrac{x-y}{2}(1,-1)$ - por exemplo, $(5,3)$ pode ser escrito tanto como $5(1,0)+3(0,1)$ quanto como $4(1,1)+(1,-1)$: De fato, $5(1,0)+3(0,1)=(5,0)+(0,3)=(5,3)$ e $4(1,1)+(1,-1)=(4,4)+(1,-1)=(5,3)$.
	
	De maneira geométrica, isso significa que para chegar em qualquer ponto no plano partindo da origem podemos dar passos horizontais e verticais ($(1,0)$ e $(0,1)$) ou passos nas direções nordeste e noroeste ($(1,1)$ e $(1,-1)$) (sempre lembrando que podemos dar passos para frente e para trás).
\end{ex}

\begin{df}
	Dados dois vetores $v,u\in\R^2$, dizemos que \textbf{$u$ é gerado por $v$} se existe algum número real $\lambda$ tal que $u=\lambda v$.
	
	Similarmente, dado um vetor $v$, o conjunto de todos os vetores gerados por $v$ será chamado de \textbf{conjunto gerado por $v$} e denotado por $\gen(v)$. Nesse caso diremos que $v$ é um \textbf{gerador de $\gen(v)$}.
\end{df}
\begin{df}
	Dados um vetor $u\in\R^2$ e uma coleção finita de vetores $V=\{v_1,v_2,\cdots,v_n\}\subseteq\R^2$, dizemos que \textbf{$u$ é gerado por $V$} se existe uma coleção de números reais $\{\lambda_1,\lambda_2,\cdots,\lambda_n\}\subseteq\R$ tal que
	\[u=\lambda_1v_1+\lambda_2v_2+\cdots+\lambda_nv_n.\]
	
	Similarmente, dada uma coleção de vetores $V$, o conjunto de todos os vetores gerados por $V$ será chamado de \textbf{conjunto gerado por $V$} e denorado por $\gen(V)$. Nesse caso diremos que $V$ é um \textbf{conjunto de geradores de $\gen(V)$} ou, equivalentemente, um \textbf{gerador de $\gen(V)$}.
\end{df}

\begin{rmk}
	A terminologia $\gen$ para se referir ao conjunto gerado vem do termo inglês \emph{generated}, ``gerado'' em português.
\end{rmk}

\begin{ex}
	Continuando o exemplo acima, vimos que qualquer vetor $(x,y)\in\R^2$ pode ser escrito tanto como $x(1,0)+y(0,1)$ quanto como $\dfrac{x+y}{2}(1,1)+\dfrac{x-y}{2}(1,-1)$. Ou seja, tanto $E=\{(1,0),(0,1)\}$ quanto $V=\{(1,1),(1,-1)\}$ são geradores de $\R^2$ - ou, em outras palavras, $\R^2=\gen(E)=\gen(V)$.
	
	Por outro lado, dado o vetor $v=(2,3)$, qual o conjunto gerado por $v$ - ou seja, quem é $\gen(v)$? Por definição, temos:
	\[\gen(v)=\{u\in\R^2\mid \exists\lambda\in\R\mbox{ tal que }u=\lambda v\},\]ou seja, $\gen(v)$ é o conjunto de todos os múltiplos de $v$. Por exemplo, $(4,6)$, $(-2,-3)$, $(2\pi,3\pi)$ e $(2000,3000)$ são gerados por $v$.
\end{ex}

\begin{prop}
	Para qualquer coleção finita de vetores $V$, temos que $\gen(V)$ é um subespaço de $\R^2$.
\end{prop}
\begin{proof}
	Vamos denotar $V=\{v_1,v_2,\cdots,v_n\}$ os elementos de $V$ (o que faz sentido já que, por hipótese, $V$ tem uma quantidade finita de elementos). Tome então $u$ e $w$ em $\gen(V)$, e qualquer número real $\alpha$.
	
	Como $u$ é gerado por $V$, por definição existem $\{\lambda_1,\lambda_2,\cdots,\lambda_n\}\subseteq\R$ tais que 
	\[u=\lambda_1v_1+\lambda_2v_2+\cdots+\lambda_nv_n.\]Similarmente, como $w$ é gerado por $V$, por definição existem $\{\mu_1,\mu_2,\cdots,\mu_n\}\subseteq\R$ tais que
	\[w=\mu_1v_1+\mu_2v_2+\cdots\mu_nv_n.\]Isso nos diz que multiplicando $w$ por $\alpha$ obtemos
	\[\alpha w=\alpha(\mu_1v_1+\mu_2v_2+\cdots\mu_nv_n)=(\alpha\mu_1)v_1+(\alpha\mu_2)v_2+\cdots(\alpha\mu_n)v_n\]o que nos diz que $\alpha w\in\gen(V)$, ou seja, $\gen(V)$ é fechado por multiplicação por escalares.
	
	Por outro lado, somando $u$ e $w$ obtemos
	\[u+w=(\lambda_1v_1+\lambda_2v_2+\cdots+\lambda_nv_n)+(\mu_1v_1+\mu_2v_2+\cdots\mu_nv_n)=(\lambda_1+\mu_1)v_1+(\lambda_2+\mu_2)v_2+\cdots+(\lambda_n+\mu_n)v_n\]o que nos diz que $u+w\in\gen(V)$, ou seja, $\gen(V)$ é fechado por somas.
	
	Juntando as duas conclusões, vemos que $\gen(V)$ é um subespaço, como queríamos mostrar.
\end{proof}

Com isso, então, vamos passar a classificar os subespaços de $\R^2$:

\begin{prop}
	O conjunto $\{(0,0)\}$ é um subespaço de $\R^2$.
\end{prop}
\begin{proof}
	Claramente $(0,0)+(0,0)=(0,0)$ e $\lambda(0,0)=(0,0)$ para qualquer $\lambda\in \R$. Segue que $\{(0,0)\}$ é fechado por somas e produtos por escalares.
\end{proof}
\begin{cor}
	$\{(0,0)\}$ é o único subespaço de $\R^2$ com apenas um ponto.
\end{cor}
\begin{proof}
	Já vimos que todo subespaço contém a origem. Então se um subespaço contém apenas um ponto, esse ponto é a origem, como queríamos mostrar.
\end{proof}

\begin{rmk}
	Muitas vezes vamos denotar o conjunto $\{(0,0)\}$ e o vetor $(0,0)$ simplesmente por $0$, quando não houver ambiguidade.
\end{rmk}

\begin{ex}
	Com o resultado acima, então, vemos que nossos subespaços sempre têm pelo menos um ponto. Será que é possível que um subespaço tenha exatamente dois pontos?
	
	Bom, se o subespaço $S$ tem dois pontos $v$ e $u$, ou $v$ ou $u$ é diferente de $(0,0)$. Suponha (sem perda de generalidade) que $v\neq(0,0)$. Nesse caso, como $S$ é subespaço, $2v\in S$. Mas como $S$ só tem dois pontos, ou $2v=u$, ou $2v=v$.
	
	Se $2v=v$, então subtraindo $v$ dos dois lados obtemos $v=(0,0)$, o que é absurdo, porque nós escolhemos $v\neq(0,0)$ acima. Então $2v$ tem que ser $u$.
	
	Mas como $v\neq(0,0)$, necessariamente $u=(0,0)$ e, portanto, $2v=(0,0)$. Mas isso nos diz, novamente, que $v=(0,0)$ que, como já vimos, contraria o fato de $v\neq(0,0)$.
	
	Ou seja, se nós fôssemos capazes
	 de construir um subespaço com exatamente dois pontos, nós seríamos capazes de mostrar que $0\neq 0$, o que é um absurdo! A única conclusão possível, então, é que \textit{não somos capazes de construir um subespaço com exatamente dois pontos}.
\end{ex}

Na verdade, o que fizemos acima foi uma prova do seguinte resultado:
\begin{prop}
	Seja $S\subseteq\R^2$. Se $S\neq 0$, então $S$ tem infinitos pontos.
\end{prop}
\begin{proof}
	Se $S\neq 0$, existe $v\in S$ diferente de $(0,0)$. Como $S$ é subespaço, $\lambda v\in S$ para todo $\lambda\in \R$. Como $\lambda v\neq \mu v$ se $\lambda\neq \mu$, temos que para cada número real temos um vetor diferente, e como existem infinitos números reais temos infinitos vetores em $S$.
\end{proof}

Isso nos dá uma intuição do próximo resultado:
\begin{prop}
	Toda reta passando pela origem é um subespaço gerado por um único vetor não-nulo, e todo subespaço gerado por um único vetor não-nulo é uma reta passando pela origem.
\end{prop}
\begin{proof}
	Tome $r$ uma reta passando pela origem. Então $r=\{(x,y)\in\R^2\mid ax+by=0\}$ para algum par de números reais $a,b\in \R$ não ambos nulos. Afirmamos que $r=\gen(v)$, em que $v=(-b,a)$.
	
	Para mostrar isso, vamos mostrar que todo ponto da reta $r$ é gerado por $v$, e que todo ponto gerado por $v$ está na reta $r$.
	
	Tome $u=\lambda v$ para algum $\lambda\in \R$, ou seja, $u=(-\lambda b,\lambda a)$. Então computando
	\[a(-\lambda b)+b(\lambda a)=-\lambda a b+\lambda ab=0\]vemos que $u\in r$.
	
	Por outro lado, tome $P\in r$, ou seja, $P=(p,q)$ com $ap+bq=0$. Como $r$ é reta, $a\neq0$ ou $b\neq 0$.
	
	Se $a\neq0$, então podemos reescrever $ap+bq=0$ como $p=\dfrac{-b}{a}q$ e, portanto, $P=(p,q)=\left(\dfrac{-b}{a}q,q\right)$. Disso temos que \(P=q\left(\dfrac{-b}{a},1\right)\) e, finalmente, $aP=q(-b,a)=qv$. Novamente, como $a\neq 0$, isso nos diz que $P=\dfrac{q}{a}v$, donde vemos que $P\in \gen(v)$.
	
	Similarmente se $b\neq 0$ podemos mostrar que $P=\dfrac{p}{b}v$, e chegar na mesma conclusão.
	
	Segue que $r=\gen v$, como queríamos mostrar.
\end{proof}

O próximo passo, a essa altura, seria mostrar que $\R^2$ é gerado por dois vetores não-nulos e que qualquer subespaço gerado por dois vetores não-nulos é $\R^2$. A primeira dessas afirmações é fácil: $\R^2$ é gerado por $\{(1,0),(0,1)\}$ que são dois vetores não-nulos. Contudo, a segunda afirmação é \textit{falsa}:

\begin{ex}
	Considere o conjunto $V=\{(1,1),(2,2)\}\subseteq\R^2$. Quem é $\gen(V)$?
	
	Por definição, são os vetores $v\in \R^2$ tais que existem números reais $\lambda_1,\lambda_2\in\R$ tais que \[v=\lambda_1(1,1)+\lambda_2(2,2).\]
	
	Primeira afirmação: $\R^2\neq \gen(V)$. Para ver isso, basta notar que o ponto $(1,0)$ não é gerado por $V$. Dito de outra maneira, suponha que existem números reais $x,y$ tais que $(1,0)=x(1,1)+y(2,2)$. Em particular, comparando as coordenadas, teríamos que $1=x+2y$ e $0=x+2y$. Disso, poderíamos concluir que $1=x+2y=0$, ou seja, $1=0$, um absurdo! Portanto não podem existir $x,y$ tais que $(1,0)=x(1,1)+y(2,2)$, ou seja, $(1,0)\notin\gen(V)$ e, portanto, $\R^2\neq \gen (V)$.
	
	Segunda afirmação: $\gen(V)=\gen((1,1))$ (e, portanto, uma reta, pois $(1,1)\neq 0$). Isso é mais simples: Tome qualquer $v\in \gen(V)$ e escreva $v=\lambda_1(1,1)+\lambda_2(2,2)$. Agora note que $(2,2)=2(1,1)$, e podemos reescrever a expressão anterior de $v$ como
	\[v=\lambda_1(1,1)+\lambda_2\cdot 2(1,1)=(\lambda_1+2\lambda_2)(1,1),\]ou seja, $v\in\gen ((1,1))$. 
	
	Por outro lado, dado qualquer $u\in\gen((1,1))$, escreva $u=\lambda (1,1)=\lambda(1,1)+0(2,2)$, logo $u\in\gen(V)$.
	
	Disso concluímos que $\gen(V)$ é uma reta, e certamente não pode ser o plano todo.
\end{ex}

Precisamos então de criar um critério para saber quando um conjunto com dois vetores gera ou não o plano todo.

\begin{df}
	Dada uma coleção finita de vetores não-nulos $V=\{v_1,v_2,\cdots,v_n\}$ diremos que ela é \textbf{linearmente dependente} se $v_i\in\gen(V-\{v_i\})$ para algum $i\in\{1,2,\cdots,n\}$, ou seja $V$ é linearmente dependente se algum dos vetores $v_i$ for gerado pelos outros.
	
	Similarmente, se a coleção $V$ não for linearmente dependente, diremos que $V$ é \textbf{linearmente independete}, ou seja, nenhum dos vetores $v_i$ é gerado pelos outros.
\end{df}

\begin{rmk}
	Vamos usar as abreviações \textbf{l.d.} e \textbf{l.i.} para linearmente dependente e linearmente independente, respectivamente.
\end{rmk}
\begin{ex}
	O exemplo que demos acima, com $V=\{(1,1),(2,2)\}$ é um exemplo de uma coleção l.d.: O vetor $(2,2)$ é gerado pelo vetor $(1,1)$.
	
	\tcblower
	A coleção $\{(1,0),(0,1),(2,3)\}$ é l.d. pois o vetor $(2,3)$ é gerado pela coleção $\{(1,0),(0,1)\}$.
	
	Por outro lado, a coleção $\{(1,0),(0,1)\}$ é l.i. pois $(1,0)$ não gera nem é gerado por $(0,1)$.
\end{ex}

\begin{rmk}
	Qualquer coleção unitária (i.e. com um só vetor não-nulo) é l.i. Por definição, considere a coleção $V=\{v\}$. Para essa coleção ser l.d. deveria haver algum vetor (diferente de $v$) em $V$ que gerasse $v$. Mas $V$ não tem nenhum vetor diferente de $v$, então não tem nenhum vetor diferente de $v$ que gere $v$ e, portanto, $V$ é l.i.
\end{rmk}

\begin{lemma}\label{lem:ld reduz pra li}
	Toda coleção finita de vetores l.d. pode ser reduzida em uma coleção l.i.
\end{lemma}
\begin{proof}
	Se a coleção $V$ é finita, ela tem uma quantidade de elementos, $n$. Como a coleção é l.d., existe algum vetor $v_1$ que é gerado pelos outros.
	
	Considere a coleção $V-\{v_1\}$. Ela tem $n-1$ elementos. Se for l.i., acabamos. Se não, existe algum vetor $v_2$ que é gerado pelos outros.
	
	Considere a coleção $V-\{v_1,v_2\}$. Ela tem $n-2$ elementos. Se for l.i., acabamos. Se não, existe algum vetor $v_3$ que é gerado pelos outros.
	
	E assim por diante.
	
	Contudo, já vimos que conjuntos unitários são l.i., então, no pior dos casos, vamos ter que remover $n-1$ vetores de $V$, ficando com o conjunto $V-\{v_1,v_2,\cdots,v_{n-1}\}=\{v_n\}$ que é l.i., como queríamos mostrar.
\end{proof}
\begin{lemma}\label{lem:ld sub li gen}
	Seja $V$ conjunto finito de vetores l.i. e $u\in\R^2$ vetor tal que $V\cup\{u\}$ é l.d. Então $\gen (V)=\gen(V\cup\{u\})$. 
\end{lemma}
\begin{proof}
	Tome qualquer $w\in\gen(V\cup\{u\})$. Vamos mostrar que $w\in\gen(V)$. 
	
	Se $V$ é l.i. com $n$ elementos e $V\cup \{u\}$ é l.d. temos duas possibilidades:
	\begin{enumerate}[1)]
		\item $u$ é gerado por $V$, ou;
		\item algum elemento de $V$ é gerado pelos outros elementos de $V$ e $\{u\}$.
		
		Nesse caso, vamos chamar esse elemento de $v_1$. Então, existem números reais $\{\lambda_2,\lambda_3,\cdots,\lambda_n,\mu\}$ tais que $$v_1=\lambda_2v_2+\lambda_3v_2+\cdots+\lambda_nv_n+\mu u.$$Se $\mu=0$, estaríamos dizendo que $v_1$ é gerado pelos elementos de $V$, o que contrariaria o fato de $V$ ser l.i. Então $\mu\neq 0$. Assim, podemos reescrever a expressão acima isolando $u$
		\[u=\frac{1}{\mu}v_1+\frac{-\lambda_2}{\mu}v_2+\cdots+\frac{-\lambda_n}{\mu}v_n\]e ver que $u\in\gen(V)$, ou seja, os dois casos são a mesma coisa.
		
		Agora, $w\in\gen(V\cup\{u\})$ implica que existem números reais $\{w_1,w_2,\cdots,w_n,w_u\}$ tais que
		\[w=w_1v_1+w_2v_2+\cdots+w_nv_n+w_uu.\] Mas como $u$ é gerado por $V$, existem $\{u_1,u_2,\cdots,u_n\}$ números reais tais que
		\[u=u_1v_1+u_2v_2+\cdots+u_nv_n\]e, substituindo acima, temos:
		\begin{align*}
			w&=w_1v_1+w_2v_2+\cdots+w_nv_n+w_uu\\
			&w_1v_1+w_2v_2+\cdots+w_nv_n+w_u(u_1v_1+u_2v_2+\cdots+u_nv_n)\\
			&(w_1+w_uu_1)v_1+(w_2+w_uu_2)v_2+\cdots+(w_n+w_uu_n)v_n,
		\end{align*}ou seja $w\in\gen(V)$, como queríamos mostrar.
	\end{enumerate}
\end{proof}
\begin{lemma}
	Se $V$ é finito e l.i., então $V-\{v\}$ também é l.i. para qualquer $v\in V$.
\end{lemma}
\begin{proof}
	Seja $V$ finito e l.i. $v\in V$. Em particular, $V-\{v\}$ também é finito. Seja $V-\{v\}=\{v_1,v_2,\cdots,v_n\}$. Suponha que $v_1$ é gerado pelos outros $v_i$, ou seja, existem números reais $\{\lambda_2,\lambda_3,\cdots,\lambda_n\}$ tais que $v_1=\lambda_2v_2+\lambda_3v_3+\cdots+\lambda_nv_n$. Em particular, poderíamos escrever $v_1=\lambda_2v_2+\lambda_3v_3+\cdots+\lambda_nv_n+0v$ e ver que $v_1$ já era gerado em $V$, o que contraria o fato de $V$ ser l.i. Segue que $v_1$ não pode ser gerado pelos outros $v_i$.
	
	Como temos um número finito de $v_i$, podemos repetir o mesmo raciocínio para eles e concluir que cada $v_j$ não podem ser gerados pelos outros $v_i$, ou seja, o conjunto $V-\{v\}$ é l.i., como queríamos mostrar.
\end{proof}

Finalmente, vamos enunciar o resultado que precisamos:

\begin{tcolorbox}[breakable,colback=red!5!white,colframe=red!80!white,title=\normalsize {\sc AVISO!},coltitle=black]
	A demonstração abaixo é bastante técnica.
	
	Não se preocupe em entendê-la por agora. Leia o enunciado do teorema e certifique-se de que entendeu.
	
	Se tiver interesse, por favor, leia e tente acompanhar a prova, mas não tenha vergonha em passar para frente caso tenha dificuldade.
\end{tcolorbox}
\begin{theorem}\label{thm:gen li mesmo tamanho}
	Sejam $V$ e $W$ dois conjuntos finitos de geradores l.i. para algum subespaço $S$. Então $\#V=\#W$.
\end{theorem}
\begin{proof}
	Como $V$ e $W$ são finitos, vamos dizer que $\#V=n$ e $\#W=m$. Escreva então
	\[V=\{v_1,v_2,\cdots,v_n\},\quad W=\{w_1,w_2,\cdots,w_m\}\]os elementos de $V$ e $W$. Como $V$ gera $S$ e $W\subseteq S$, podemos escrever cada $w_i\in W$ como
	\[w_i=\lambda_{i,1}v_1+\lambda_{i,2}v_2+\cdots+\lambda_{i,n}v_n.\] Dito de outra maneira, temos uma matriz $m\times n$
	\[M=\begin{pmatrix}
	\lambda_{1,1}&\lambda_{1,2}&\cdots&\lambda_{1,n}\\
	\lambda_{2,1}&\lambda_{2,2}&\cdots&\lambda_{2,n}\\
	\vdots&\vdots&\ddots&\vdots\\
	\lambda_{m,1}&\lambda_{m,2}&\cdots&\lambda_{m,n}
	\end{pmatrix}\]tal que
	\[\begin{pmatrix}
	\lambda_{1,1}&\lambda_{1,2}&\cdots&\lambda_{1,n}\\
	\lambda_{2,1}&\lambda_{2,2}&\cdots&\lambda_{2,n}\\
	\vdots&\vdots&\ddots&\vdots\\
	\lambda_{m,1}&\lambda_{m,2}&\cdots&\lambda_{m,n}
	\end{pmatrix}\begin{pmatrix}
	v_1\\v_2\\\vdots\\v_n
	\end{pmatrix}=\begin{pmatrix}
	w_1\\w_2\\\vdots\\w_m
	\end{pmatrix}.\]
	
	Analogamente, como $W$ gera $S$ e $V\subseteq S$, podemos escrever cada $v_i\in V$ como
	\[v_i=\mu_{i,1}w_1+\mu_{i,2}w_2+\cdots+\mu_{i,m}w_m.\]Dito de outra maneira, temos uma matriz $n\times m$
	\[N=\begin{pmatrix}
	\mu_{1,1}&\mu_{1,2}&\cdots&\mu_{1,m}\\
	\mu_{2,1}&\mu_{2,2}&\cdots&\mu_{2,m}\\
	\vdots&\vdots&\ddots&\vdots\\
	\mu_{n,1}&\mu_{n,2}&\cdots&\mu_{n,m}
	\end{pmatrix}\]tal que
	\[\begin{pmatrix}
	\mu_{1,1}&\mu_{1,2}&\cdots&\mu_{1,m}\\
	\mu_{2,1}&\mu_{2,2}&\cdots&\mu_{2,m}\\
	\vdots&\vdots&\ddots&\vdots\\
	\mu_{n,1}&\mu_{n,2}&\cdots&\mu_{n,m}
	\end{pmatrix}\begin{pmatrix}
	w_1\\w_2\\\vdots\\w_m
	\end{pmatrix}=\begin{pmatrix}
	v_1\\v_2\\\vdots\\v_n
	\end{pmatrix}.\]
	
	Substituindo a expressão obtida acima para a matriz $\begin{pmatrix}
	v_1\\v_2\\\vdots\\v_n
	\end{pmatrix}$ na equação anterior, temos:
	\[\begin{pmatrix}
	\lambda_{1,1}&\lambda_{1,2}&\cdots&\lambda_{1,n}\\
	\lambda_{2,1}&\lambda_{2,2}&\cdots&\lambda_{2,n}\\
	\vdots&\vdots&\ddots&\vdots\\
	\lambda_{m,1}&\lambda_{m,2}&\cdots&\lambda_{m,n}
	\end{pmatrix}\begin{pmatrix}
	\mu_{1,1}&\mu_{1,2}&\cdots&\mu_{1,m}\\
	\mu_{2,1}&\mu_{2,2}&\cdots&\mu_{2,m}\\
	\vdots&\vdots&\ddots&\vdots\\
	\mu_{n,1}&\mu_{n,2}&\cdots&\mu_{n,m}
	\end{pmatrix}\begin{pmatrix}
	w_1\\w_2\\\vdots\\w_m
	\end{pmatrix}=\begin{pmatrix}
	w_1\\w_2\\\vdots\\w_m
	\end{pmatrix}.\] Vamos reescrever a matriz $MN$ como
	\[MN=\begin{pmatrix}
	a_{1,1}&a_{1,2}&\cdots&a_{1,m}\\
	a_{2,1}&a_{2,2}&\cdots&a_{2,m}\\
	\vdots&\vdots&\ddots&\vdots\\
	a_{m,1}&a_{m,2}&\cdots&a_{m,m}
	\end{pmatrix}.\] Então podemos reescrever a equação matricial anterior como
	\[\begin{pmatrix}
	a_{1,1}&a_{1,2}&\cdots&a_{1,m}\\
	a_{2,1}&a_{2,2}&\cdots&a_{2,m}\\
	\vdots&\vdots&\ddots&\vdots\\
	a_{m,1}&a_{m,2}&\cdots&a_{m,m}
\end{pmatrix}\begin{pmatrix}
w_1\\w_2\\\vdots\\w_m
\end{pmatrix}=\begin{pmatrix}
w_1\\w_2\\\vdots\\w_m
\end{pmatrix}\]e obter as equações $w_i=a_{i,1}w_1+a_{i,2}w_2+\cdots+a_{i,m}w_m$, ou seja, $(1-a_{i,i})w_i=a_{i,1}w_1+a_{i,2}w_2+\cdots+a_{i,i-1}w_{i-1}+a_{i,i+1}w_{i+1}+\cdots+a_{i,m}w_m$. Se $1-a_{i,i}\neq 0$, podemos dividir ambos os lados por $1-a_{i,i}$ e concluir que $w_i$ é combinação linear dos outros elementos de $W$. 

Mas $W$ é l.i., o que nos levaria a um absurdo. Então $1-a_{i,i}=0$ e, portanto, $a_{i,i}=1$ para todos os valores de $i$, ou seja, a diagonal de $MN$ é composta de $1$s.

Contudo, as equações acima nos garantem que $(1-a_{i,i})w_i=a_{i,1}w_1+a_{i,2}w_2+\cdots+a_{i,i-1}w_{i-1}+a_{i,i+1}w_{i+1}+\cdots+a_{i,m}w_m$, e como já sabemos que $a_{i,i}=1$, temos que $a_{i,1}w_1+a_{i,2}w_2+\cdots+a_{i,i-1}w_{i-1}+a_{i,i+1}w_{i+1}+\cdots+a_{i,m}w_m=0$. 

Se algum dos $a_{i,j}$ for diferente de 0, seríamos capazes de isolar $w_j$ e escrever ele como combinação linear dos outros elementos de $W$. Mas $W$ é l.i., o que nos diz que isso é impossível - ou seja, $a_{i,j}=0$ sempre que $i\neq j$. Isso nos diz que nossa matriz $MN$ tem 0 em todo elemento fora da diagonal.

Juntando essas duas conclusões, vemos que $MN=I_m$, a matriz identidade $m\times m$.

Um raciocínio análogo nos permite concluir que $NM=I_n$, a matriz identidade $n\times n$.

Mas uma matriz possui inverso bilateral se, e somente se, a matriz é quadrada - segue que tanto $M$ quanto $N$ são quadradas, ou seja, $n=m$, como queríamos mostrar. 
\end{proof}

O que esse teorema nos garante é que sempre que nós escolhermos uma coleção l.i. de geradores de algum subespaço, seja ela qual for, ela vai ter sempre a mesma quantidade de elementos. Isso nos permite fazer a seguinte definição:

\begin{df}
	Seja $S$ um subespaço de $\R^2$. Definimos como \textbf{base} qualquer conjunto de vetores l.i. que geram $S$.
	
	Analogamente, definimos a \textbf{dimensão} de $S$ como sendo o número natural $\dim S$ definido por $$\dim S:=\#B,$$ em que $B$ é qualquer base de $S$.
\end{df}
\begin{ex}
	O subespaço $\{(0,0)\}$ só tem um vetor - $(0,0)$. Por definição, esse conjunto não é l.i. apesar de ser gerador. Assim, esse subespaço não possui base e, portanto, possui dimensão 0.
	
	\bigskip
	Por outro lado, considere o subespaço gerado pelo vetor $(2,3)$, que já vimos ser uma reta que passa pela origem e por $(2,3)$. Claramente o conjunto $\{(2,3)\}$ é l.i. (pois é unitário) e gera esse subespaço (por definição) - ou seja, é uma base. Segue que a reta que passa pela origem e pelo ponto $(2,3)$ tem dimensão 1.
	
	Não é difícil ver que qualquer reta que passa pela origem é um subespaço de dimensão 1.
	
	\bigskip
	Finalmente, considere $\R^2$. Já sabemos que $\R^2$ é gerado por $\{(1,0),(0,1)\}$ e que esse conjunto é l.i. - ou seja, uma base de $\R^2$. Segue que $\dim\R^2=2$.
	
	\bigskip
	Olha que interessante: Nós mostramos que subespaços gerados ou têm dimensão 0 (o subespaço 0), ou têm dimensão 1 (as retas pela origem), ou têm dimensão 2 (o plano todo).
\end{ex}

Para finalizar, mais alguns resultados técnicos.

\begin{lemma}
	Dada uma coleção finita de conjuntos não-nulos $V$ e um subespaço $S$ de dimensão $n$, as seguintes afirmações são equivalentes:
	\begin{enumerate}[(a)]
		\item $V$ é uma base para $S$;
		\item $V$ é l.i. e tem $n$ elementos;
		\item $V$ gera $S$ e tem $n$ elementos.
	\end{enumerate}
\end{lemma}
\begin{proof}
	Claramente (a) implica tanto (b) quanto (c).
	
	Suponha (b), ou seja, que $V$ é l.i. e tem $n$ elementos. Se $V$ não gera, existe algum elemento $v\in S$ tal que $v\notin\gen(V)$ - em particular, o conjunto $V\cup\{v\}$ seria l.i. e teria $n+1$ elementos. 
	
	Isso é um absurdo, pois se $V\cup\{v\}$ gera $S$, temos então um conjunto de geradores l.i. com mais de $n$ elementos. Por outro lado, se $V\cap\{v\}$ não gera, isso é pior ainda - significa que precisamos adicionar ainda mais gente para gerar $S$, e no final vamos ficar com muito mais do que $n$ elementos.
	
	Como isso não pode acontecer, $V$ tem que já gerar $S$, ou seja, supondo (b) concluímos (a) e (c).
	
	\bigskip
	Analogamente, suponha (c), ou seja, $V$ gera e tem $n$ elementos. Se $V$ não é l.i., ou $V=\{0,0\}$ ou $V$ é l.d. Como estamos assumindo, por hipótese, que $V$ contém apenas vetores não nulos, podemos descartar o primeiro caso.
	
	Como $V$ é l.d., podemos reduzir $V$ até obter um conjunto $V'$ l.i. tal que $\gen(V)=\gen(V')$ (pelos lemas \ref{lem:ld reduz pra li} e \ref{lem:ld sub li gen}). Como $\gen(V)=S$, por hipótese, temos que $V'$ é um conjunto gerador l.i. com menos elementos do que $n$ - o que contraria o \Cref{thm:gen li mesmo tamanho}, um absurdo.
	
	Segue que $V$ é l.i., ou seja, (c) implica (a) e (b), o que encerra a demonstração.

\end{proof}

Finalmente, vamos mostrar que todo subespaço é gerado:
\begin{prop}
	Seja $S$ subespaço de $\R^2$. Então existe $V\subseteq S$ conjunto de geradores de $S$ - ou seja, $\gen(V)=S$.
\end{prop}
\begin{proof}
	Certamente $(0,0)\in S$. Se $S-\{(0,0)\}=\varnothing$, acabou: $S=0$ e portanto é gerado por $(0,0)$.
	
	Caso contrário, existe algum $v\in S$ diferente de $(0,0)$. Então $\gen(v)\subseteq S$, já que $S$ é subespaço. Se $S-\gen(v)=\varnothing$, acabou: mostramos que $S=\gen(v)$.
	
	Caso contrário, existe algum $w\in S$ diferente de $(0,0)$ e que não é gerado por $v$. Então $\gen({v,w})\subseteq S$, já que $S$ é subespaço.
	
	Contudo, como $\{v,w\}$ foi construído l.i., o lema acima nos garante que $\{v,w\}$ é base de $\R^2$, donde $\gen(\{v,w\})=\R^2$. Então temos $\R^2\subseteq S\subseteq\R^2$, ou seja, $S=\R^2$, e $S$ é, portanto, gerado por $\{v,w\}$, como queríamos mostrar.
\end{proof}

Juntando isso tudo, temos um grande corolário que encerra esta seção:
\begin{prop}
	Todos os subespaços de $\R^2$ são: $0$, retas que passam pela origem e $\R^2$.
\end{prop}
\begin{proof}
	Já vimos, no exemplo, que todo subespaço gerado tem dimensão 0, 1 ou 2. Contudo, a proposição acima nos diz que todo subespaço é gerado. O resultado se segue.
\end{proof}
\begin{exerc}
	Mostre que a dimensão do núcleo de uma transformação linear é exatamente o número de linhas nulas na forma escalonada da matriz dessa transformação linear.
\end{exerc}
\begin{exerc}
	Mostre que a dimensão da imagem de uma transformação linear é exatamente o número de linhas não-nulas na forma escalonada da matriz dessa transformação.
\end{exerc}
\begin{exerc}
	Junte os dois exercícios acima para concluir que dada uma transformação linear $f:\R^2\to\R^2$, temos que $\dim\Ker f+\dim\im f=\dim\R^2=2$.
\end{exerc}

\section{Produtos interno e vetorial}

\subsection{Projeção ortogonal}

Em muitos problemas de física é interessante decompor um vetor não como tendo uma coordenada $X$ e uma coordenada $Y$, mas em termos de outras coordenadas arbitrárias.

\begin{ex}
	Considere os vetores abaixo:
	\[
	\definecolor{uuuuuu}{rgb}{0.26666666666666666,0.26666666666666666,0.26666666666666666}
	\begin{tikzpicture}[line cap=round,line join=round,>=triangle 45,x=1.0cm,y=1.0cm]
	\begin{axis}[
	x=1.0cm,y=1.0cm,
	axis lines=middle,
	xmin=-2,
	xmax=7,
	ymin=-1,
	ymax=6,
	xtick={-1.0,0.0,...,6.0},
	ytick={-1.0,0.0,...,5.0},]
	\clip(-1.84,-1.54) rectangle (6.94,6.04);
	\draw [->] (0.,0.) -- (6.,3.);
	\draw [->] (0.,0.) -- (2.,5.);
%	\draw [->] (0.,0.) -- (-1.,2.);
	\begin{scriptsize}
	\draw [fill=uuuuuu] (0.,0.) circle (2.0pt);
	\draw[color=black] (3.24,1.45) node {$u$};
	\draw[color=black] (1.28,2.61) node {$v$};
%	\draw[color=black] (-0.64,0.85) node {$w$};
	\end{scriptsize}
	\end{axis}
	\end{tikzpicture}\]em que $u=(6,3)$ e $v=(2,5)$.
	
	 A \textbf{projeção ortogonal de $v$ em $u$} é o vetor $\pi_{v,u}$ representado abaixo:
	 \[
	 \definecolor{uuuuuu}{rgb}{0.26666666666666666,0.26666666666666666,0.26666666666666666}
	 \begin{tikzpicture}[line cap=round,line join=round,>=triangle 45,x=1.0cm,y=1.0cm]
	 \begin{axis}[
	 x=1.0cm,y=1.0cm,
	 axis lines=middle,
	 xmin=-2,
	 xmax=7,
	 ymin=-1,
	 ymax=6,
	 xtick={-1.0,0.0,...,6.0},
	 ytick={-1.0,0.0,...,5.0},]
	 \clip(-1.84,-1.54) rectangle (6.94,6.04);
	 \draw [->] (0.,0.) -- (6.,3.);
	 \draw [->] (0.,0.) -- (2.,5.);
%	 \draw [->] (0.,0.) -- (-1.,2.);	 
	 \draw [dash pattern=on 3pt off 3pt] (3.608,1.804)-- (2.,5.);
	 \draw [->,color=red] (0.,0.) -- (3.608,1.804);
	 \begin{scriptsize}
	 \draw [fill=uuuuuu] (0.,0.) circle (2.0pt);
	 \draw[color=black] (5,2) node {$u$};
	 \draw[color=black] (1.28,2.61) node {$v$};
%	 \draw[color=black] (-0.64,0.85) node {$w$};	 
	 \draw[color=black] (1.96,0.5) node[color=red] {$\pi_{v,u}$};
	 \end{scriptsize}
	 \end{axis}
	 \end{tikzpicture}\]Esse vetor tem a seguinte propriedade especial: Considere a reta ortogonal a $u$ que passa pela origem e qualquer vetor $w$ nela:
	 \[
	 \definecolor{uuuuuu}{rgb}{0.26666666666666666,0.26666666666666666,0.26666666666666666}
	 \begin{tikzpicture}[line cap=round,line join=round,>=triangle 45,x=1.0cm,y=1.0cm]
	 \begin{axis}[
	 x=1.0cm,y=1.0cm,
	 axis lines=middle,
	 xmin=-2,
	 xmax=7,
	 ymin=-1,
	 ymax=6,
	 xtick={-1.0,0.0,...,6.0},
	 ytick={-1.0,0.0,...,5.0},]
	 \clip(-1.84,-1.54) rectangle (6.94,6.04);
	 \draw [->] (0.,0.) -- (6.,3.);
%	 \draw [->] (0.,0.) -- (2.,5.);
	 \draw [->] (0.,0.) -- (-1.,2.);	 
%	 \draw [dash pattern=on 3pt off 3pt] (3.608,1.804)-- (2.,5.);
	 \draw [dotted,domain=-5.98:15.46] plot(\x,{(-0.--6.*\x)/-3.});
%	 \draw [->,color=red] (0.,0.) -- (3.608,1.804);
	 \begin{scriptsize}
	 \draw [fill=uuuuuu] (0.,0.) circle (2.0pt);
	 \draw[color=black] (5,2) node {$u$};
%	 \draw[color=black] (1.28,2.61) node {$v$};
	 	 \draw[color=black] (-0.64,0.85) node {$w$};	 
%	 \draw[color=black] (1.96,0.5) node[color=red] {$\pi_{v,u}$};
	 \end{scriptsize}
	 \end{axis}
	 \end{tikzpicture}\]nesse caso, $w=(-1,2)$. Como $u\perp w$, claramente o conjunto $\{u,w\}$ é l.i. e, portanto, gera $\R^2$ - em particular, existem $\lambda_1,\lambda_2$ em $\R$ tais que $v=\lambda_1u+\lambda_2w$, como vemos abaixo:
	 \[
	 \definecolor{uuuuuu}{rgb}{0.26666666666666666,0.26666666666666666,0.26666666666666666}
	 \begin{tikzpicture}[line cap=round,line join=round,>=triangle 45,x=1.0cm,y=1.0cm]
	 \begin{axis}[
	 x=1.0cm,y=1.0cm,
	 axis lines=middle,
	 xmin=-2,
	 xmax=7,
	 ymin=-1,
	 ymax=6,
	 xtick={-1.0,0.0,...,6.0},
	 ytick={-1.0,0.0,...,5.0},]
	 \clip(-1.84,-1.54) rectangle (6.94,6.04);
	 \draw [->] (0.,0.) -- (6.,3.);
	 \draw [->] (0.,0.) -- (2.,5.);
	 \draw [->] (0.,0.) -- (-1.,2.);	 
	 \draw [dash pattern=on 3pt off 3pt] (3.608,1.804)-- (2.,5.);
	 \draw [dash pattern=on 3pt off 3pt] (-1.6,3.2)-- (2.,5.);
	 \draw [dotted,domain=-5.98:15.46] plot(\x,{(-0.--6.*\x)/-3.});
	 \draw [->,color=red] (0.,0.) -- (3.608,1.804);	 
	 \draw [->,color=blue] (0.,0.) -- (-1.6,3.2);
	 \begin{scriptsize}
	 \draw [fill=uuuuuu] (0.,0.) circle (2.0pt);
	 \draw[color=black] (5,2) node {$u$};
	 \draw[color=black] (1.28,2.61) node {$v$};
	 \draw[color=black] (-0.64,0.85) node {$w$};	 
	 \draw[color=black] (1.96,0.5) node[color=red] {$\lambda_1u$};
	 \draw[color=black] (-.8,2.5) node[color=blue] {$\lambda_2w$};
	 \end{scriptsize}
	 \end{axis}
	 \end{tikzpicture}\]Assim, a projeção ortogonal de $v$ em $u$ é definida como sendo $\pi_{v,u}:=\lambda_1u$, e a projeção ortogonal de $v$ em $w$ é definida como sendo $\pi_{v,w}:=\lambda_2w$.
	 \tcblower
	 Para calcular essas projeções, então, precisamos encontrar $\lambda_1$ e $\lambda_2$, ou seja, resolver a equação
	 \[(2,5)=\lambda_1(6,3)+\lambda_2(-1,2)\]que se traduz nas equações $2=6\lambda_1-\lambda_2$ e $5=3\lambda_1+2\lambda_2$. Ora, já sabemos que podemos representar isso por uma equação matricial
	 \[\begin{pmatrix}
	 6&-1\\3&2
	 \end{pmatrix}\begin{pmatrix}
	 \lambda_1\\\lambda_2
	 \end{pmatrix}=\begin{pmatrix}
	 2\\5
	 \end{pmatrix},\]ou seja, encontrar as projeções equivale a resolver o sistema linear acima.
	 
	 Vamos lá!
	 \begin{align*}
	 	\begin{augmatrix}{cc:c}
	 	6&-1&2\\3&2&5
	 	\end{augmatrix}\rightsquigarrow\begin{augmatrix}{cc:c}
	 	6&-1&2\\0&5&8
	 	\end{augmatrix}\rightsquigarrow\begin{augmatrix}{cc:c}
 	30&0&18\\0&5&8
 \end{augmatrix}\rightsquigarrow\begin{augmatrix}{cc:c}
 5&0&3\\0&5&8
 \end{augmatrix}
	 \end{align*}então nosso sistema tem solução única $5\lambda_1=3$ e $5\lambda_2=8$, ou seja, $\lambda_1=3/5$ e $\lambda_2=8/5$ (de fato, $6(3/5)-8/5=18/5-8/5=10/5=2$ e $3(3/5)+2(8/5)=9/5+16/5=25/5=5$). Com isso, então, podemos ver que $\pi_{v,u}=\lambda_1u=3/5(6,3)$ e $\pi_{v,w}=\lambda_2w=8/5(-1,2)$.
\end{ex}

\subsection{Produto interno}

\begin{ex}
	Continuando o exemplo acima, vamos calcular a projeção ortogonal de um vetor arbitrário $v=(v_1,v_2)$ em um vetor arbitrário $u=(u_1,u_2)$. Nesse caso, vamos usar $w=u^{\perp}=(-u_2,u_1)$ - isso pode ser visto notando que $\tan \theta=-\tan (\theta+90^\circ)$ - como podemos ver no desenho abaixo:
	
	\[\definecolor{zzttqq}{rgb}{0.6,0.2,0.}
	\definecolor{qqwuqq}{rgb}{0.,0.39215686274509803,0.}
	\definecolor{ffqqqq}{rgb}{1.,0.,0.}
	\definecolor{uuuuuu}{rgb}{0.26666666666666666,0.26666666666666666,0.26666666666666666}
	\begin{tikzpicture}[line cap=round,line join=round,>=triangle 45,x=1.0cm,y=1.0cm]
	\begin{axis}[
	x=1.0cm,y=1.0cm,
	axis lines=middle,
	ymajorgrids=true,
	xmajorgrids=true,
	xmin=-3.9800000000000004,
	xmax=6.980000000000004,
	ymin=-1.0200000000000038,
	ymax=6.9799999999999995,
	xtick={-3.0,-2.0,...,6.0},
	ytick={-1.0,0.0,...,6.0},]
	\clip(-3.98,-1.02) rectangle (6.98,6.98);
	\draw[color=ffqqqq,fill=ffqqqq,fill opacity=0.10000000149011612] (0.3794733192202056,0.1897366596101028) -- (0.18973665961010283,0.5692099788303084) -- (-0.1897366596101028,0.3794733192202056) -- (0.,0.) -- cycle; 
	\draw [color=qqwuqq,fill=qqwuqq,fill opacity=0.10000000149011612] (0,0) -- (0.:0.6) arc (0.:26.56505117707799:0.6) -- cycle;
	\draw [color=qqwuqq,fill=qqwuqq,fill opacity=0.10000000149011612] (0,0) -- (0.:1.1) arc (0.:116.56505117707799:1.1) -- cycle;
	\fill[color=zzttqq,fill=zzttqq,fill opacity=0.10000000149011612] (6.,3.) -- (6.,0.) -- (0.,0.) -- cycle;
	\fill[color=zzttqq,fill=zzttqq,fill opacity=0.10000000149011612] (-3.,6.) -- (-3.,0.) -- (0.,0.) -- cycle;
	\draw [shift={(2.02,2.)},color=qqwuqq,fill=qqwuqq,fill opacity=0.10000000149011612] (0,0) -- (-153.43494882292202:0.6) arc (-153.43494882292202:-90.:0.6) -- cycle;
	\draw [shift={(-6.98,5.)},color=qqwuqq,fill=qqwuqq,fill opacity=0.10000000149011612] (0,0) -- (-90.:0.6) arc (-90.:-63.43494882292201:0.6) -- cycle;
	\draw [color=qqwuqq,fill=qqwuqq,fill opacity=0.10000000149011612] (0,0) -- (116.56505117707799:0.6) arc (116.56505117707799:180.:0.6) -- cycle;
	\draw [->] (0.,0.) -- (6.,3.);
	\draw [dash pattern=on 2pt off 2pt,domain=-3.98:6.98] plot(\x,{(-0.--6.*\x)/-3.});
	\draw [->] (0.,0.) -- (-3.,6.);
%	\draw [shift={(0.,0.)},color=qqwuqq] (0.:0.6) arc (0.:26.56505117707799:0.6);
	\draw[color=qqwuqq] (0.510955719470559,0.12062028328736464) -- (0.6569430678907181,0.1550832213694685);
	\draw [color=zzttqq] (6.,3.)-- (6.,0.);
	\draw [color=zzttqq] (6.,0.)-- (0.,0.);
	\draw [color=zzttqq] (0.,0.)-- (6.,3.);
	\draw [color=zzttqq] (-3.,6.)-- (-3.,0.);
	\draw [color=zzttqq] (-3.,0.)-- (0.,0.);
	\draw [color=zzttqq] (0.,0.)-- (-3.,6.);
%	\draw [shift={(6.,3.)},color=qqwuqq] (-153.43494882292202:0.6) arc (-153.43494882292202:-90.:0.6);
	\draw[color=qqwuqq] (5.775024455910556,2.5256467512900755) -- (5.710745729027857,2.3901172516586686);
	\draw[color=qqwuqq] (5.676337678186497,2.586638534163751) -- (5.583862729096923,2.4685352582105358);
%	\draw [shift={(-3.,6.)},color=qqwuqq] (-90.:0.6) arc (-90.:-63.43494882292201:0.6);
	\draw[color=qqwuqq] (-2.879379716712635,5.489044280529442) -- (-2.8449167786305316,5.343056932109282);
%	\draw [shift={(0.,0.)},color=qqwuqq] (116.56505117707799:0.6) arc (116.56505117707799:180.:0.6);
	\draw[color=qqwuqq] (-0.47435324870992435,0.22497554408944334) -- (-0.6098827483413313,0.28925427097214207);
	\draw[color=qqwuqq] (-0.4133614658362491,0.3236623218135027) -- (-0.531464741789463,0.41613727090307534);
	\begin{scriptsize}
	\draw [fill=uuuuuu] (0.,0.) circle (2.0pt);
	\draw[color=black] (3.1,1.23) node {$u$};
	\draw[color=black] (-1.92,2.75) node {$w$};
	\end{scriptsize}
	\end{axis}
	\end{tikzpicture}\]Assim, vamos calcular $\pi_{v,u}$ como sendo igual ao $\lambda_1\in\R$ tal que $v=\lambda_1u+\lambda_2w$, ou seja,
	\begin{align*}
		&\begin{augmatrix}{cc:c}
		u_1&-u_2&v_1\\u_2&u_1&v_2
		\end{augmatrix}\rightsquigarrow\begin{augmatrix}{cc:c}
		1&-u_2/u_1&v_1/u_1\\u_2&u_1&v_2
		\end{augmatrix}\rightsquigarrow\begin{augmatrix}{cc:c}
		1&-u_2/u_1&v_1/u_1\\0&u_1-u_2(-u_2/u_1)&v_2-u_2(v_1/u_1)
		\end{augmatrix}\rightsquigarrow\\&
		\rightsquigarrow\begin{augmatrix}{cc:c}
		1&-u_2/u_1&v_1/u_1\\0&(u_1^1+u_2^2)/u_1&(u_1v_2-u_2v_1)/u_1
		\end{augmatrix}\rightsquigarrow\begin{augmatrix}{cc:c}
		1&-u_2/u_1&v_1/u_1\\0&u_1^1+u_2^2&u_1v_2-u_2v_1
		\end{augmatrix}\rightsquigarrow\\&\rightsquigarrow\begin{augmatrix}{cc:c}
		1&-u_2/u_1&v_1/u_1\\0&1&(u_1v_2-u_2v_1)/(u_1^2+u_2^2)
		\end{augmatrix}\rightsquigarrow\begin{augmatrix}{cc:c}
		1&0&v_1/u_1+(u_2/u_1)(u_1v_2-u_2v_1)/(u_1^2+u_2^2)\\0&1&(u_1v_2-u_2v_1)/(u_1^2+u_2^2)
		\end{augmatrix}
	\end{align*}então nosso $\lambda_1$ vale... $v_1/u_1+(u_2/u_1)(u_1v_2-u_2v_1)/(u_1^2+u_2^2)$?! Ok, sem pânico, vamos resolver isso:
	\begin{align*}
		v_1/u_1+(u_2/u_1)(u_1v_2-u_2v_1)/(u_1^2+u_2^2)&=((v_1/u_1)(u_1^2+u_2^2)+(u_2/u_1)(u_1v_2-u_2v_1))/(u_1^2+u_2^2)\\
		&=(v_1u_1+v_1u_2^2/u_1+u_2v_2-u_2^2v_1/u_1)/(u_1^2+u_2^2)\\
		&=(v_1u_1+v_2u_2)/(u_1^2+u_2^2)
	\end{align*}Pronto! Agora sabemos que $\lambda_1=(v_1u_1+v_2u_2)/(u_1^2+u_2^2)$. Se notarmos que o denominador é simplesmente $\lVert u\rVert^2$, podemos escrever uma expressão fechada para $\pi_{v,u}$:
	\[\pi_{v,u}=(v_1u_1+v_2u_2)/\lVert u\rVert^2.\]
	
	O que significa o termo $v_1u_1+v_2u_2$ que apareceu acima?
	
	\bigskip
	Se pensarmos novamente em vetores como matrizes coluna, não é difícil ver que $v_1u_1+v_2u_2=[v]^t[u]$, onde $^t$ representa a matriz transposta, ou seja:
	\[\begin{pmatrix}
	v_1 &v_2
	\end{pmatrix}\begin{pmatrix}
	u_1\\u_2
	\end{pmatrix}=\begin{pmatrix}
	v_1u_1+v_2u_2
	\end{pmatrix}.\]
	
	Ou seja, usando o conceito de vetores como matrizes nós fomos capazes de inventar uma multiplicação de vetores cujo resultado é um escalar!
\end{ex}

\begin{df}
	Dados dois vetores $v,u\in\R^2$, definimos o \textbf{produto interno (ou escalar) de $v$ e $u$} como sendo o número real $\inprod{v,u}\in\R$ dado por
	\[\inprod{v,u}:=[v]^t[u].\]
\end{df}

\begin{prop}
	Para quaisquer vetores $v,u,w\in\R^2$ e qualquer número real $\lambda\in\R$ valem as seguintes propriedes:
	\begin{itemize}
		\item (Linearidade à esquerda) $\inprod{v+\lambda u,w}=\inprod{v,w}+\lambda\inprod{u,w}$;
		\item (Linearidade à direita) $\inprod{v,u+\lambda w}=\inprod{v,u}+\lambda\inprod{v,w}$;
		\item (Comutatividade) $\inprod{v,u}=\inprod{u,v}$;
		\item (Não-degenerado) $\inprod{v,v}\geq 0$ e a igualdade acontece apenas no caso $v=0$.
	\end{itemize}
\end{prop}

\begin{lemma}
	Para qualquer vetor $v\in\R^2$, temos que $\inprod{v,v}=\lVert v\rVert^2$.
\end{lemma}

\begin{ex}
	Vamos voltar ao exemplo motivador do produto interno: Projeções ortogonais.
	
	Agora, com o produto interno, dados dois vetores $v,u\in\R^2$ podemos escrever a projeção de $v$ em $u$ como sendo $\pi_{v,u}=\inprod{v,u}/\inprod{u,u}$ - dito de outra maneira, $\inprod{v,u}=\pi_{v,u}\inprod{u,u}$.
	
	O que acontece então se escolhermos $v$ ortogonal a $u$, ou seja, $v=\lambda u^\perp$, para algum $\lambda\in \R$? Nesse caso, temos:
	\[\inprod{v,u}=\inprod{\lambda u^\perp,u}=\lambda\inprod{u^\perp,u}=\lambda(-u_2u_1+u_1u_2)=\lambda(0)=0\]ou seja, se $v\perp u$, temos que $\inprod{v,u}=0$ e, portanto, $\pi_{v,u}=0$ - o que condiz com nossa expectativa geométrica!
	
	Por outro lado, se temos algum vetor $w\in\R^2$ tal que $\inprod{w,u}=0$, então temos que $w_1u_1+w_2u_2=0$, ou seja, $w_1/w_2=-u_2/u_1$ e, portanto $w\perp u$ - geométricamente isso é óbvio: Se algum vetor tem projeção de tamanho 0 em $u$, esse vetor só pode ser ortogonal a $u$!
\end{ex}

\begin{lemma}
	Dados dois vetores $v,u\in\R^2$, temos que $v\perp u$ se, e somente se, $\inprod{v,u}=0$.
\end{lemma}

Isso nos permite, então, verificar ortogonalidade sem fazer desenhos, usando apenas coordenadas!

Para encerrar esta subseção, vamos dar um aplicação interessante:
\begin{ex}
	Já vimos que uma reta em $\R^2$ pode ser descrita de duas maneiras distintas: como todas as soluções de uma equação do tipo $ax+by=0$ ou como todos os múltiplos de algum vetor $u$ fixado. Como essas definições se relacionam?
	
	Dito de outra maneira, como encontrar os coeficientes $a,b$ da equação conhecendo apenas o vetor $u$ que gera a reta; e como descobrir qual o vetor gerador de uma reta sabendo apenas os coeficientes da equação?
	
	Simples! Vamos mostrar que uma reta é dada pelo conjunto de soluções reais da equação $ax+by=0$ se, e somente se, é gerada pelo vetor $(a,b)^\perp$.
	
	Um lado é trivial: Considere $r=\{v\in\R^2\mid \exists\lambda\in \R\mbox{ tal que }v=\lambda(a,b)^\perp\}$ e lembre que $(a,b)^\perp=(-b,a)$. Ou seja, todo elemento de $r$ é da forma $\lambda(-b,a)$. Claramente elementos dessa forma são soluções da equação $ax+by=0$.
	
	Por outro lado, se $v$ é solução da equação $ax+by=0$, isso significa que $av_1+bv_2=0$, ou seja, $\inprod{(a,b),v}=0$ - ou seja, $(a,b)\perp v$ e, portanto, $v$ é gerado por $(a,b)^\perp$, como queríamos mostrar.
	
	\tcblower
	Resumindo, quando escrevemos uma reta usando sua \textit{equação} estamos, implicitamente, definindo uma reta como ``o conjunto de todos os vetores ortogonais a um vetor fixado'', e quando escrevemos uma reta com um \textit{subespaço gerado por um vetor} estamos definindo a reta como ``o conjunto de todos os vetores múltiplos de um vetor fixado''.
\end{ex}

%\nocite{*}
%\bibliographystyle{alphanum}
%\bibliography{test} 
\end{document}